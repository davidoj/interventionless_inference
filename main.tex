%!TEX root = main.tex

% Due to file upload restrictions, this archive is missing the figures to render the string diagrams. They have been uploaded separately.
% They should be extracted to the figures folder in this directory

\documentclass{article}
\usepackage{arxiv}

%!TEX root = main.tex


%----------------------------------------------------------------------------------------

%----------------------------------------------------------------------------------------
%	Optional Packages
%----------------------------------------------------------------------------------------
\usepackage{graphicx}
\usepackage[figuresright]{rotating}

\usepackage[font={itshape,raggedright},begintext=``,endtext="]{quoting}
\usepackage{microtype}
\usepackage[framemethod=TikZ]{mdframed}
\usepackage{multirow}
\usepackage{tabulary}
\usepackage{glossaries}
\usepackage{textcomp}
\usepackage{natbib}
% \usepackage[]{todonotes}


\usepackage{hyperref}       % hyperlinks
\PassOptionsToPackage{hyphens}{url}
\usepackage{amsfonts}       % blackboard math symbols
\usepackage{nicefrac}       % compact symbols for 1/2, etc.

%%%%%%%%%%%%%%%%%%%%%%%%%%%%%%%%%%%%%%%%%%%%%%%%%%%%%%%%%%%%%%%%%%%%%%%%%%
% Following additional macros are required to function some 
% functions which are not available in the class used.
%%%%%%%%%%%%%%%%%%%%%%%%%%%%%%%%%%%%%%%%%%%%%%%%%%%%%%%%%%%%%%%%%%%%%%%%%%


% My packages
\usepackage{tikzit}
\input{diagrams.tikzstyles}
\usepackage[mathscr]{euscript}
\usepackage {tikz}
\usetikzlibrary {positioning}
\usetikzlibrary{shapes.misc}
\usetikzlibrary{shapes.geometric}
\usetikzlibrary{calc}
\usetikzlibrary{arrows.meta}
\usetikzlibrary{intersections}
\usepackage{amsthm}
\usepackage{amssymb}

\usepackage{dsfont}
\usepackage{stmaryrd }
\usepackage{csquotes}
\usepackage{wasysym}
\usepackage[shortlabels]{enumitem}
\usepackage{bm}
\usepackage{isomath}
\usepackage{mathtools}
\usepackage{algpseudocode}
\usepackage{algorithm}
\usepackage{multirow}

\usepackage{mathtools}
\mathtoolsset{showonlyrefs}


\hyphenation{un-con-found-ed-ness-like}
\hyphenation{un-con-found-ed-ness}
%----------------------------------------------------------------------------------------
%	MATHS SETTINGS
%----------------------------------------------------------------------------------------


\makeatletter
\newcommand{\newreptheorem}[2]
  {\newtheorem*{rep@#1}{\rep@title}\newenvironment{rep#1}[1]
  {\def\rep@title{#2 \ref*{##1}}\begin{rep@#1}}{\end{rep@#1}}}
\makeatother

\theoremstyle{plain}
\newtheorem{theorem}{Theorem}[section]
\newtheorem{corollary}[theorem]{Corollary}
\newtheorem{lemma}[theorem]{Lemma}
\newtheorem{proposition}[theorem]{Proposition}
\newreptheorem{theorem}{Theorem}
\newreptheorem{lemma}{Lemma}
\newreptheorem{definition}{Definition}

\newtheorem{innercustomthm}{Theorem}
\newenvironment{customthm}[1]
  {\renewcommand\theinnercustomthm{#1}\innercustomthm}
  {\endinnercustomthm}

\theoremstyle{definition}
\newtheorem{definition}[theorem]{Definition}
\newtheorem{example}[theorem]{Example}
\newtheorem{notation}[theorem]{Notation}


% \DeclareMathAlphabet{\mathsfit}{T1}{\sfdefault}{\mddefault}{\sldefault}

\newcommand{\CI}{\mathrel{\text{\scalebox{1.07}{$\perp\mkern-10mu\perp$}}}}
\newcommand{\CII}{\mathrel{\text{\scalebox{1.07}{$\perp\mkern-10mu\perp\mkern-10mu\perp$}}}}
\newcommand{\RV}[1]{\ensuremath{\mathsf{#1}}}
\newcommand{\node}[1]{\ensuremath{\mathsfit{#1}}}
\newcommand{\graph}[1]{\ensuremath{\mathsfbfit{#1}}}
\newcommand{\URV}[1]{\ensuremath{\underline{\RV{#1}}}}
\newcommand{\PA}[2]{\ensuremath{\text{Pa}_{#1}(#2)}}
\newcommand{\ND}[2]{\ensuremath{\text{ND}_{#1}(#2)}}
\newcommand{\CH}[2]{\ensuremath{\text{Ch}_{#1}(#2)}}
\newcommand{\DE}[2]{\ensuremath{\text{De}_{#1}(#2)}}
\newcommand{\ID}[1]{\ensuremath{\text{Id}_{#1}}}
\newcommand{\utimes}{\ensuremath{\underline{\otimes}}}
\newcommand{\prob}[1]{\ensuremath{\mathbb{#1}}}
\newcommand{\disint}[1]{\ensuremath{\overline{\prob{#1}}}}
\newcommand{\kernel}[1]{\ensuremath{\mathbb{#1}}}
\newcommand{\model}[1]{\ensuremath{\mathbb{#1}}}
\newcommand{\diagram}[1]{\ensuremath{\mathscr{#1}}}
\newcommand{\sigalg}[1]{\ensuremath{\mathcal{#1}}}
\newcommand{\vecRV}[1]{\ensuremath{\mathsfbfit{#1}}}
\newcommand{\vecVal}[1]{\ensuremath{\mathbf{#1}}}
\newcommand{\prodSet}[1]{\ensuremath{\mathbf{#1}}}
\newcommand{\indx}[1]{\ensuremath{\mathcal{#1}}}
\newcommand{\nod}[1]{\ensuremath{\mathsfit{#1}}}
\newcommand{\kto}{\ensuremath{\rightarrowtriangle}}
\newcommand{\proc}[1]{\ensuremath{\mathscr{#1}}}
\newcommand{\yields}{\ensuremath{\bowtie}}
\newcommand{\submodel}{\ensuremath{\sqsubset}}
\newcommand{\seedo}[5]{\ensuremath{\model{#1}^{\RV{#3}|\RV{#2}\square\RV{#5}|\RV{#4}}}}
\newcommand{\rseedo}[6]{\ensuremath{\model{#1}^{\RV{#3}|\RV{#2}\framebox{#6}\RV{#5}|\RV{#4}}}}
\newcommand{\set}{\ensuremath{\bowtie}}
\newcommand{\cprod}{\ensuremath{\odot}}
\newcommand{\bigcprod}{\ensuremath{\bigodot}}
\newcommand{\combprod}{\ensuremath{\underline{\cprod}}}
\newcommand{\combbreak}{\ensuremath{\wr}}
\newcommand{\combgap}{\ensuremath{\shortleftarrow}}
\newcommand{\bigcombprod}{\ensuremath{\underline{\bigcprod}}}
\newcommand{\varlessthan}{\ensuremath{\preccurlyeq}}
\algnewcommand\algorithmicassert{\texttt{assert}}
\algnewcommand\Assert[1]{\State \algorithmicassert(#1)}%



\providecommand\longrightarrowRHD{\relbar\joinrel\relbar\joinrel\mathrel\RHD}
\providecommand\longleftarrowRHD{\mathrel\LHD\joinrel\relbar\joinrel\relbar}

\makeatletter
\newcommand*\bigcdot{\mathpalette\bigcdot@{.5}}
\newcommand*\bigcdot@[2]{\mathbin{\vcenter{\hbox{\scalebox{#2}{$\m@th#1\bullet$}}}}}
\makeatother

\tikzset{
    triangle/.style = {regular polygon, regular polygon sides=3 },
    node rotated/.style = {rotate=90},
    border rotated/.style = {shape border rotate=90},
    dist/.style = {triangle,draw,border rotated, inner sep=0pt},
    smalldist/.style = {triangle,draw,border rotated},
    kernel/.style={rectangle,draw,inner sep = 2pt},
    expectation/.style = {triangle,draw,inner sep=0pt,shape border rotate=270},
    copymap/.style = {circle,fill,inner sep=1pt}}

\newcommand\DCI{
    \begin{tikzpicture}[scale=0.35]
    \draw[->] (1,0) -- (0,0);
    \draw (0.6,0) -- (0.6,0.75);
    \draw (0.4,0) -- (0.4,0.75);
    \end{tikzpicture}
}

\newcommand\splitter[1]{%
\begin{tikzpicture}[scale=#1]
\draw (0,-1) -- (0,0);
\draw (0,0) to [bend right] (1,1);
\draw (0,0) to [bend left] (-1,1);
\end{tikzpicture}
}

\newcommand\stopper[1]{%
\begin{tikzpicture}[scale=#1]
\draw[-{Rays [n=8]}] (0,-1) -- (0,0);
\end{tikzpicture}
}

\newcommand\swap[1]{%
\begin{tikzpicture}[scale=#1]
\draw (0,0) to [out=90, in=270] (0.5,1);
\draw (0.5,0) to [out=90,in=270] (0,1);
\end{tikzpicture}
}

\newcommand\source[1]{%
\begin{tikzpicture}[scale=#1]
\path (0,0) node[prob,fill=gray] (P) {};
\draw (P) -- ($(P.east) + (1,0)$);
\end{tikzpicture}
}

\DeclareMathOperator*{\argmax}{arg\,max}
\DeclareMathOperator*{\argmin}{arg\,min}
\DeclareMathOperator*{\arginf}{arg\,inf}
\DeclareMathOperator*{\argsup}{arg\,sup}

\newcommand{\cheng}[1]{ {\color{purple}[{\bf Cheng:~{#1}}]} }


%----------------------------------------------------------------------------------------
%	BOX SETTINGS
%----------------------------------------------------------------------------------------
% from https://texblog.org/2015/09/30/fancy-boxes-for-theorem-lemma-and-proof-with-mdframed/

%----------------------------------------------------------------------------------------
%	MARGIN SETTINGS
%----------------------------------------------------------------------------------------




 \author{ \href{https://orcid.org/0000-0003-3122-9767}{\includegraphics[scale=0.06]{orcid.pdf}\hspace{1mm}David O. Johnston}\thanks{} \\
  Australian National University \\
  \texttt{davidoj@fastmail.com.au}\\
  %% examples of more authors
  \And
  Cheng Soon Ong \\
  Data61\\
  \And
  Robert C. Williamson \\
  Universität Tübingen \\
}

\renewcommand{\shorttitle}{Causal Inference Without Interventions}

%%% Add PDF metadata to help others organize their library
%%% Once the PDF is generated, you can check the metadata with
%%% $ pdfinfo template.pdf
\hypersetup{
pdftitle={Causal Inference Without Interventions},
pdfauthor={David O. Johnston, Cheng Soon Ong, Robert C. Williamson},
pdfkeywords={causal inference, decision theory},
}
  
\begin{document}
 

\title{Causal Inference Without Interventions}

\maketitle

\begin{abstract}
{If we have data and want to evaluate the consequences of an action, we can use a causal model. In this model, actions are represented by structural interventions. However, for many variables the connection between actions and structural interventions is not obvious. This raises the question: if we learn a causal model but do not know how actions correspond to interventions, have we learned anything useful? We show, eventually, that the answer may be ``yes''. We begin with \emph{decision models}, which map options to distributions over outcomes but are otherwise agnostic to causality. We first analyse inference under the assumption of \emph{conditionally independent and identical responses} (CIIR), an assumption of a consistent relationship between pairs of input and output variables. For CIIR models, we prove that arbitrary sequences of input-output pairs can be exchanged given sufficient data. On this basis, we argue for a negative result: it is usually unreasonable to assume causal effects are identified in observational data (this is a common view, but our argument is novel). However, we also show a positive result: \emph{precedent} is a version of the CIIR assumption in which inputs are unobserved. We show that precedent plus the right causal structure and conditional independence implies pairs of observed variables also satisfy CIIR. Here. causal structures encode the assumption of independent causal mechanisms but not intervention operations. That is, even without interventions, causal structure and precedent can be enough to learn about the consequences of your actions.}
\end{abstract}
  \keywords{causal inference \and decision theory} 


\section{Introduction}

Sometimes we want to make decisions supported by data. Structural causal models are a standard framework for addressing this kind of problem. Roughly speaking, there are two different ways to use the structural modelling framework to help make decisions: in some situations, we are confident in our ability to construct the correct structural model from prior information, and the model helps us draw valid deductions from this prior information. On the other hand, in some situations we have little prior information about the correct structural model, and we might attempt \emph{causal discovery} in order to learn it from the given data. We can then use the learned model to draw valid inferences as before.

It is a well-known difficulty of causal inference that the available data may not be sufficient to identify the causal effects of interest. A further difficulty is that the options a decision maker has available often do not have a simple correspondence to interventions on the learned causal graph. This means that, even if a structural model is known, additional knowledge is needed to determine how the consequences of actions should be modeled. This may generally be particularly problematic in settings that call for causal discovery, because in such situations we do not have detailed prior knoweldge about causal relationships.

Consider, for example, an author who wants to know what genre to pick for their next book in order to maximise sales -- science fiction or romance. Suppose that, using a large dataset from a bookseller, they are able to learn a structural model. The exact nature of the model is not critical for the example, but for concreteness we will suppose it contains three variables: sales $\RV{S}$, genre $\RV{G}$ (as judged by the bookseller) and ``covariates'' $\RV{V}$, which is actually a pair of variables: a statistic based on the author's historical sales record, and a statistic based on recent trends in genre sales.

\begin{align}
  \tikzfig{dag_book_model}\label{eq:dag_book_model}
\end{align}

The theory of perfect interventions tells us that, given a distribution $\prob{P}(\RV{S},\RV{G},\RV{V})$ estimated from the data, an intervention on genre will yield the intervened distribution

\begin{align}
  \prob{P}(\RV{S},\RV{V}|\mathrm{do}(\RV{G}=x)) = \prob{P}(\RV{V})\prob{P}(\RV{S}|\RV{V},\RV{G}=x)\label{eq:perfect_intervention}
\end{align}

Suppose that the conditional $\prob{P}(\RV{S}|\RV{V},\RV{G})$ tells us that at any level of author skill, given observed recent sales trends, romance novels are expected to outsell science fiction novels. This seems to suggest that the author is well-advised to write a romance novel.

However, there is a difference between deciding to write a romance novel and actually writing one. Having made the decision, one is not guaranteed to end up having written a romance novel (at least, according to the bookseller) -- or even having written a novel at all. Furthermore, there are an enormous number of different ways to end up with a book classed as ``romance''. The author could, for example, write any novel at all and press booksellers to label it ``romance''. Even if they do write a book that is honestly classed as ``romance'', there are an enormous number of different books that qualify, and many of these different books are likely to have different sales prospects.

Inspection of the model \eqref{eq:dag_book_model} does not tell us which, if any, of these options is a ``canonical intervention on genre''. This problem is somewhat reminiscent of the problem of ambiguous manipulations raised by \citet{spirtes_causal_2004}. Spirtes and Scheines note that when a variable is composed of multiple variables with clear intervention semantics, the composite variable often fails to have clear intervention semantics of its own. \citet[Ch. ~11]{pearl_causality:_2009} states ``there is no way a model can predict the effect of an action unless one specifies which variables are affected by the action and how.''

We could propose is that causal models are not the right tool for judging the consequences of ``coarse'' actions like choosing the genre of a book to write -- perhaps they are meant to give more precise answers to more precise questions. However, on the face of it, it seems reasonable to ask: on the basis of given data, what genre should I aim to write? Perhaps our author could do better by deliberating over more detailed plans, but picking the genre suggested by \eqref{eq:dag_book_model} seems better than picking randomly.

There are many other decision problems for which the relationship between actions and interventions is ambiguous. A notable example is the idea of an intervention on \emph{body mass index} examined by \citep{hernan_does_2008,noauthor_does_2016}. Hernán and Taubman consider the example of different options that are known a priori to affect a person's body mass index, including diet plans, gastric bypass surgery and limb removal. As with actions affecting genre, none of these options seem to be viable candidates for a ``canonical intervention on body mass index'', and opinion shared by other authors \citep{pearl_does_2018,hernanInvitedCommentaryCausal2009,shahar_association_2009}).

In this paper, we investigate an approach to modelling decision problems that takes supporting decisions, rather than representing causal relationships, to be the primary objective. \emph{Decision models} represent actions we can take and consequences we want to evaluate, while representing causal relationships is entirely optional. Such models have been studied previously by numerous authors, including \citet{heckerman_decision-theoretic_1995,dawid_decision-theoretic_2012,dawid_decision-theoretic_2020,lattimore_causal_2019,lattimore_replacing_2019}. An advantage of decision models is that, by construction, it is clear how actions correspond to features of the model. A disadvantage is that, unlike causal models, decision models do not come equipped with inference rules ``by default''. Thus the main objective of this paper is to explore inference rules in decision models.

Independent and identically distributed (IID) sequences of variables are a foundational concept in statistics. The first inference rule we introduce is a generalisation of the IID assumption. Variable sequences with independent and identical responses (IIR) feature pairs of ``input'' and ``output'' variables related by identical stochastic response functions. For IIR decision models with unknown response functions, we prove a generalisation of De Finetti's representation theorem \citep{de_finetti_foresight_1992}. We show that an IIR decision model with an unknown response function is equivalent to an \emph{input-output contractible} decision model (Theorem \ref{th:ciid_rep_kernel}). IO-contractibility is a somewhat complex symmetry of decision models which implies (among other things) the interchangeability of sufficiently large samples of experimental and observational data.

IIR sequences, however, is not a promising inference rule in general. We argue that sufficiently large samples of experimental and observational data are usually not interchangeable, and thus it is usually unreasonable to use IIR decision models. This is not a novel view; it is analogous to the standard view that causal effects are typically not known to be identified in observational data. However, our argument for this conclusion is new, which shows that the decision model framework can at least offer a different perspective on familiar problems.

We then turn to a more promising inference rule. This inference rule has a number of conditions which are somewhat complex, and will be explained in more detail in Section \ref{sec:precedent}. Briefly, the first condition is \emph{precedent}. In the context of our book writing example, this is the assumption that, given either of the available options (write a science fiction or romance book), the distribuiton over sales is given by some unknown reweighting of the observations. The next condition is an observed conditional independence -- in our example, this could be the observation that book sales are independent of the author's identity given the genre and the identified covariates. The final condition is \emph{absolute continuity of conditionals}. This requirement is not easy to explain without the formalism we introduce, and so we will save the explanation for the relevant section. This condition is implied by certain causal structures along with the assumption of \emph{independent causal mechanisms}, such as the following:

\begin{align}
  \tikzfig{dag_book_model_augmented}\label{eq:dag_book_model2}
\end{align}

These three conditions, together, justify assuming that the consequences of (successfully) writing a romance novel are the interventional consequences as given by the original model \eqref{eq:dag_book_model} (Theorem \ref{th:latent_to_observable}). Thus, rather than needing to assume a correspondence between actions and interventions -- which we've argued can be problematic -- we can derive it from the previously mentioned conditions.

The assumption of independent causal mechanisms underpins the assumption of \emph{faithfulness} that facilitates conditional independence based causal discovery algorithms \citep{meek_strong_1995}, as well as many alternative approaches to causal discovery \citep{lemeire_replacing_2013}. Theorem \ref{th:latent_to_observable} also suggests that it could also underpin the interventional interpretation of structural causal models, without needing to introduce interventions as an additional assumption (though this is only a preliminary suggestion, and it may fail to play out under further investigation).

\subsection{Connections to previous work in causal inference}\label{sec:prev_work}

Our approach starts with the assumption that we are trying to model options and consequences, and we do not demand that our models capture any other notion of causation. This assumption motivates the formalism of ``decision models''. This approach is in the tradition of the the \emph{decision theoretic approach to causal inference} that has been formalised in slightly different ways by \citet{heckerman_decision-theoretic_1995} and \citet{dawid_decision-theoretic_2012,dawid_decision-theoretic_2020}. While \citet{lattimore_causal_2019,lattimore_replacing_2019} do not explicitly call their approach ``decision theoretic'', it is also similar to our approach in that options are explicitly incorporated into the model.

\citet{lindley_role_1981} discussed of sequences of exchangeable observations along with ``one more (possibly non-exchangeable) observation''. This construction is very similar to our ``see-do models'' (Definition \ref{def:seedo}). Lindley mentioned the application of this model to questions of causation, but did not explore this deeply due to the perceived difficulty of finding a satisfactory definition of causation. 

There have been a number of works on symmetries in causal inference reminiscent of our work on input-output contractibility. \citet{rubin_causal_2005, imbens_causal_2015} made use of the assumption of models with exchangeable potential outcomes to prove several identification results. \citet{saarela_role_2020}, used graphical causal models to propose \emph{conditional exchangeability}, defined as the exchangeability of the non-intervened causal parents of a target variable under intervention on its remaining parents. Sareela et. al. suggested that this could be interpreted as a symmetry of an experiment involving administering treatments to patients with respect to exchanging the patients in the experiment. \citet{hernan_estimating_2006,hernan_beyond_2012,greenland_identifiability_1986,banerjee_chapter_2017,dawid_decision-theoretic_2020} all discuss similar experimental symmetries. A key difference between all of these causal symmetries and input-output contractibility is that these are all counterfactual symmetries -- they say that, had the experiment been performed differently (say, if different treatments had been administered to different patients), the same model would be used to analyse it. Input-output contractibility, on the other hand, is a data symmetry -- it holds that there are certain transformations that do not affect the choice of appropriate model. Our work on input-output contractibility is also distinguished by the fact that we prove the equivalence of input-output contractible decision models and decision models with conditionally independent and identical responses, which is required in any case where any conditionals that arise ``as a consequence of my actions'' are thought to be identical to conditionals in previously observed data.

A different kind of regularity of causal models is given by the stable unit treatment distribution assumption (SUTDA) in \citet{dawid_decision-theoretic_2020} and the stable unit treatment value assumption (SUTVA) in \citep{rubin_causal_2005}. This regularity is similar to the condition of \emph{locality}, a subassumption of input-output contractibility.

Theorem \ref{th:latent_to_observable} was inspired by \emph{causal inference by invariant prediction} \citep{peters_causal_2016}. While both the assumptions and the conclusions drawn in that work differ from the assumptions and conclusion of Theorem \ref{th:latent_to_observable}, both look for variable pairs $\RV{X}$ and $\RV{Y}$ such that the distribution of $\RV{Y}$ given $\RV{X}$ doesn't change when actions are taken. Unlike Peters et. al., our result does not make use of structural interventions, and the connection to the principle of independent causal mechanisms is original to this work.

Finally, \citet{guoCausalFinettiIdentification2022} have recently generalised De Finetti's theorem to causal graphs in a different manner to the present work and analysed how causal structure may be inferred from independences in exchangeable models.

\subsection{Outline}

Section \ref{sec:tech_prereq} outlines our mathematical framework and provides a brief reference on notation. Section \ref{sec:ccontracibility} introduces decision models with \emph{conditionally independent and identical responses}, a generalisation of conditionally independent and identicallly distributed variables. We then introduce and explains Theorem \ref{th:ciid_rep_kernel}, a ``decision model analogue'' for De Finetti's represention theorem, and finally argue, on the basis of this theorem, that the assumption of conditionally independent and identical responses is often unreasonable.

Section \ref{sec:precedent} looks at a more promising inference rule. It introduces the notion of precedent and then proves Theorem \ref{th:latent_to_observable}, which establishes that precedent along with some additional conditions implies conditionally independent and identical responses. We then examine these additional conditions in more detail, and show that they are supported by structural assumptions, if those assumptions are interpreted as expressing the independence of certain conditional probabilities.

%!TEX root = main.tex

\section{Technical Prerequisites}\label{sec:tech_prereq}

Our approach to causal inference is based on probability theory. Many results and conventions will be familiar to readers, and these are collected in Section \ref{sec:standard_prob}. Because decision models are stochastic functions rather than probability measures (Section \ref{sec:probability_sets}), we make use a generalisation of conditional independence called \emph{extended conditional independence}, explained in Section \ref{sec:eci}.

Section \ref{sec:d_graphs} defines some standard terms relating to directed acyclic graphs, which are likely familiar to anyone acquainted with structural causal models.

\subsection{Probability Theory}\label{sec:standard_prob}

\subsubsection{Measurable spaces}

\begin{definition}[Sigma algebra]
Given a set $A$, a $\sigma$-algebra $\mathcal{A}$ is a collection of subsets of $A$ where
\begin{itemize}
	\item $A\in \mathcal{A}$ and $\emptyset\in \mathcal{A}$
	\item $B\in \mathcal{A}\implies B^{\complement}\in\mathcal{A}$
	\item $\mathcal{A}$ is closed under countable unions: For any countable collection $\{B_i|i\in Z\subset \mathbb{N}\}$ of elements of $\mathcal{A}$, $\cup_{i\in Z}B_i\in \mathcal{A}$ 
\end{itemize}
\end{definition}

\begin{definition}[Measurable space]
A measurable space $(A,\mathcal{A})$ is a set $A$ along with a $\sigma$-algebra $\mathcal{A}$.
\end{definition}

\begin{definition}[Sigma algebra generated by a set of events]
Given a set $A$ and an arbitrary collection of subsets $U\supset\mathscr{P}(A)$, the $\sigma$-algebra generated by $U$, $\sigma(U)$, is the smallest $\sigma$-algebra containing $U$.
\end{definition}

\paragraph{Common $\sigma$ algebras}

For any $A$, $\{\emptyset,A\}$ is a $\sigma$-algebra. In particular, it is the only sigma algebra for any one element set $\{*\}$.

For countable $A$, the power set $\mathscr{P}(A)$ is known as the discrete $\sigma$-algebra.

Given $A$ and a collection of subsets of $B\subset\mathscr{P}(A)$, $\sigma(B)$ is the smallest $\sigma$-algebra containing all the elements of $B$. 

If $A$ is a topological space with open sets $T$, $\mathcal{B}(\mathbb{R}):=\sigma(T)$ is the \emph{Borel $\sigma$-algebra} on $A$.

If $A$ is a separable, completely metrizable topological space, then $(A,\mathcal{B}(A))$ is a \emph{standard measurable set}. All standard measurable sets are isomorphic to either $(\mathbb{R},B(\mathbb{R}))$ or $(C,\mathscr{P}(C))$ for denumerable $C$ \citep[Chap. 1]{cinlar_probability_2011}.

\subsubsection{Probability measures and Markov kernels}

\begin{definition}[Probability measure]\label{def:prob_meas}
Given a measurable space $(E,\sigalg{E})$, a map $\mu:\sigalg{E}\to [0,1]$ is a \emph{probability measure} if
\begin{itemize}
	\item $\mu(E)=1$, $\mu(\emptyset)=0$
	\item Given countable collection $\{A_i\}\subset\mathscr{E}$, $\mu(\cup_{i} A_i) = \sum_i \mu(A_i)$
\end{itemize}
\end{definition}

\begin{definition}[Set of all probability measures]\label{no:prob_meas_set}
The set of all probability measures on $(E,\sigalg{E})$ is written $\Delta(E)$. We equip $\Delta(E)$ with the coarsest $\sigma$-algebra such that the evaluation maps $\eta_B:\nu\mapsto \nu(B)$ are measurable for all $B\in \sigalg{F}$.
\end{definition}

\begin{definition}[Probability space]
A probability space is a triple $(\mu,E,\sigalg{E})$ consisting of a probability measure and a measurable space.
\end{definition}

\begin{definition}[Markov kernel]\label{def:markov_kern}
Given measurable spaces $(E,\sigalg{E})$ and $(F,\sigalg{F})$, a \emph{Markov kernel} or \emph{stochastic function} is a map $\kernel{M}:E\times\sigalg{F}\to [0,1]$ such that
\begin{itemize}
	\item The map $\kernel{M}(A|\cdot):x\mapsto \kernel{M}(A|x)$ is $\sigalg{E}$-measurable for all $A\in \sigalg{F}$
	\item The map $\kernel{M}(\cdot|x):A\mapsto \kernel{M}(A|x)$ is a probability measure on $(F,\sigalg{F})$ for all $x\in E$
\end{itemize}
\end{definition}

\begin{notation}[Signature of a Markov kernel]
Given measurable spaces $(E,\sigalg{E})$ and $(F,\sigalg{F})$ and $\kernel{M}:E\times\sigalg{F}\to [0,1]$, we write the signature of $\kernel{M}:E\kto F$, read ``$\kernel{M}$ maps from $E$ to probability measures on $F$''.
\end{notation}

\begin{definition}[Deterministic Markov kernel]
A \emph{deterministic} Markov kernel $\kernel{A}:E\to \Delta(\mathcal{F})$ is a kernel such that $\kernel{A}_x(B)\in\{0,1\}$ for all $x\in E$, $B\in\mathcal{F}$.
\end{definition}

\paragraph{Common probability measures and Markov kernels}

\begin{definition}[Dirac measure]\label{def:dirac_meas}
The \emph{Dirac measure} $\delta_x\in \Delta(X)$ is a probability measure such that $\delta_x(A)=\llbracket x\in A \rrbracket$
\end{definition}

\begin{definition}[Markov kernel associated with a function]\label{def:mkern_func}
Given measurable $f:(X,\sigalg{X})\to (Y,\sigalg{Y})$, $\kernel{F}_f:X\kto Y$ is the Markov kernel given by $x\mapsto \delta_{f(x)}$
\end{definition}

\begin{definition}[Markov kernel associated with a probability measure]
Given $(X,\sigalg{X})$, a one-element measurable space $(\{*\},\{\{*\},\emptyset\})$ and a probability measure $\mu\in \Delta(X)$, the associated Markov kernel $\kernel{Q}_\mu:\{*\}\kto X$ is the unique Markov kernel $*\mapsto \mu$
\end{definition}

\subsubsection{Variables, conditionals and marginals}

\begin{definition}[Random variable]\label{def:variable}
Given a measurable space $(\Omega,\sigalg{F})$, which we refer to as a \emph{sample space}, and a measurable space of values $(X,\sigalg{X})$, an \emph{$X$-valued random variable on $\Omega$} is a measurable function $\RV{X}:(\Omega,\sigalg{F})\to (X,\sigalg{X})$.
\end{definition}

A sequence of random variables is also a random variable.

\begin{definition}[Sequence of variables]\label{def:seqvar}
Given a sample space $(\Omega,\sigalg{F})$ and two random variables $\RV{X}:(\Omega,\sigalg{F})\to (X,\sigalg{X})$, $\RV{Y}:(\Omega,\sigalg{F})\to (Y,\sigalg{Y})$, $(\RV{X},\RV{Y}):\Omega\to X\times Y$ is the random variable $\omega\mapsto (\RV{X}(\omega),\RV{Y}(\omega))$.
\end{definition}

We define a partial order on random variables such that $\RV{Y}$ is higher than $\RV{X}$ if $\RV{X}$ is given by application of a function to $\RV{Y}$. For example, $\RV{Y}\varlessthan (\RV{W},\RV{Y})$ as $\RV{Y}$ can be obtained by composing a projection with $(\RV{W},\RV{Y})$.

\begin{definition}[Random variables determined by another random variable]\label{def:variable_po}
Given a sample space $(\Omega,\sigalg{F})$ and variables $\RV{X}:\Omega\to X$, $\RV{Y}:\Omega\to Y$, $\RV{X}\varlessthan \RV{Y}$ if there is some $f:Y\to X$ such that $\RV{X}=f\circ \RV{Y}$.
\end{definition}

We use superscripts to specify marginal and conditional distributions, as subscrips (which are a somewhat more common notation) are reserved for specifying options in decision models (Section \ref{sec:probability_sets}).

\begin{definition}[Marginal distribution]\label{def:pushforward}
Given a probability space $(\mu,\Omega,\sigalg{F})$ and a variable $\RV{X}:\Omega\to (X,\sigalg{X})$, the \emph{marginal distribution} of $\RV{X}$ with respect to $\mu$, $\mu^{\RV{X}}:\sigalg{X}\to [0,1]$ by $\mu^{\RV{X}}(A):=\mu(\RV{X}^{-1}(A))$ for any $A\in \sigalg{X}$.
\end{definition}

\begin{definition}[Conditional distribution]\label{def:disint}
Given a probability space $(\mu,\Omega,\sigalg{F})$ and variables $\RV{X}:\Omega\to X$, $\RV{Y}:\Omega\to Y$, the \emph{conditional distribution} of $\RV{Y}$ given $\RV{X}$ is any Markov kernel $\mu^{\RV{Y}|\RV{X}}:X\kto Y$ such that
\begin{align}
	\mu^{\RV{XY}}(A\times B)&=\int_{A} \mu^{\RV{Y}|\RV{X}}(B|x) \mathrm{d}\mu^{\RV{X}}(x) &\forall A\in \sigalg{X}, B\in \sigalg{Y}
\end{align}
\end{definition}

\begin{definition}[Trivial variable]\label{no:single_valued}
We let $*$ stand for a single-valued variable $*:\Omega\to \{*\}$.
\end{definition}

% \subsubsection{Markov kernel product notation}\label{ssec:product_notation}

% Three pairwise \emph{product} operations involving Markov kernels can be defined: measure-kernel products, kernel-kernel products and kernel-function products. These are analagous to row vector-matrix products, matrix-matrix products and matrix-column vector products respectively.

% \begin{definition}[Measure-kernel product]
% Given $\mu\in \Delta(\mathcal{X})$ and $\kernel{M}:X\kto Y$, the \emph{measure-kernel product} $\mu\kernel{M}\in \Delta(Y)$ is given by
% \begin{align}
% \mu\kernel{M} (A) := \int_X \kernel{M}(A|x) \mu(\mathrm{d}x)
% \end{align}
% for all $A\in \sigalg{Y}$.
% \end{definition}

% \begin{definition}[Kernel-kernel product]\label{def:kproduct}
% Given $\kernel{M}:X\kto Y$ and $\kernel{N}:Y\kto Z$, the \emph{kernel-kernel product} $\kernel{M}\kernel{N}:X\kto Z$ is given by
% \begin{align}
% \kernel{MN} (A|x) := \int_Y \kernel{N}(A|x) \kernel{M}(\mathrm{d}y|x)
% \end{align}
% for all $A\in \sigalg{Z}$, $x\in X$.
% \end{definition}

% \begin{definition}[Kernel-function product]
% Given $\kernel{M}:X\kto Y$ and $f:Y\to Z$, the \emph{kernel-function product} $\kernel{M}f:X\to Z$ is given by
% \begin{align}
% \kernel{M}f (x) := \int_Y f(y)\kernel{N}(\mathrm{d}y|x)
% \end{align}
% for all $x\in X$.
% \end{definition}

% \begin{definition}[Tensor product]
% Given $\kernel{M}:X\kto Y$ and $\kernel{L}:W\kto Z$, the tensor product $\kernel{M}\otimes\kernel{N}:X\times W\kto Y\times Z$ is given by
% \begin{align}
% 	(\kernel{M}\otimes\kernel{L})(A\times B|x,w):=\kernel{M}(A|x)\kernel{L}(B|w)
% \end{align}
% For all $x\in X$, $w\in W$, $A\in \sigalg{Y}$ and $B\in \sigalg{Z}$.
% \end{definition}

% All products are associative \citep[Chapter 1]{cinlar_probability_2011}.

\subsection{Decision models}\label{sec:probability_sets}

A \emph{decision model} is a Markov kernel $\prob{P}_\cdot$ from an option set $(C,\sigalg{C})$ to a sample space $(\Omega,\sigalg{F})$.

\begin{definition}[Decision model]\label{def:dec_model}
A decision model is a triple $(\prob{P}_\cdot, (\Omega,\sigalg{F}), (C,\sigalg{C}))$ where $\prob{P}_\cdot:C\kto \Omega$ is a Markov kernel, $(\Omega,\sigalg{F})$ is the sample space and $(C,\sigalg{C})$ is the option set.
\end{definition}

For an option $\alpha\in C$, we say $\prob{P}_\alpha$ is the model $\prob{P}_\cdot$ evaluated at $\alpha$.

\begin{definition}[Almost sure equality]
Given a decision model $(\prob{P}_\cdot, (\Omega,\sigalg{F}), (C,\sigalg{C}))$ and random variables $\RV{X}:\Omega\to X$, $\RV{Y}:\Omega\to Y$, two Markov kernels $\kernel{K}:X\kto Y$ and $\kernel{L}:X\kto Y$ are $\prob{P}_\cdot,\RV{X},\RV{Y}$-almost surely equal if for all $A\in\sigalg{X}$, $B\in \sigalg{Y}$, $\alpha\in C$
\begin{align}
    \int_A \kernel{K}(B|x)\prob{P}_\alpha^{\RV{X}}(\mathrm{d}x) = \int_A\kernel{L}(B|x)\prob{P}_\alpha^{\RV{X}}(\mathrm{d}x)
\end{align}
we write this as $\kernel{K}\overset{\prob{P}_\cdot^{\RV{X}}}{\cong}\kernel{L}$.
\end{definition}

Equivalently, $\kernel{K}$ and $\kernel{L}$ are almost surely equal if the set $C:\{x|\exists B\in\sigalg{Y}:\kernel{K}(B|x)\neq\kernel{L}(B|x)\}$ has measure 0 with respect to $\prob{P}_\alpha^{\RV{X}}$ for all $\alpha\in C$.

\subsection{Extended conditional independence}\label{sec:eci}

Because decision models aren't standard probability spaces, we need some version of conditional independence for decision models. Such a notion has already been worked out in some detail: it is the idea of \emph{extended conditional independence} defined in \citet{constantinou_extended_2017}. Extended conditional independence is substantially more general than we need for our purposes, and in fact we only consider two special cases of it. However, we still make use of the notational convention introduced in that paper.

We will first define regular conditional independence. We define it in terms of a having a conditional that ``ignores one of its inputs'', which, provided conditional probabilities exist, is equivalent to other common definitions \cite{fritz_synthetic_2020}.

\begin{definition}[Conditional independence]\label{def:ci}
Given a decision model $(\prob{P}_\cdot, (\Omega,\sigalg{F}), (C,\sigalg{C}))$, variables $\RV{X},\RV{Y},\RV{Z}$ and fixing some $\alpha\in C$, we say $\RV{Y}$ is conditionally independent of $\RV{X}$ given $\RV{Z}$, written $\RV{Y}\CI_{\model{P}_{\alpha}}\RV{X}|\RV{Z}$, if there exists some $\kernel{K}:Z\kto Y$ such that
\begin{align}
    \prob{P}^{\RV{Y}|\RV{XZ}}(A|x,z) &\overset{\prob{P}_\alpha^{\RV{XZ}}}{\cong} \prob{K}(A|z)&\forall A\in \sigalg{Y}
\end{align}
\end{definition}

Extended conditional independence as introduced by \citet{constantinou_extended_2017} is defined using pairs of  ``complementary nonstochastic variables'' on the option set $C$.

\begin{definition}[Nonstochastic variable]\label{def:nonstoc_var}
Given a decision model $(\prob{P}_\cdot,(\Omega,\sigalg{F}),(C,\sigalg{C}))$ and a measurable set $(X,\sigalg{X})$, a nonstochastic variable is a measurable function $\phi:C\to X$.
\end{definition}

\begin{definition}[Complementary nonstochastic variables]\label{def:comp_var}
A pair of nonstochastic variables $\phi$ and $\xi$ are complementary if the pair $(\phi,\xi)$ is invertible.
\end{definition}

Unlike \citet{constantinou_extended_2017}, we limit ourselves to a definition of extended conditional independence where regular uniform conditional probabilities exist. Our definition is otherwise identical.

\begin{definition}[Extended conditional independence]\label{def:eci_orig}
Given a probability set $\prob{P}_C$, variables $\RV{X}$, $\RV{Y}$ and $\RV{Z}$ and complementary nonstochastic variables $\phi$ and $\xi$, the extended conditional independence $\RV{Y}\CI^e_{\prob{P}_\cdot} \RV{X} \phi|\RV{Z} \xi$ holds if for each $a\in \xi(C)$, $\alpha,\alpha'\in \xi^{-1}(a)$,
\begin{align}
    \prob{P}_{\alpha}^{\RV{Y}|\RV{XZ}} &\overset{\prob{P}_{\cdot}}{\cong} \prob{P}_{\alpha'}^{\RV{Y}|\RV{XZ}}
\end{align}
and for all $\alpha\in C$
\begin{align}
    \prob{P}_{\alpha}^{\RV{Y}|\RV{XZ}}(A|x,z) &\overset{\prob{P}_{\alpha}}{\cong} \prob{P}_{\alpha}^{\RV{Y}|\RV{Z}}(A|z)&\forall A\in \sigalg{Y},(x,z)\in X\times Z\label{eq:eci}
\end{align}
\end{definition}

In this work we only ever consider the complimentary pair $(\mathrm{id}_C,*)$ where $*$ is the trivial variable $\cdot \mapsto *$., in which case extended conditional independence breaks down into two special cases: \emph{global conditional independence} and \emph{uniform conditional independence}. The former can be understood as ``conditional independence for every $\alpha\in C$'', while the latter means ``conditional independence for every $\alpha\in C$, and moreover conditionally independent of $\mathrm{id}_C$''.

\begin{definition}[Global conditional independence]\label{def:eci_glob}
Given a decision model $(\prob{P}_\cdot, (\Omega,\sigalg{F}), (C,\sigalg{C}))$ and variables $\RV{X}$, $\RV{Y}$ and $\RV{Z}$, $\RV{Y}$ is globally independent of $\RV{X}$ given $\RV{Z}$, written $\RV{Y}\CI^e_{\prob{P}_\cdot} \RV{X} |(\RV{Z}, \mathrm{id}_C)$ if for each $\alpha\in C$
\begin{align}
    \prob{P}_{\alpha}^{\RV{Y}|\RV{XZ}}(A|x,z) &\overset{\prob{P}_{\alpha}^{\RV{XZ}}}{\cong} \prob{P}_{\alpha}^{\RV{Y}|\RV{Z}}(A|z)&\forall A\in \sigalg{Y},(x,z)\in X\times Z\label{eq:gci}
\end{align}
\end{definition}

\begin{definition}[Uniform conditional independence]\label{def:eci}
Given a decision model $(\prob{P}_\cdot, (\Omega,\sigalg{F}), (C,\sigalg{C}))$ and variables $\RV{X}$, $\RV{Y}$ and $\RV{Z}$, the uniform conditional independence $\RV{Y}\CI^e_{\prob{P}_\cdot} (\RV{X}, \mathrm{id}_C)|\RV{Z}$ holds if $\RV{Y}\CI^e_{\prob{P}_\cdot} \RV{X} |(\RV{Z}, \mathrm{id}_C)$ and furthermore for all $\alpha,\alpha'\in C$
\begin{align}
    \prob{P}_\alpha^{\RV{Y}|\RV{XZ}} &\overset{\prob{P}_\alpha^{\RV{XZ}}}{\cong} \prob{P}_{\alpha'}^{\RV{Y}|\RV{XZ}}\label{eq:uci}
\end{align}
\end{definition}

For countable sets $C$ we can reason with collections of extended conditional independence statements as if they were regular conditional independence statements. In the following rules, $\phi$ and $\xi$ refer to complementary variables on the set $C$ (see \citet{constantinou_extended_2017} for details), but for our purposes we only consider the cases where either $\phi=\mathrm{id}_C$ and $\xi=*$ or $\phi=*$ and $\xi=\mathrm{id}_C$. In the rest of this text, we will omit the trivial variable from extended conditional independence statements.

\begin{enumerate}
    \item Symmetry: $\RV{X}\CI_{\prob{P}_{\cdot}}^e (\RV{Y}, \phi)|(\RV{Z}, \xi)$ iff $\RV{Y}\CI_{\prob{P}_{\cdot}}^e (\RV{X}, \phi)|(\RV{Z},\xi)$
    \item $\RV{X}\CI_{\prob{P}_{\cdot}}^e (\RV{Y}, \mathrm{id}_C)| (\RV{Y}, \mathrm{id}_C)$
    \item Decomposition: $\RV{X}\CI_{\prob{P}_{\cdot}}^e (\RV{Y}, \phi)|\RV{W}\xi$ and $\RV{Z}\varlessthan\RV{Y}$ implies $\RV{X}\CI_{\prob{P}_{\cdot}}^e(\RV{Z},\phi)|(\RV{W},\xi)$
    \item Weak union:
    \begin{enumerate}
     	\item $\RV{X}\CI^e_{\prob{P}_{\cdot}} (\RV{Y}, \phi)|(\RV{W}, \xi)$ and $\RV{Z}\varlessthan \RV{Y}$ implies $\RV{X}\CI_{\prob{P}_{\cdot}}^e(\RV{Y},\phi)|(\RV{Z},\RV{W}, \xi)$
     	\item $\RV{X}\CI_{\prob{P}_{\cdot}}^e \RV{Y} \mathrm{id}_{C}|\RV{W}$ implies $\RV{X}\CI_{\prob{P}_{\cdot}}^e\RV{Y}|(\RV{W},\mathrm{id}_C)$
     \end{enumerate} 
    \item Contraction: $\RV{X}\CI_{\prob{P}_{\cdot}}^e(\RV{Z},phi)|(\RV{W},\xi)$ and $\RV{X}\CI_{\prob{P}_{\cdot}}^e(\RV{Y},\phi)|(\RV{Z},\RV{W})\xi$ implies $\RV{X}\CI_{\prob{P}_{\cdot}}^e(\RV{Y},\RV{Z},\phi)|(\RV{W},\xi)$
\end{enumerate} 

If we have the extended conditional independence $\RV{Y}\CI^e_{\prob{P}_\cdot} \mathrm{id}_C | \RV{X}$, then by definition for all $\alpha,\alpha'\in C$ we have $\prob{P}_\alpha^{\RV{Y}|\RV{X}}=\prob{P}_{\alpha'}^{\RV{Y}|\RV{X}}$. In this case, we use the notation $\prob{P}_C^{\RV{Y}|\RV{X}}$ to indicate that the conditional distribution does not depend on the choice of $\alpha$

\begin{definition}[Uniform conditional distribution]\label{def:uci}
Given a decision model $(\prob{P}_\cdot, (\Omega,\sigalg{F}), (C,\sigalg{C}))$ and variables $\RV{X}$, $\RV{Y}$, if $\RV{Y}\CI^e_{\prob{P}_\cdot} \mathrm{id}_C | \RV{X}$ then
\begin{align}
    \prob{P}_C^{\RV{Y}|\RV{X}} &= \prob{P}_\alpha^{\RV{Y}|\RV{X}}
\end{align}
for any $\alpha\in C$. If $\RV{Y}\not \CI^e_{\prob{P}_\cdot} \mathrm{id}_C | \RV{X}$ then $\prob{P}_C^{\RV{Y}|\RV{X}}$ is not defined.
\end{definition}

\subsection{Directed graphs}\label{sec:d_graphs}


\begin{definition}[Directed graph]
A directed acyclic graph $\mathcal{G}$ is a set of nodes $\mathcal{V}$ and a set of edges $\mathcal{E}$. Each edge is an ordered pair of nodes $(V_i,V_j)\in \mathcal{V}^2$, with $V_i$ the source and $V_j$ the destination. An acyclic graph must have no directed path that begins and ends at $V_i$ for any $V_i\in\mathcal{V}$.
\end{definition}

\begin{definition}[Directed path]
Given a graph $\mathcal{G}=(\mathcal{V},\mathcal{E})$, a directed path is a sequence of edges $((V^1_{k},V^2_{k}))_{k\in [n]}$ from $\mathcal{E}$ such that for any $k\in [n]$, $V^2_k=V^1_{k+1}$. A directed path begins as $V^1_1$ and ends at $V^2_k$.
\end{definition}

\begin{definition}[Directed acyclic graph]
A directed graph $\mathcal{G}$ is a directed acyclic graph if it contains no directed paths beginning and ending at the same node.
\end{definition}

\begin{definition}[Parents]
Given a graph $\mathcal{G}=(\mathcal{V},\mathcal{E})$, the parents of a node $V_i$ are all the nodes $V_j$ such that there is an edge $(V_j, V_i)\in \mathcal{E}$: $\mathrm{Pa}(V_i)=\{V_j|(V_j,V_i)\in \sigalg{E}\}$.
\end{definition}

\begin{definition}[Model graph association]\label{def:mga}
Given a set of variables $(\RV{V}_i)_{i\in [k]}$, an \emph{associated} directed acyclic graph $\mathcal{G}=(\mathcal{V},\mathcal{E})$ is a graph with a node $V_i$ for each variable $\RV{V}_i$. We define the parents of a variable via this association: $\mathrm{Pa}_\mathcal{G}(\RV{V}_i)=\{\RV{V}_j|(V_j,V_i)\in \sigalg{E}\}$.
\end{definition}

\todo{parameters}
%!TEX root = main.tex

\section[Identical responses]{Inferring consequences when observations and consequences share identical responses}\label{sec:evaluating_decisions}

Recall to the example discussed in the introduction: an author wants to choose the genre of a book they will write. There, we proposed a causal model that predicted that the distribution of sales conditional on genre, the author's historical sales success and global trends sales does not change under intervention on genre.

We can postulate a stronger notion of invariance: not only does the distribution of sales conditional on genre and the covariates not change under intervention -- whatever that means, precisely -- but it doesn't change under \emph{any} of the author's available actions. In this case, the author can reason as follows: while they don't know exactly what consequences deciding to write a romance novel ($\alpha_{\mathrm{romance}}$) will have, they know that under this choice they are more likely to produce a romance novel vs deciding to write a science fiction novel ($\alpha_{\mathrm{sf}}$). They also know that their choice will not affect their history of book sales, nor recent global trends in sales. Thus, despite their uncertainty over the details of the consequences, choosing to write a romance novel will lead to more sales in expectation.

This assumption allows the author to identify ``choosing to write a romance novel'' with a causal intervention on genre. It cannot always hold -- the author could, after all, choose to write the worst romance novel that they can imagine, which should surely be expected to sell worse than a novel that they attempt to make good. We will discuss the possibility of unusual or pathological actions further in Section \ref{sec:precedent}. If unusual plans present an obstacle to inference, the author may decide not to consider them. Even if the author avoids considering unusual plans, it is still unclear that it is reasonable to \emph{assume} that conditionals cannot change in this manner. Nevertheless, it forms the basis of a simple and important inference rule. For example, in the interventionist framework, conditionals invariant to intervention play a role in every identification result \citep{richardson_nested_2017}. For this reason, we think it is still worth studying.

Here is a mathematical formulation of our assumption: the author has a decision model $(\prob{P}_{\cdot}, (C,\sigalg{C}), (\Omega,\sigalg{F}))$ together with a sequence of random variables $(\RV{S}_i, \RV{G}_i, \RV{V}_i)_{i\in [m]\cup \{q\}}$ where the indices $[m]$ refer to the books observed in the dataset so far, and the special index $q$ refers to the book the author is hoping to write. The assumption we are discussing is that there is some unknown stochastic function, which we will call $\RV{H}$ (we represent unknown quantities with random variables), such that for all $\alpha$, $i$, $j$
\begin{align}
    \prob{P}_\alpha^{\RV{S}_i|\RV{G}_i\RV{V}_j\RV{H}} &= \prob{P}_\alpha^{\RV{S_j}|\RV{G}_j\RV{V}_j\RV{H}} = \RV{H}
\end{align}
we call $\RV{H}$ a ``response function'' because it models the way $\RV{S}_i$ responds to $\RV{G}_i$ and $\RV{V}_i$. We also assume that once we know $\RV{H}$, we have nothing more to learn about this response from observing more $(\RV{S}_i, \RV{G}_i, \RV{V}_i)$ triples. That is, we assume $\RV{S}_i \CI (\RV{S}_{\{i\}^\complement}, \RV{V}_{\{i\}^\complement}, \RV{G}_{\{i\}^\complement}) | (\RV{G}_i, \RV{V}_i, \RV{H}, \RV{C})$.

Thus this assumption has two components: first, conditional on $\RV{H}$, the response of every $\RV{S}_i$ to $(\RV{G}_i, \RV{V}_i)$ is identical regardless of $i$ and secondly, conditional on $\RV{H}$, $\RV{S}_i$ is independent of other triples $(\RV{S}_i, \RV{G}_i, \RV{V}_i)$. It is the assumption of \emph{conditionally independent and identical responses} (CIIR for short).

Is it ever reasonable to assume CIIR? Sometimes it might be -- perhaps the system being modelled has been deliberately engineered for regularity. A switch reliably turns on a light if you flick it, and a function in a piece of code reliably returns the same result given the same input. Book sales or human health are not examples of systems like this, however. Alternatively, we might suppose that everything of interest follows classical physics, and so in principle if we knew the right details of the initial condition in each case, we would observe a regular relationship between initial conditions and final states. This does not tell us whether any particular collection of variables should be expected to demonstrate a fixed response relationship.

Instead of appealing to our prior knowledge of mechanisms as we do in the case of engineered systems, we could try to appeal to knowledge of symmetries of the problem. The inspiration for this approach comes from De Finetti's work on Bayesian probabilistic inference. De Finetti, observing that many statistical models assumed a sequence of independent and identically distributed (IID) random variables conditional on an unknown ``true parameter'', argued that this assumption usually did not make much sense: it is often hard to convincingly argue for the existence of a mechanism that behaves in this manner. Instead, he suggested, we could appeal to the symmetry of our knowledge about a problem. Suppose, as far as we are concerned, the problem is not changed in any important way by permuting the measurements we have taken. De Finetti showed that the class of probability models with this symmetry (called \emph{exchangeability}) is equivalent to the class of models of IID random variables conditional on an unknown parameter \citep{de_finetti_foresight_1992}. De Finetti's argument is controversial -- for example, \citet{walley_statistical_1991} has argued that it does not under imprecise probability models. Nevertheless, it gave us substantial insight into the assumptions underpinning IID statistical models, and a similar result may help us understand CIIR decision models.

That is the line of thinking we pursue in this section. In particular, we prove a similar result to De Finetti's -- that result is that the class of CIIR decision models is equivalent to decision models with a symmetry we call \emph{input-output contractibility} (or IO contractibility). However, we are unable to come up with any new arguments for assuming CIIR on the basis of this result. IO contractibility is a less intuitive property than exchangeability, so we have overlooked some possibilities. We are able to provide an additional argument \emph{against} assumiming CIIR in general. IO contractibility implies that, after having seen infinite data, input-output pairs can be exchanged. Thus, if the author assumes CIIR they must accept that these two problems are identical:
\begin{itemize}
    \item Observe an infinite dataset of book sales and predict the sales of one more book in the dataset after observing its genre and covariates
    \item Observe the same infinite dataset of book sales and predict the sales of their own book after observing its genre and covariates
\end{itemize}
However, we think that these problems will generally not be identical -- the second problem will generally be harder than the first. Alternatively, suppose that instead of choosing a genre for 1 book, the author chooses a genre for $1000$ books, which are all launched at the same moment (so sales history & genre trends are held constant). In this case, the author is perfectly happy to disregard the obsrvations of 999 of their own books in order to predict the sales of their 1000th. The CIIR assumption is very dogmatic -- it doesn't just treat responses as identical before any evidence to the contrary is obtained, it even continues to insist they are the same once such evidence begins to accumulate. This assumption is far too strong for the many cases where we do not have clear reasons to assume CIIR. We need weaker assumptions if we are to consider them plausible, and we will discuss one candidate in the next section.

\subsection[CIIR sequences]{Conditionally independent and identical responses}\label{sec:response_functions}

We now turn to the formal treatment of the CIIR assumption, and its equivalence to IO contractibility. First, we define sequential input-output models as a shorthand for a decision model along with a sequence of random variable pairs.

\begin{definition}[Sequential input-output model]\label{def:seq_io}
A decision model $(\prob{P}_{\cdot}, (C,\sigalg{C}), (\Omega,\sigalg{F}))$ and two sequences of variables $\RV{Y}:=(\RV{Y}_i)_{i\in \mathbb{N}}$ and $\RV{D}:=(\RV{D}_i)_{i\in\mathbb{N}}$ is a sequential input-output model, which we specify with the shorthand $(\prob{P}_\cdot,\RV{D},\RV{Y})$.
\end{definition}

Sequential input-output pairs $(\RV{D}_i,\RV{Y}_i)_{i\in \mathbb{N}}$ share conditionally independent and identical responses if there is an unknown stochastic function $\RV{H}$ taking values in $\Delta(Y)^D$ -- that is, in the set of maps from $D$ to probability distributions over $Y$ -- such that every output $\RV{Y}_i$ ``responds to'' $\RV{D}_i$ according to the same $\RV{H}$. Note that this assumption does not specify any relationship between options and the behaviour of inputs $\RV{D}_i$. The CIIR assumption is useful when the decision maker has some knowledge about how to control some inputs, but whether or not they have such knowledge is a separate question not relevant to this analysis.

\begin{definition}[Conditionally independent and identical responses]\label{def:cii_rf}
Given a sequential input-output model $(\prob{P}_\cdot,\RV{D},\RV{Y})$ along with some random variable $\RV{V}$, the $(\RV{D}_i,\RV{Y}_i)$ pairs are related by \emph{independent and identical responses conditional on} $\RV{H}$ if for all $i$, $\RV{Y}_i\CI (\RV{D}_{[1,i)},\RV{Y}_{[1,i)})|(\RV{D}_i,\RV{H},\RV{C})$ and $\prob{P}_\alpha^{\RV{Y}_i|\RV{D}_i\RV{H}}\overset{C}{\cong}\prob{P}_\alpha^{\RV{Y}_j|\RV{D}_j\RV{H}}$ for all $i,j$.
\end{definition}

% Definition \ref{def:cii_rf} asserts that there are versions of all the conditional distributions $\prob{P}_\alpha^{\RV{Y}_i|\RV{D}_i\RV{V}}$ that are pairwise almost surely equal (recall that $\overset{C}{\cong}$ is short hand for ``almost surely equal for all $\alpha\in C$''). Theorem \ref{th:repr_cond} shows that this is sufficient for the existence of a single conditional distribution that is a version of $\prob{P}_\alpha^{\RV{Y}_i|\RV{D}_i\RV{V}}$ for all $i$.

% % \begin{theorem}[Existence of representative conditional distribution]\label{th:repr_cond}
% % Given a sequential input-output model $(\prob{P}_\cdot,\RV{D},\RV{Y})$, if the $(\RV{D}_i,\RV{Y}_i)$ pairs are related by independent and identical responses conditional on $\RV{V}$, define $\prob{P}_\alpha^{\RV{Y}_i|\RV{X}_i\RV{H}}(\cdot|h)\overset{C}{\cong} h^Y_X$ for all $i$.

% % We refer to the function $\RV{H}^Y_X:h\mapsto h^Y_X$ as a \emph{representative conditional distribution}.
% % \end{theorem}

% % \begin{proof}
% % Fix $v$, $\alpha$ and take $h^Y_{X,i}:=\prob{P}_\alpha^{\RV{Y}_i|\RV{X}_i\RV{H}}(\cdot|\cdot,h)$ to be an arbitrary version of the conditional distribution for all $i$.

% % For $i,j\in \mathbb{N}$, take $S_{ij} := \{x|h^Y_{x,i}\text{ is not a version of }\prob{P}_\alpha^{\RV{Y}_j|\RV{X}_j\RV{H}}(\cdot|\cdot,h)\}$. Note that $S_i:= \cup_{j\in \mathbb{N}} S_{ij}$ is a countable union of sets of $\prob{P}_\alpha^{\RV{X}_i|\RV{H}}(\cdot|h)$-measure 0, hence is also a set of $\prob{P}_\alpha^{\RV{X}_i|\RV{H}}(\cdot|h)$-measure 0.

% % Define
% % \begin{align}
% %     h^Y_X(A|x) := \sum_{i\in \mathh{N}} \mathds{1}_{S_i^\complement\setminus \cup_{j\in[i]}S_j^\complement}(x) h^Y_{X,i}(A|x)
% % \end{align}

% % By construction, $h^Y_X$ differs from each $h^Y_{X,i}$ by a measure 0 set with respect to $\prob{P}_\alpha^{\RV{X}_i|\RV{H}}(\cdot|h)$. Hence it is a version of $\prob{P}_\alpha^{\RV{Y}_i|\RV{X}_i\RV{H}}(\cdot|h)$ for every $i$.
% % \end{proof}

Note that in general, we only require the outputs $\RV{Y}_i$ be independent of \emph{previous} inputs and outputs conditional on $\RV{H}$ and $\RV{D}_i$. The reason for this is that, if we suppose that the variable indices match the time-ordering of variables, it's possible that $\RV{D}_i$ is chosen based on previous data (i.e. some author in the dataset might have chosen the genre of \emph{their} book based on their previous observations). This means that, in general, there may be relationships between $\RV{D}_j$ and $\RV{Y}_i$ for $j>i$ even after conditioning on $\RV{D}_i$ and $\RV{H}$. However, for our purposes we will use a stronger assumption that we call \emph{weak data-independence}, which means that conditional on $\RV{H}$ and past inputs $\RV{D}_{[1,i]}$, $\RV{Y}_i$ is also independent of all future inputs. Generalising our result to data-dependent inputs is an open question.

\begin{definition}[Weakly data-independent]\label{def:weak_di}
A sequential input-output model $(\prob{P}_C,\RV{D},\RV{Y})$ with independent and identical responses conditional on $\RV{H}$ is weakly data-independent if $\RV{Y}_i\CI \RV{D}_{\{i\}^\complement}|(\RV{D}_{i},\RV{H},\RV{C})$.
\end{definition}

\subsection[Conditional probability symmetries]{Symmetries of sequential conditional probabilities}\label{sec:ccontracibility}

Given the previously mentioned sequences $\RV{D}$ and $\RV{Y}$, the decision maker has for each option $\alpha\in C$ a conditional probability $\prob{P}_\alpha^{\RV{Y}|\RV{D}}$ (note the absence of $\RV{H}$ in the conditioning arguments). In general, we don't have $\RV{Y}_i\CI (\RV{Y}_{\{i\}^\complement},\RV{D}_{\{i\}^\complement}) | \RV{D}_i$, because additional examples of the input-output relationship enable the decision maker to learn about the ``true form'' of the relationship in more detail. However, one way that they might consider one pair $(\RV{D}_i,\RV{Y}_i)$ to be ``essentially the same'' as $(\RV{D}_j, \RV{Y}_j)$ is if there is no effective difference made by swapping the pairs, so $\prob{P}_\alpha^{\RV{Y}_i\RV{Y}_j|\RV{D}_i\RV{D}_j}$ is essentially the same as $\prob{P}_\alpha^{\RV{Y}_j\RV{Y}_i|\RV{D}_j\RV{D}_i}$. Or, more generally, given any permutation $\rho:\mathbb{N}\to \mathbb{N}$, define $\RV{Y}_\rho:=(\RV{Y}_{\rho(i)})_{i\in\mathbb{N}}$ and $\RV{D}_{\rho}$ similarly. Then we could propose a symmetry such that for all $\alpha$, $\rho$

\begin{align}
    \prob{P}_\alpha^{\RV{Y}|\RV{D}} &= \prob{P}_\alpha^{\RV{Y}_\rho|\RV{D}_\rho}
\end{align}

This symmetry is reminiscent of exchangeability, and in Theorem \ref{lem:exch_prod_ciid} we show that it implies that the $(\RV{D}_i,\RV{Y}_i)$ share conditionally independent and identical responses. However, the converse is not true. The reason for this is that we assume the responses are identical, but we don't require that the inputs are identical. Thus it may be the case that we learn more from $\RV{D}_j$ than we do from $\RV{D}_i$. Example \ref{ex:no_swapping} shows this in more detail.

\begin{example}\label{ex:no_swapping}
Suppose there is a machine with two arms $D=\{0,1\}$, one of which always pays out \$100 and the other that pays out nothing. A decision maker (DM) doesn't know which is which, but DM watches two people operate the machine. The first person in the sequence knows exactly which arm is good, and the second one has no idea. The first person will always pull the good arm, while the second person will pull the good arm $50\%$ of the time. The response $\RV{H}$ takes values that can be summarised as ``0 is good'' and ``1 is good'' (which we'll just refer to as $\{0,1\}$), and the DM assigns 50\% probability to each initially. Then for any $\alpha$
\begin{align}
    \prob{P}_\alpha^{\RV{Y}_2|\RV{D}_2\RV{D}_1}(100|1, 0) &= \prob{P}_\alpha^{\RV{Y}_2|\RV{D}_2\RV{H}}(100|1,0)\prob{P}_\alpha^{\RV{H}|\RV{D}_2\RV{D}_1}(0|1,0) + \prob{P}_\alpha^{\RV{Y}_2|\RV{D}_2\RV{H}}(100|1,1)\prob{P}_\alpha^{\RV{H}|\RV{D}_2\RV{D}_1}(1|1, 0)\\
    &= 0\cdot 1 + 1\cdot 0\\
    &= 0
\end{align}
while
\begin{align}
    \prob{P}_\alpha^{\RV{Y}_1|\RV{D}_1\RV{D}_2}(100|1,0) &= \prob{P}_\alpha^{\RV{Y}_1|\RV{D}_1\RV{H}}(100|1,0)\prob{P}_\alpha^{\RV{H}|\RV{D}_1\RV{D}_2}(0|1) + \prob{P}_\alpha^{\RV{Y}_1|\RV{D}_1\RV{H}}(100|1,1)\prob{P}_\alpha^{\RV{H}|\RV{D}_1\RV{D}_2}(1|1,0)\\
    &= 0\cdot0 + 1\cdot 1\\
    &= 1\\
    &\neq \prob{P}_C^{\RV{Y}_2|\RV{D}_2\RV{D}_1}(100|1,0)
\end{align}
\end{example}

Example \ref{ex:no_swapping} motivates the weaker symmetry we call \emph{exchange commutativity}. The key difference is that exchange commutativity allows for the permutation of pairs after conditioning on some variable $\RV{W}$. That is, a sequential input-output model $(\prob{P}_C,\RV{D},\RV{Y})$ is exchange commutative if there is some variable $\RV{W}$ such that the conditional $\prob{P}_\alpha^{\RV{Y}|\RV{WD}}$ is symmetric to paired swaps of $\RV{Y}$ and $\RV{D}$.

\begin{definition}[Exchange commutativity]\label{def:caus_exch}
Given a sequential input-output model $(\prob{P}_C,\RV{D},\RV{Y})$ along with some $\RV{W}:\Omega\to W$, we say $(\prob{P}_C,\RV{D},\RV{Y})$ \emph{commutes with exchange} over $\RV{W}$ if for all finite permutations $\rho:\mathbb{N}\to\mathbb{N}$ and all $\alpha\in C$
\begin{align}
    \prob{P}_\alpha^{\RV{Y}|\RV{WD}} &=  \prob{P}_\alpha^{\RV{Y}_\rho|\RV{WD}_\rho}
\end{align}
We say $(\prob{P}_C,\RV{D},\RV{Y})$ commutes with exchange if there is some $\RV{W}$ such that $(\prob{P}_C,\RV{D},\RV{Y})$ commutes with exchange over $\RV{W}$.   
\end{definition}

A second regularity condition we will consider can be roughly understood as the idea that $\RV{Y}_i$ doesn't ``depend on'' $\RV{D}_j$ for $j\neq i$. As Example \ref{ex:no_swapping} suggests, this cannot be an assumption that $\RV{Y}_i$ doesn't depend on $\RV{D}_j$ unconditionally; $\RV{D}_j$ could, after all, offer some evidence about the state of the unknown response function $\RV{H}$. Instead, we assume that $\RV{Y}_i$ doesn't depend on non-corresponding $\RV{X}_j$ after conditioning on some auxiliary $\RV{W}$.

\begin{definition}[Locality]\label{def:caus_cont}
Given a sequential input-output model $(\prob{P}_C,\RV{D},\RV{Y})$ along with some $\RV{W}:\Omega\to W$, the model is \emph{local} over $\RV{W}$ if for all $\alpha\in C$, $n\in \mathbb{N}$, $\RV{Y}_i\CI^e_{\prob{P}_C} \RV{X}_{\{i,\infty)}|(\RV{W},\RV{X}_i,\text{id}_C)$. If there is some $\RV{W}$ such that $(\prob{P}_C,\RV{D},\RV{Y})$ is local over $\RV{W}$ then we say $(\prob{P}_C,\RV{D},\RV{Y})$ is local.
\end{definition}

If an input-output model is both exchange commutative and local, then we say it is \emph{input-output contractible}. This term is chosen because such a model is unchanged by contractions of the input and output indices - see Theorem \ref{th:equal_of_condits}.

\begin{definition}[Input-output contractibility]\label{def:ccontract}
A sequential input-output model $(\prob{P}_\cdot,\RV{D},\RV{Y})$ along with some $\RV{W}:\Omega\to W$ is \emph{input-output contractible} (IO contractible) over $\RV{W}$ if it is local and commutes with exchange over $\RV{W}$.
\end{definition}

\begin{theorem}[Equality of equally sized conditionals]\label{th:equal_of_condits}
Given a sequential input-output model $(\prob{P}_C,\RV{D},\RV{Y})$ and some $\RV{W}$, $\prob{P}_\alpha^{\RV{Y}|\RV{WD}}$ is IO contractible over $\RV{W}$ if and only if for all subsequences $A,B\subset \mathbb{N}$ with $|A|=|B|$ and for every $\alpha$
\begin{align}
    \prob{P}_\alpha^{\RV{Y}_A|\RV{WD}_{A,\mathbb{N}\setminus A}} &= \prob{P}_\alpha^{\RV{Y}_B|\RV{WD}_{B,\mathbb{N}\setminus B}}\\
    &= \prob{P}_\alpha^{\RV{Y}_A|\RV{WD}_A}\otimes \text{del}_{D^{|\mathbb{N}\setminus A|}}
\end{align}
\end{theorem}

\begin{proof}
Appendix \ref{sec:equal_condits}
\end{proof}

Appendix \ref{app:examples_symmetries} explores out two additional properties of these two symmetries. Example \ref{ex:no_implication} shows that neither locality nor exchange commutativity is implied by the other. Example \ref{ex:interference_w_locality} shows that locality by itself does not rule out everything that we might intuitively describe as ``interference'' between pairs.

We might wonder if both locality and exchange commutativity are needed, seeing as exchange commutativity itself appears to be a generalisation of exchangeability - and in fact, if we take the inputs to be trivial $*$ then it coincides precisely with exchangeability. However, for nontrivial inputs, we can construct exchance commutative models where the response function depends on a symmetric function of the full set of inputs $\RV{D}_i$. An example of this possibility is a crude model of inflation: if you give any one person \$100, they'll be \$100 richer in real terms, but if you give everyone \$100 you cause a lot of inflation so the benefits are reduced and may even be negative to sufficiently wealthy people. That is, the impact on someone's wealth $\RV{Y}_i$ doesn't just depend on $\RV{D}_i$, but on the entire sequence $\RV{D} = (\RV{D}_i)_{i\in [n]}$. In this model it doesn't make much sense to give half of the people \$100 then measure the effects, then estimate the impact on the remaining half, because the second half of the inputs change the final outcomes for the first half of the distribution.

\subsection[Representation]{Representation of IO contractible models}\label{sec:rep_theorem}

In this section, we state Theorem \ref{th:ciid_rep_kernel}: a sequential input output model $(\prob{P}_\cdot,\RV{D},\RV{Y})$ features pairs $(\RV{D}_i,\RV{Y}_i)$ related by conditionally independent and identical responses if and only if it is IO contractible over some variable $\RV{W}$.

The proof of the theorem is involved, and can be found in its entirety Appendix \ref{app:representation_proof}. Note that we employ a string diagram notation in some steps of the proof, explained in Appendix \ref{ssec:mken_diagrams}. In the main paper, we just introduce enough to explain the key terms in the theorem statement. 

\subsection{Preliminaries}\label{sec:rep_theorem_background}

\begin{definition}[Input count variable]\label{def:count_of_inputs}
Given a sequential input-output model $(\prob{P}_\cdot,\RV{D},\RV{Y})$ with countable $D$, $\#_{j}^k$ is the variable
\begin{align}
    \#_{\RV{D}_{\cdot}=j}^k := \sum_{i=1}^{k-1} \llbracket \RV{D}_i = j \rrbracket
\end{align}
That is, $\#_{\RV{D}_{\cdot}=j}^k$ is equal to the number of times $\RV{D}_i=j$ over all $i<k$.
\end{definition}

If we have an infinite sequence of pairs $(\RV{D}_i,\RV{Y}_i)$, we can wrap the sequence $\RV{Y}$ into a table $\RV{Y}^D$ such that $\RV{Y}^D_{11}$ is equal to the value of the first $\RV{Y}_i$ such that $\RV{D}_i=1$, $\RV{Y}^D_{21}$ is equal to the value of the second such $\RV{Y}_i$ and so forth. We call it a ``tabulated conditional'' because, under the assumption of CIIRs, we can evaluate a conditional $\prob{P}_\alpha^{\RV{Y}|\RV{D}}(\cdot|d_1,d_2,...)$ by ``looking up'' the marginal distribution $\prob{P}_\alpha^{\RV{Y}^D_{1 d_1}\RV{Y}^D_{2 d_2}...}$ over the appropriate elements of $\RV{Y}^D$.

\begin{definition}[Tabulated conditional distribution]\label{def:tab_cd}
Given a sequential input-output model $(\prob{P}_\cdot,\RV{D},\RV{Y})$ on $(\Omega,\sigalg{F})$, define the \emph{tabulated conditional distribution} $\RV{Y}^D:\Omega\to Y^{\mathbb{N}\times D}$ by
\begin{align}
    \RV{Y}^D_{ij} = \sum_{k=1}^{\infty} \llbracket \#_{\RV{D}_{\cdot}=j}^k = i\rrbracket \llbracket \RV{D}_k = j \rrbracket \RV{Y}_k
\end{align}
That is, the $(i,j)$-th coordinate of $\RV{Y}^D$ is equal to the value of $\RV{Y}_k$ for which the corresponding $\RV{D}_k$ is the $i$th instance of the value $j$ in the sequence $(\RV{D}_1,\RV{D}_2,...)$, or 0 if there are fewer than $i$ instances of $j$ in this sequence.
\end{definition}

The \emph{directing random measure} of a sequence of exchangeable variables is defined as the map from the set of events of each variable in the sequence the limit of normalised partial sums of indicator functions over that set \citep{kallenberg_basic_2005}. The directing random measure is a probability measure. For completeness, we also define a directing random measure in the case that the relevant limit does not exist, although we are only practically interested in using the definition where the limit does exist.

\begin{definition}[Directing random measure]\label{def:dir_rand_meas}
Given a decision model $(\prob{P}_\cdot,\Omega,\sigalg{F})$ and a sequence $\RV{X}:=(\RV{X}_i)_{i\in\mathbb{N}}$, the directing random measure of $\RV{X}$ written $\RV{H}:\Omega\to \Delta(X)$ is the function
\begin{align}
    \RV{H} := A \mapsto \begin{cases}
    \lim_{n\to \infty}\frac{1}{n} \sum_{i=1}^{\infty} \mathds{1}_{A}(\RV{X}_{i}) & \text{this limit exists for all }\alpha\in C\\
    \llbracket A = X \rrbracket &\text{otherwise}
    \end{cases} 
\end{align}
\end{definition}

Given input and output sequences $\RV{D}$ and $\RV{Y}$ we define the \emph{directing random conditional} as the directing random measure of the tabulated conditional $\RV{Y}^D$ interpreted as a sequence of column vectors $((\RV{Y}^D_{1j})_{j\in D},(\RV{Y}^D_{2j})_{j\in D},...)$.

\begin{definition}[Directing random conditional]\label{def:dir_rand_cond}
Given a sequential input-output model $(\prob{P}_\cdot,\RV{D},\RV{Y})$, we will say the directing random conditional $\RV{H}:\Omega\to \Delta(Y^D)$ is the function
\begin{align}
    \RV{H} := \bigtimes_{j\in D} A_j \mapsto \begin{cases}
    \lim_{n\to \infty}\frac{1}{n} \sum_{i=1}^{\infty} \prod_{j\in D} \mathds{1}_{A_j}(\RV{Y}^D_{ij}) & \text{this limit exists}\\
    \llbracket \bigtimes_{j\in D} A_j = Y^D \rrbracket &\text{otherwise}
    \end{cases} 
\end{align}
\end{definition}

A finite permutation of rows is a function that independently permutes a finite number of elements in each row of a table. A special case of such a function is one that swaps entire columns (that is, a permutation of rows that applies the same permutation to each row).

\begin{definition}[Permutation of rows]
Given a sequence of indices $(i,j)_{i\in \mathbb{N},j\in D}$ a finite permutation of rows is a function $\eta:\mathbb{N}\times D\to \mathbb{N}\times D$ such that for each $j\in D$, $\eta_j:=\eta(\cdot,j)$ is a finite permutation $\mathbb{N}\to \mathbb{N}$ and $\eta(i,j)=(\eta_j(i),j)$.
\end{definition}

Lemma \ref{th:table_rep_kernel} shows that an IO contractible conditional distribution can be represented as the product of a column exchangeable probability distribution and a ``lookup function'' or ``switch''. This lookup function is also used in the representation of potential outcomes models (see, for example, \citet{rubin_causal_2005}), but we do not assume that the tabulated conditional $\RV{Y}^D$ is interpretable as potential outcomes. By representing a conditional probability as an exchangeable regular probability distribution, we can apply De Finetti's theorem, which is a key step in proving the main result of Theorem \ref{th:ciid_rep_kernel}.

To prove Lemma \ref{th:table_rep_kernel}, we assume that the set of input sequences in which each value appears infinitely often has measure 1 for every option in $C$. Without this assumption, we would have to accept positive probability that we run out of $\RV{D}_i$s taking some value $j\in D$ preventing us from filling out the ``tabulated conditional'' $\RV{Y}^D$ correctly. We call this side condition \emph{infinite support}.

\begin{definition}[Almost surely infinite]\label{def:infinite_support}
Given a sequential input-output model $(\prob{P}_\cdot,\RV{D},\RV{Y})$ with $D$ countable if, letting $E\subset D^{\mathbb{N}}$ be the set of all sequences such that for all $j\in D$
\begin{align}
    x\in E \implies \sum_{i=0}\llbracket x_i = j\rrbracket = \infty
\end{align}
we have $\prob{P}_\alpha^{\RV{D}|\RV{W}}(E|w)=1$ for all $\alpha,w$, then we say $\RV{D}$ is \emph{almost surely infinite } over $\RV{W}$.
\end{definition}

The key property of the tabulated conditional is that we can evaluate the regular conditional $\prob{P}_\alpha^{\RV{Y}|\RV{WD}}$ by ``looking up'' the appropriate marginal of $\prob{P}_\alpha^{\RV{Y}^D}$.

\begin{lemma}\label{th:table_rep_kernel}
Suppose a sequential input-output model $(\prob{P}_\cdot,\RV{D},\RV{Y})$ is given with $D$ countable and $\RV{D}$ infinitely supported over $\RV{W}$. Then for some $\RV{W}$, $\alpha$, $\prob{P}_\alpha^{\RV{Y}|\RV{WD}}$ is IO contractible if and only if
\begin{align}
    \prob{P}_\alpha^{\RV{Y}|\RV{WD}}(\bigtimes_{i\in \mathbb{N}}A_i|w,(d_i)_{i\in \mathbb{N}}) &= \prob{P}_\alpha^{(\RV{Y}^D_{i d_i})_{i\in\mathbb{N}}|\RV{W}}(\bigtimes_{i\in \mathbb{N}}A_i|w)&\forall A_i\in \sigalg{Y}^{D}, w\in W, d_i\in D
\end{align}
and for any finite permutation of rows $\eta:\mathbb{N}\times D\to \mathbb{N}\times D$
\begin{align}
    \prob{P}_\alpha^{(\RV{Y}^D_{ij})_{\mathbb{N}\times D}|\RV{W}}&= \prob{P}_\alpha^{(\RV{Y}^D_{\eta(i,j)})_{\mathbb{N}\times D}|\RV{W}}\label{eq:col_exch}
\end{align}
\end{lemma}

\begin{proof}
Only if: We define a random invertible function $\RV{R}:\Omega\times \mathbb{N}\to \mathbb{N}\times {D}$ that reorders the indicies so that, for $i\in \mathbb{N},j\in D$, $\RV{D}_{\RV{R}^{-1}(i,j)}=j$ almost surely. We then use IO contractibility to show that $\prob{P}_\alpha^{\RV{Y}|\RV{D}}(\cdot|d)$ is equal to the distribution of the elements of $\RV{Y}^D$ selected according to $d\in D^{\mathbb{N}}$.

If: We construct a conditional probability according to Definition \ref{def:tab_cd} and verify that it satisfies IO contractibility.

The full proof can be found in Appendix \ref{app:representation_proof}. Note that the proof uses string diagram notation explained in Appendix \ref{ssec:mken_diagrams}.
\end{proof}

Because the distribution $\prob{P}_\alpha^{\RV{Y}^D|\RV{W}}$ from Lemma \ref{th:table_rep_kernel} is row-exchangeable, the limit in the definition of the directing random conditional $\RV{H}$ exists almost surely (see Lemma \ref{lem:ciid_yd}).  In fact, we do not need the full sequence of pairs $(\RV{D},\RV{Y})$ to calculate $\RV{H}$; any subsequence $A\subset\mathbb{N}$ that satisfies the condition that $\RV{D}_A$ is infinitely supported over $\RV{W}$ is sufficient.

\begin{theorem}\label{th:any_infinite_sequence}
Suppose a sequential input-output model $(\prob{P}_\cdot,\RV{D},\RV{Y})$ is given with $D$ countable,  $\RV{D}$ infinitely supported over $\RV{W}$ and for some $\RV{W}$, $\prob{P}_\alpha^{\RV{Y}|\RV{WD}}$ is IO contractible for all $\alpha$. Consider an infinite set $A\subset \mathbb{N}$, and let $\RV{D}_A:=(\RV{D}_i)_{i\in A}$ and $\RV{Y}_A:=(\RV{Y}_i)_{i\in A}$ such that $\RV{D}_A$ is also infinitely supported over $\RV{W}$. Then $\RV{H}_A$, the directing random conditional of $(\prob{P}_\cdot,\RV{D}_A,\RV{Y}_A)$ is almost surely equal to $\RV{H}$, the directing random conditional of $(\prob{P}_\cdot,\RV{D},\RV{Y})$.
\end{theorem}

\begin{proof}
The strategy we pursue is to show that an arbitrary subsequence of $(\RV{D}_i,\RV{Y}_i)$ pairs induces a random contraction of the rows of $\RV{Y}^D$. Then we show that the contracted version of $\RV{Y}^D$ has the same distribution as the original, and consequently the normalised partial sums converge to the same limit.

The proof is in Appendix \ref{app:representation_proof}.
\end{proof}

We are now ready to state the main result, Theorem \ref{th:ciid_rep_kernel}. Assuming a sequential input-output model $(\prob{P}_\cdot,\RV{D},\RV{Y})$ (Definition \ref{def:seq_io}) with inputs $\RV{D}$ infinitely supported (Definition \ref{def:infinite_support}) over some random variable $\RV{W}$, $(\prob{P}_\cdot,\RV{D},\RV{Y})$ is IO contractible over the same $\RV{W}$ if and only if the pairs $(\RV{D}_i, \RV{Y}_i)$ share conditionally independent and identical responses (Definition \ref{def:cii_rf}), given by the directing random conditional $\RV{H}$ (Definition \ref{def:dir_rand_cond}) and $(\prob{P}_\cdot,\RV{D},\RV{Y})$ is weakly data-independent.

\subsection{Statement of the representation theorem}\label{sec:reptheorem_statement}

\begin{theorem}[Representation of IO contractible models]\label{th:ciid_rep_kernel}
Suppose a sequential input-output model $(\prob{P}_\cdot,\RV{D},\RV{Y})$ with sample space $(\Omega,\sigalg{F})$ is given with $D$ countable and $\RV{D}$ infinitely supported over $\RV{W}$. Then the following are equivalent:
\begin{enumerate}
    \item There is some $\RV{W}$ such that $\prob{P}_\alpha^{\RV{Y}|\RV{WD}}$ is IO contractible for all $\alpha$
    \item For all $i$, $\RV{Y}_i\CI (\RV{Y}_{\neq i},\RV{D}_{\neq i})|(\RV{H}, \RV{D}_i, \RV{C})$ and for all $i,j, \alpha$ $$\prob{P}_\alpha^{\RV{Y}_i|\RV{H}\RV{D}_i}=\prob{P}_\alpha^{\RV{Y}_j|\RV{H}\RV{D}_j}$$
    \item There is some $\kernel{L}:H\times D\kto Y$ such that for all $\alpha$, $$\prob{P}_\alpha^{\RV{Y}|\RV{DH}}(\bigtimes_{i\in\mathbb{N}} A_{i}|d,h)= \prod_{i\in\mathbb{N}} \kernel{L}(A_i|d_i,h)$$
\end{enumerate}
\end{theorem}

\begin{proof}
(1)$\implies$(3):
We apply Lemma \ref{th:table_rep_kernel} followed by Lemma \ref{lem:ciid_yd} followed by Lemma \ref{lem:hw_interchange}.


(3)$\implies$ (2):
We verify that the required conditional independences hold assuming (3).

(2)$\implies$ (1):
We show that, assuming (2), then $\prob{P}_\alpha^{\RV{Y}|\RV{WD}}$ is IO contractible over $\RV{W}$ for all $\alpha$.

See Appendix \ref{sec:io_contract_models} for the full proof. Note that the proof uses string diagram notation explained in Appendix \ref{ssec:mken_diagrams}.
\end{proof}

Whenever we have an input-output model with conditionally independent and identical responses given some arbitrary $\RV{W}$, then we also have conditionally independent and identical responses given the directing random conditional $\RV{H}$.

\begin{corollary}\label{lem:ci_drc}
If a sequential input-output model $(\prob{P}_C,\RV{D},\RV{Y})$ has independent and identical responses conditional on some variable $\RV{W}$ and $\RV{D}$ has infinite support over the same $\RV{W}$, then letting $\RV{H}$ be the directing random conditional with respect to inputs $\RV{D}$ and outputs $\RV{Y}$, it follows that for for all $i$, $\RV{Y}_i\CI \RV{W}|(\RV{D}_i,\RV{H},\RV{C})$ and for all $\alpha, i, j$, $\prob{P}_\alpha^{\RV{Y}_i|\RV{D}_i\RV{H}}=\prob{P}_\alpha^{\RV{Y}_j|\RV{D}_j\RV{H}}$.
\end{corollary}

\begin{proof}
We have by Theorem \ref{th:ciid_rep_kernel} that $\prob{P}_\alpha^{\RV{Y}|\RV{W}\RV{D}}$ is IO contractible over $\RV{W}$. The conclusion follows by applying Theorem \ref{th:ciid_rep_kernel} a second time.
\end{proof}

Building on Corollary \ref{lem:ci_drc}, Theorem \ref{th:infinite_condition_swaps} shows the assumption that the pairs $(\RV{D}_i,\RV{Y}_i)$ are related by conditionally independent and identical responses implies that, for the purposes of learning the response function $\RV{H}$, all infinite subsequences of $(\RV{D}_i,\RV{Y}_i)$ pairs with appropriate support are interchangeable. That is, suppose we have some infinite $A\subset \mathbb{N}$ for such that $(\prob{P}_\cdot,\RV{D}_A,\RV{Y}_A)$ is unimpeachably IO contractible over $*$ -- perhaps all pairs indexed by $A$ are derived from a carefully conducted experiment in precisely the conditions of interest to the decision maker and are therefore considered interchangeable in this strong sense. If we have some other infinite set $B\subset \mathbb{N}\setminus A$ of pairs derived from passive observation, then the assumption of conditionally independent and identical responses for the whole collection of pairs $(\RV{D}_i,\RV{Y}_i)_{i\in\mathbb{N}}$ implies that while we may not be able to swap individual pairs in $A$ with individual pairs in $B$, we must be able to swap the whole set $A$ for the whole set $B$ for the purposes of learning the response function $\RV{H}$.

\begin{theorem}\label{th:infinite_condition_swaps}
A data-independent sequential input-output model $(\prob{P}_C,\RV{D},\RV{Y})$ with directing random conditional $\RV{H}$ and $\RV{D}$ infinitely supported over $\RV{H}$ features conditionally independent and identical response functions $\prob{P}_C^{\RV{Y}_i|\RV{D}_i\RV{H}}$ only if for any sets $A,B\subset \mathbb{N}$ such that $\RV{D}_A$ and $\RV{D}_B$ are also infinitely supported over $\RV{H}$ and any $i,j\in \mathbb{N}$ such that $i\not\in A$, $j\not\in B$, 
\begin{align}
\prob{P}_\alpha^{\RV{Y}_i|\RV{D}_i\RV{Y}_A,\RV{D}_A}=\prob{P}_\alpha^{\RV{Y}_j|\RV{D}_j\RV{Y}_B\RV{D}_B}
\end{align}
If in addition each $\prob{P}_\alpha^{\RV{YD}}$ is dominated by some exchangeable $\prob{Q}_\alpha^{\RV{Y}\RV{D}}$, then the reverse implication also holds.
\end{theorem}

\begin{proof}
See Appendix \ref{sec:data_independent_proofs}.
\end{proof}

\subsection[Does IO contractibility help?]{Does IO contractibility help us understand identification?}\label{sec:symmetries_discussion}

One of the key contributions of De Finetti's representation theorem was to provide an alternative justification for the common modelling assumption that a sequence of variables were all distributed according to a shared but unknown ``true distribution''. De Finetti regarded the notion of an ``unknown true distribution'' as nonsensical, and through his representation theorem suggested that we could instead justify this structure by arguing that the experiment that produced the sequence of variables was, from the point of view of the analyst seeking to make predictions, invariant to reindexing the variables in the sequence.

Can IO contractibility help to justify common causal assumptions in a similar way? This question is less straightforward because IO contractibility is not such a straightforward symmetry. However, we think it does offer some insight into a common kind of causal assumption. Rather than lending justification to this assumption, the we think that it strengthens the case that this assumption is usually unreasonable.

The particular assumption we have in mind is, in the world of causal graphical models, the assumption that backdoor adjustment is possible and in the world of potential outcomes it is the assumption of \emph{conditional ignorability} \citep{rubin_causal_2005}. Both assumptions hold that, given a treatment $\RV{D}_i$, covariates $\RV{X}_i$ and an outcome $\RV{Y}_i$, there is an unknown but common conditional distribution of $\RV{Y}_i$ given $\RV{D}_i$ and $\RV{X}_i$ for all $i$, where $i$ ranges over passive observations as well as the consequences of actions. That is, we assume that the pairs $((\RV{D}_i,\RV{X}_i),\RV{Y}_i)$ share conditionally independent and identical responses. The key implication is Theorem \ref{th:infinite_condition_swaps}, which holds that, if the sequences of observations and consequences are both infinite, then for the purpose of learning the response function the problem is unchanged by swapping any subset of the indices corresponding to observations with any subset of those corresponding to consequences. That is, there is no difference between predicting the response function of the passive observations from an infinite sequence of passive observational data and predicting the response function of the consequences of the decision makers actions from the same sequence of passive observational data.

In practice, we propose that it would be very rare to have both of these datasets and treat them as interchangeable in this manner. Example \ref{ex:no_infinite_swapping} makes a similar point.

\begin{example}\label{ex:no_infinite_swapping}
Suppose an experiment is done which assigns some medical treatment $\RV{D}_i$ uniformly according to some random signal to patients for even $i$, and allows assignment by patient and doctor discretion for odd $i$. $\RV{Y}_i$ is a binary variable recording some health outcome of interest and $\RV{X}_i$ is some vector of covariates. The sequence $(\RV{D}_c,\RV{X}_c,\RV{Y}_c)$ is associated with the consequences of a decision maker's choices, where $c$ is some special character not in $\mathbb{N}$.

According to Theorem \ref{th:infinite_condition_swaps}, the assumption of conditionally independent and identical responses applied to $((\RV{D},\RV{X}),\RV{Y})$ implies
\begin{align}
    \prob{P}_\alpha^{\RV{Y}_c|\RV{D}_c\RV{X}_c\RV{D}_{\text{odds}}\RV{X}_{\text{odds}}\RV{Y}_{\text{odds}}}&=\prob{P}_\alpha^{\RV{Y}_c|\RV{D}_c\RV{X}_c\RV{D}_{\text{evens}\setminus\{0\}}\RV{X}_{\text{evens}\setminus\{0\}}\RV{Y}_{\text{evens}\setminus\{0\}}}\\
    &=\prob{P}_\alpha^{\RV{Y}_2|\RV{D}_2\RV{X}_2\RV{D}_{\text{evens}}\RV{X}_{\text{evens}\setminus\{2\}}\RV{Y}_{\text{evens}\setminus\{2\}}}\\
    &=\prob{P}_\alpha^{\RV{Y}_2|\RV{D}_2\RV{X}_2\RV{D}_{\text{odds}}\RV{X}_{\text{odds}}\RV{Y}_{\text{odds}}}
\end{align}

That is, under this assumption, the following four problems are deemed identical:
\begin{itemize}
    \item Predicting the outcome of the decision maker's input from the experimental data
    \item Predicting the outcome of the decision maker's input from the observational data
    \item Predicting a held-out experimental outcome from the experimental data
    \item Predicting a held-out experimental outcome from the observational data
\end{itemize}
Any answer to one problem is, under this assumption, an answer for all of them. This is an assumption; we do not conclude this by comparing answers to these different problems and finding them to be the same, we simply assume it is so. The proposition that these problems are \emph{identical} is hard to swallow: it seems very unlikely, for example, if an analyst aiming to predict experimental results with access to the experimental data would be satisfied with their previous answer derived from the observational data.
\end{example}

In practice, when both experimental and observational data are available, they are \emph{not} assumed to be interchangeable in this sense -- in fact, the question of how well the observational data predicts experimental outputs is one of substantial interest \citet{eckles_bias_2021,gordon_comparison_2018,gordon_close_2022}.


%!TEX root = main.tex

\section[Precedented options]{Inferring consequences when options have precedent}\label{sec:precedent}

We have suggested that conditionally independent and identical responses is usually an unreasonably strong assumption for a decision maker to make, on the grounds that it implies overly strong interchangeability properties between different datasets. One way to get around this objection is to suppose that conditionally independent and identical responses are shared by pairs $(\RV{E}_i,\RV{X}_{i})$ where the $\RV{E}_i$ are in fact latent variables. In this case, the assumption would still assert that infinite $(\RV{E}_i,\RV{X}_{i})$ sequences arising from observation would be interchangeable with infinite $(\RV{E}_j,\RV{X}_{j})$ sequences arising as consequences of actions, but because the $\RV{E}_i$ are never observed these interchanges do not imply that we would use the same model for different experiments.

To simplify the presentation, we will consider a specific kind of decision model featuring long sequence of exchangeable observations indexed by natural numbers that are unresponsive to the decision maker's choice and ``one more'' variable representing the ``consequences of action'' indexed by the special character $c$ that may be responsible ot the decision maker's choice. That is, we have $(\RV{X}_i)_{i\in \mathbb{N}}$ unresponsive to the decision maker and $(\RV{X}_c)$ responsive to the decision maker. Call this setup a ``see-do model''.

\begin{definition}[See-do model]
A see-do model is an decision model $(\prob{P}_\cdot,\Omega,C)$ along with a sequence of variables $\RV{X}_{\mathbb{N}\cup\{c\}}$ where $\RV{X}_{\mathbb{N}}\CI^e_{\prob{P}_\cdot} \mathrm{id}_C$. Variables indexed with $i\in \mathbb{N}$ are referred to as \emph{observations} and variables indexed with the special index $c$ are referred to as \emph{consequences}. We specify a see-do model with the shorthand $(\prob{P}_\cdot,\RV{X}_{\mathbb{N}\cup\{c\}})$.
\end{definition}

In this section, we will consider the following kind of ``standard'' see-do model: we have some observed variables $(\RV{X},\RV{Y},\RV{Z})$ and an unobserved variable $\RV{E}$ such that the observation pairs $(\RV{Z}_{i},(\RV{E}_i,\RV{X}_i,\RV{Y}_i))_{i\in \mathbb{N}}$ share conditionally independent and identical responses. Typically, this might be because we assume observations are exchangeable, but we also allow for cases where $\RV{Z}_i$ is not exchangeable -- for example, perhaps it is a time variable which monotonically increases. We also assume that the pairs $(\RV{E}_i,(\RV{X}_i,\RV{Y}_i))_{i\in\mathbb{N}\cup\{c\}}$ share conditionally independent and identical responses for all indices.

Recall that in Section \ref{sec:evaluating_decisions} we suggested that many systems might exhibit (probabilistically) regular input-output behaviours, but where we might not know or observe the right ``inputs''. The assumption that the pairs $(\RV{E}_i,(\RV{X}_i,\RV{Y}_i))_{i\in\mathbb{N}\cup\{c\}}$ share conditionally indpendent and identical responses can be viewed as a formalisation of this intuition; there is some unknown and unobserved state $\RV{E}_i$ which $\RV{X}_i$ and $\RV{Y}_i$ respond to in a regular manner no matter what else is happening.

Note that we make no assumptions about the distribution of $\RV{Z}_c$.

\begin{definition}[Latent CIIR see-do model]
A \emph{latent CIIR see-do model} is a see-do model $(\prob{P}_\cdot,(\RV{E}_i,\RV{Z}_{i},\RV{X}_i,\RV{Y}_i)_{i\in \mathbb{N}\cup\{c\}})$ such that the observation pairs $(\RV{Z}_{i},(\RV{E}_i,\RV{X}_i,\RV{Y}_i))_{i\in \mathbb{N}}$ share conditionally independent and identical responses and the pairs $(\RV{E}_i,(\RV{X}_i,\RV{Y}_i))_{i\in\mathbb{N}\cup\{c\}}$ also share conditionally independent and identical responses. We say the $\RV{E}_i$s are ``latent'' variables, which informally means that we typically do not get to observe them. We adopt the convention that the directing random conditional of $(\prob{P}_\cdot,\RV{Z}_{\mathbb{N}},(\RV{E}_i,\RV{X}_i,\RV{Y}_i)_{i\in \mathbb{N}})$.
\end{definition}

We can take any see-do model $(\prob{P}_\cdot,\RV{X}_{\mathbb{N}\cup\{c\}})$ with exchangeable observations and turn it into a latent CIIR see-do model by setting $\RV{Z}_i=*$ and $\RV{E}_i=(\RV{X}_i,\RV{Y}_i)$. This trivial construction typically isn't very helpful, though. One particular feature we might want is for a latent CIIR model to express the fact that ``things we can do have been done before''; that is, any setting of the unobserved state $\RV{E}_c$ that our actions might yield has positive probability in the observed data. Example \ref{ex:construction_latent_set} illustrates model constructions with and without this property.

\begin{example}\label{ex:construction_latent_set}
Suppose we have a see-do model $(\prob{P}_\cdot, \RV{X}_{\mathbb{N}\cup\{c\}})$ where each $\RV{X}_i$ takes values in a binary set, and the control we can exert is to choose either $\prob{P}_{0}^{\RV{X}_c} = \frac{1}{4}\delta_0 + \frac{3}{4}\delta_1$ or $\prob{P}_{1}^{\RV{X}-c} = \frac{1}{2}\delta_0 + \frac{1}{2}\delta_1$, independent of all other observations. Suppose further that for $i\in \mathbb{N}$, $\prob{P}_C^{\RV{X}_i}=\frac{3}{4}\delta_0 + \frac{1}{4}\delta_1$ independent of all other observations. Then we can consider this model to be IO contractible with latent binary inputs $\RV{E}_i$ such that
\begin{align}
	\prob{P}_\alpha^{\RV{X}_i|\RV{E}_i}(\cdot|e) &= \delta_e
\end{align}

This is not the only way to construct such a model. We could instead choose latent binary inputs $\RV{E}_i'$ such that
\begin{align}
	\prob{P}_\alpha^{\RV{X}_i|\RV{E}_i'}(\cdot|e) &= \begin{cases}
		\frac{3}{4}\delta_0 + \frac{1}{4}\delta_1 & e=0 \\
		\frac{1}{4}\delta_0 + \frac{3}{4}\delta_1 & e=1
	\end{cases}
\end{align}

On the other hand, the choice $\RV{E}_i^{\prime\prime}$ with
\begin{align}
	\prob{P}_\alpha^{\RV{X}_i|\RV{E}_i''}(\cdot|e) &= \begin{cases}
		\frac{1}{2}\delta_0 + \frac{1}{2}\delta_1 & e=0 \\
		\frac{1}{4}\delta_0 + \frac{3}{4}\delta_1 & e=1
	\end{cases}
\end{align}
cannot be latent binary inputs for a conditionally independent and identical response model, as the observational distribution cannot be written as any convex combination of $\prob{P}_\alpha^{\RV{X}_i|\RV{E}_i''}(\cdot|0)$ and $\prob{P}_\alpha^{\RV{X}_i|\RV{E}_i''}(\cdot|1)$.
\end{example}

In the first construction in Example \ref{ex:construction_latent_set}, but not the following two, we have $\prob{P}_C^{\RV{E}_i} \gg \prob{P}_\alpha^{\RV{E}_c}$ for all $\alpha$. We say under this construction the options have \emph{precedent}; they have, in a sense, ``been done before''. The assumption of precedent by itself has some implications -- for example, if a decision maker considers precedent a reasonable assumption and they have access to a lot of data, they they should not expect any of their actions to lead to consequences that have never appeared before in the observational data. In Theorem \ref{th:latent_to_observable}, we will make use a stronger version of this assumption where the conditional distribution over $\RV{E}_i$ is different for each value of $\RV{Z}_i$, which leads to stronger conclusions. We will discuss the plausibility of the stronger assumption afterwards.

Theorem \ref{th:latent_to_observable} is motivated by the following example:

\begin{example}\label{ex:doctor_precedent}
Suppose a decision maker collects data about a group of peope who have variously engaged the services of dietiticians, sporting coaches, general practitioners, bariatric surgeons and none of the above, with practitioner choice recorded under the variable $\RV{Z}_i$. The decision maker has also collected data on each person's body mass index $\RV{X}_i$ at the beginning of the study and followed mortality outcomes $\RV{Y}_i$ for a considerable period of time. A decision maker is reviewing this data, and in particular is wondering if steps they take to manage their weight $\RV{X}_c$ are likely to improve their own mortality prospects $\RV{Y}_c$.

Our decision maker presumes that each group of people $\RV{Z}_i$ has, in aggregate, different strategies for pursuing weight management and different contextual reasons for doing so (though, for the sake of this example, we suppose that the decision maker doesn't collect data on any of these facts). Because of this variation, the decison maker reasons, people in these different groups with different levels of body mass index should see different mortality results \emph{if, conditional on body mass index, the different circumstances and management strategies actually lead to different results}. Conversely, if there is \emph{no} variation in results for these different groups of people, then it would appear that, at least with regard to mortality, the eventual body mass index achieved is apparently the \emph{only} important feature of any management plan.

This inference might fail if, for any reason, the variation in treatment plans and contexts between the different groups of people surveyed masks the variation in their effects. For example, if all groups of people overwhelmingly choose to pursue diet changes in the end and other dimensions of variation are simply not very important to the outcome, then their results will not reveal any variation in mortality outcomes due to different treatment strategies. Alternatively, it might be the case that everybody is making choices that achieve nearly optimal mortality prospects given their unobserved context and that the best achievable mortality outcomes are approximately the same for each person's achievable level of body mass index. In this case there may still be substantial variation in outcomes from different weight management strategies, but it is masked by the fact that everyone is making near-optimal choices.

If the decision maker finds that $\RV{Y}_i$ is not independent of $\RV{Z}_i$ given $\RV{X}_i$, they may also consider whether $\RV{Y}_i$ is independent of $\RV{Z}_i$ given $(\RV{V}_i,\RV{X}_i)$ for some set of covariates $\RV{V}_i$.
\end{example}

Theorem \ref{th:latent_to_observable} establishes formal conditions for the informal deduction described in Example \ref{ex:doctor_precedent}. We assume that all variables of interest are discrete, and make use of an alternative notation for discrete conditional probabilities.

\begin{definition}[Index notation for discrete conditionals]
Given a joint probability distribution $\mu^{\RV{XY}}$ with $\RV{X}$ and $\RV{Y}$ discrete, let $\mu^y_x:=\mu^{\RV{Y}|\RV{X}}(\{y\}|x)$ and $\mu^Y_X:= (x,y)\mapsto \mu^y_x$
\end{definition}

The key assumption for Theorem \ref{th:latent_to_observable} is an assumption we call \emph{diverse precedent}. It's a rather complicated assumption. It imposes a domination condition that requires (roughly speaking) that the distribution of the latent input $\RV{E}_i$ given event $\RV{Z}_i=z$ almost surely dominates the distribution induced by any option we can choose (as in the discussion of precedent above) \emph{and} is almost surely ``diverse'' for different values of $\RV{Z}_i$.

\begin{definition}[Diverse precedent]\label{def:diverse_precedent}
Given a latent CIIR see-do model $(\prob{P}_\cdot,(\RV{E}_i,\RV{X}_i,\RV{Y}_i,\RV{Z}_i)_{i\in\mathbb{N}\cup\{c\}})$ with $E,X,Y$ and $Z$ all discrete, recall $\RV{G}$ is the directing random conditional of $(\prob{P}_\cdot,\RV{Z}_{\mathbb{N}},(\RV{E}_i,\RV{X}_i,\RV{Y}_i)_{i\in \mathbb{N}})$. 

We say that the options $C$ have \emph{diverse precedent} with respect to $(\prob{P}_\cdot,(\RV{E}_i,\RV{X}_i,\RV{Y}_i,\RV{Z}_i)_{i\in\mathbb{N}\cup\{c\}})$ if $\prob{P}_\cdot$ satisfies the diversity condition:
\begin{align}
    \prob{P}_{\alpha}^{\RV{G}^{EX}_{Z}|\RV{G}^{Y}_{EXZ}}(\cdot|g^{Y}_{EXZ}) &\ll U_{\Delta(E\times X)}& \forall \alpha, z, \prob{P}_\alpha-\text{almost all }g^{Y}_{EXZ}\label{eqApp:lebesgue_dom}
\end{align}
as well as the precedent condition:
\begin{align}
    \prob{P}_\alpha^{\RV{E}_c|\RV{G}} &\ll \sum_{z\in Z}\prob{P}_\alpha^{\RV{E}_i|\RV{G}}(\cdot|g)&\prob{P}_\alpha-\text{almost all }g
\end{align}
Where $U_{\Delta(E)}$ is the uniform measure on the $|E-1|$ simplex of discrete probability distributions with $|E|$ outcomes.
\end{definition}

For Theorem \ref{th:latent_to_observable}, we assume that on the basis of observations we condition the probability on some event $I$ (in particular, we are interested in the case where $I$ is the event that a certain conditional independence holds).

\begin{theorem}[Latent to observable IO contractibility]\label{th:latent_to_observable}
Given a latent CIIR see-do model $(\prob{P}_\cdot,(\RV{E}_i,\RV{X}_i,\RV{Y}_i,\RV{Z}_i)_{i\in\mathbb{N}\cup\{c\}})$ with $E,X,Y$ and $Z$ all discrete, recall $\RV{G}$ is the directing random conditional of $(\prob{P}_\cdot,\RV{Z}_{\mathbb{N}},(\RV{E}_i,\RV{X}_i,\RV{Y}_i)_{i\in \mathbb{N}})$.

Let $I\subset \Delta(Y)^{XZ}$ be the event $\RV{G}^Y_{Xz}=\RV{G}^Y_{Xz'}$ for all $z,z'\in Z$; i.e. the event that $\RV{Y}_i$ is independent of $\RV{Z}_i$ conditional on $\RV{X}_i$ and $\RV{G}^Y_{XZ}$. Define $\prob{Q}_\alpha\in \Delta(\Omega)$ to be the probability measure such that, for all $A\in \sigalg{F}$
\begin{align}
\prob{Q}_\alpha(A) := \prob{P}_\alpha^{\mathrm{id}_\Omega|\mathds{1}_I\circ \RV{G}}(A|1)
\end{align}
i.e. $\prob{Q}_\alpha$ is $\prob{P}_\alpha$ conditioned on $\RV{G}^Y_{XZ}\in I$, so $\RV{Y}_i\CI^e_{\prob{Q}_\cdot} \RV{Z}_i|(\RV{X}_i,\mathrm{id}_C)$.

If the options $C$ have diverse precedent with respect to $(\prob{Q}_\cdot,(\RV{E}_i,\RV{X}_i,\RV{Y}_i,\RV{Z}_i)_{i\in\mathbb{N}\cup\{c\}})$, then $(\prob{Q}_\cdot,\RV{X},\RV{Y})$ is also IO contractible.
\end{theorem}

\begin{proof}
We show that the assumption of conditional independence imposes a polynomial constraint on $\RV{G}^d_z$ which is nontrivial unless $\RV{Y}_i\CI^e (\RV{Z}_i,\RV{E}_i,\text{id}_C)|(\RV{X}_i,\RV{H})$, and hence the solution set $S$ for this constraint has measure 0 when this conditional independence does not hold.

Full proof in Appendix \ref{sec:proof_precedent}.
\end{proof}

\section{Continuity and discovery}{Causal discovery and the absolute continuity assumption}

We've suggested that the precedent assumption may be plausible when we suppose that the outcomes of interest exhibit a probabilistically regular response to some unobserved state, and that the values of this unobserved state that can be brought about through our actions have some precedent in the observed data. The continuity assumption deserves some more attention. Note that we require the continuity assumption to hold after conditioning on the independence of $\RV{Y}_i$ from $\RV{Z}_i$ given $\RV{X}_i$; the core of Theorem \ref{th:latent_to_observable} is that if we rule out the possibility of explaining this independence through a lack of ``continuity'' of the conditionals $\RV{G}^{X}_{EZ}$ and $\RV{G}^{E}_{Z}$, then it must be due to $\RV{Y}_i$ also being independent of $\RV{E}_i$ given $\RV{X}_i$. Under what circumstances would we want to rule out explaining this independence through a lack of continuity?

We do not propose a conclusive answer to this question. However, the causal discovery literature offers a possible approach which we will explain here. Causal structures have two useful properties for this purpose: first, they can ``explain'' conditional independences like $\RV{Y}_i$ is independent of $\RV{Z}_i$ given $\RV{X}_i$. Second, under standard interpretations, given a structural causal model, parental conditional probabilities are mutually continuous (in our terminology). This property has been noted before, and has been informally stated as the \emph{principle of independent causal mechanisms} (in spite of the name, parental conditional probabilities need not be probabilistically independent) \citep{lemeire_replacing_2013,peters_elements_2017}.

To illustrate why we need to explain the conditional independence, suppose we pick a ``prior'' distribution of $\RV{G}^{EXY}_{Z}$ (by which we just mean $\prob{P}_{\cdot}^{\RV{G}^{EXY}_{Z}}$) such that for each $z,z'\in Z$, the distribution of $\RV{G}^{EXY}_z$ conditional on $\RV{G}^{EXY}_{z'}$ is absolutely continuous with respect to the uniform measure on $E\times X\times Y$. In this case, the event $\RV{G}^Y_{Xz}=\RV{G}^Y_{Xz'}$ would have probability 0 for all $z,z'\in Z$. As such, this ``prior'' does not tell us whether diverse precedent holds after conditioning on this event, as conditional probabilities are not determined by the joint distribution on measure 0 events. Thus this prior does not tell us whether diverse precedent holds after conditioning on the independence of $\RV{Y}_i$ from $\RV{Z}_i$ given $\RV{X}_i$.

Bayesian causal discovery approaches prior specification in a different way. This approach considers a mixture of graphical hypotheses which each imply certain conditional independences \citep{heckerman_learning_1995}. Because each graphical hypothesis is given positive probability, independences like $\RV{Y}_i$ independent of $\RV{Z}_i$ given $\RV{X}_i$ also have positive probability, in contrast with the approach that sets $\RV{G}^{EXY}_z$ absolutely continuous with respect to the uniform measure.

Another conventional feature of structure learning is the assumption that the parameters associated with parental conditional distributions are mutually continuous. That is, for any $\RV{X}$, fixing a hypothesised structure $\mathcal{G}$, $\prob{P}_\cdot^{\RV{G}^{X}_{\mathrm{Pa}_\mathcal{G}(X)}|\RV{G}^{X^\complement}_{\mathrm{Pa}_\mathcal{G}(X^\complement)}}$ is absolutely continuous (see Definition \ref{def:mga} for the meaning of $\mathrm{Pa}_{\mathcal{G}}(\RV{X})$). \citep{heckerman_learning_1995} assumes these parameters are mutually independent with absolutely continuous marginals. While not explicitly Bayesian, \citet{meek_strong_1995} argues that given a hypothesised causal model, \emph{unfaithful}\footnote{See the referenced paper for a definition.} distributions are unlikely because they violate mutual absolute continuity.

Putting these properties together, we have:
\begin{itemize}
	\item We may ``explain'' the independence $\RV{Y}_i\CI^e_{\prob{Q}} \RV{Z}_i | (\RV{G}, \mathrm{id}_C)$ with a hypothesised causal structure $\mathcal{G}$
	\item If \emph{either} $\RV{Z}_i$ is a parent of $\RV{E}_i$ or $\RV{E}_i$ and $\RV{Z}_i$ are parents of $\RV{X}_i$ in $\mathcal{G}$ then the relevant continuity property for Theorem \ref{th:latent_to_observable} holds
\end{itemize}

For example, suppose we consider a class of structural hypotheses $\mathcal{G}_i$ where, in all structures, we have $\RV{E}_i\rightarrowtriangle \RV{X}_i$ and $\RV{G}_i\rightarrowtriangle \RV{X}_i$. One such structure is illustrated below:

\begin{align}
	\mathcal{G}_1 := \tikzfig{structural_model_1} \label{eq:structural_model_1}
\end{align}

In $\mathcal{G}_1$ (and, indeed, in every $\mathcal{G}_i$) we have $\mathrm{Pa}_{\mathcal{G}_i}(\RV{X}_i)=(\RV{Z}_i,\RV{E}_i)$. Furthermore, $\RV{G}^E_Z$ is determined by $\RV{G}^E$ and $\RV{G}^Z_E$, which are both parental parameters in $\mathcal{G}_1$ and therefore mutually absolutely continuous with $\RV{G}^X_{EZ}$. Finally, $\RV{G}^Y_{EXz} = \RV{G}^Y_{EX}$ for all $z\in Z$ is also a parental conditional and therefore also mutually absolutely continuous with $\RV{G}^X_{EZ}$. Thus, given hypothesis $\mathcal{G}_1$, if we also accept the assumption of precedent, Theorem \ref{th:latent_to_observable} follows. In fact, we can similarly argue that Theorem \ref{th:latent_to_observable} follows for all structures $\mathcal{G}_i$ where:
\begin{itemize}
	\item $\RV{E}_i\rightarrowtriangle \RV{X}_i$ and $\RV{Z}_i\rightarrowtriangle \RV{X}_i$
	\item $\RV{E}_i\rightarrowtriange \RV{X}_i$ and $\RV{Z}_i\rightarrowtriangle \RV{E}_i$
\end{itemize}

On the other hand, consider
\begin{align}
	\mathcal{G}_2 := \tikzfig{structural_model_2} \label{eq:structural_model_2}
\end{align}

Here we also have $\RV{Y}_i\CI^e_{\prob{Q}} \RV{Z}_i | (\RV{G}, \mathrm{id}_C)$, but absolute continuity is not obviously supported; neither $\RV{G}^X_{EZ}$ nor $\RV{G}^E_Z$ are parental parameters. In fact, we can observe that in $\mathcal{G}_2$ we have $\RV{Y}_i\CI^e_{\prob{Q}} \RV{Z}_i | (\RV{X}_i, \RV{G}, \mathrm{id}_C)$ given an arbitrary choice of $\RV{G}^Y_{EX}$, but not $\RV{Y}_i\CI^e_{\prob{Q}} \RV{E}_i | (\RV{X}_i, \RV{G}, \mathrm{id}_C)$, so by the contrapositive of Theorem \ref{th:latent_to_observable} we must not have the relevant absolutely continuous conditionals.

If, in our original example, instead of conditioning on medical practitioners, we take $\RV{Z}_i$ to be an individual's clothing size and (for the sake of argument) find the same result: mortality outcomes $\RV{Y}_i$ are independent of clothing size $\RV{Z}_i$ conditional on body mass index $\RV{X}_i$. There are no doubt some differences between people with the same body mass index who wear differently sized clothes -- height, for example -- but it is not clear from the given data whether we should conclude that any common actions affecting a person's health do so via their body mass index, or whether there are features relevant to a person's health that fail to be correlated with clothing size after conditioning on body mass index. This example is motivated by structure \eqref{eq:structural_model_2}, though it's hard to come up with an example that is inarguably an instance of this structure and also not excessively convoluted.


While we've argued that structure \eqref{eq:structural_model_1} implies $\RV{Y}_i \CI^e_{\prob{P}} \RV{E}_i | (\RV{X}_i, \mathcal{G}_1, \RV{G}, \mathrm{id}_C)$ via Theorem \ref{th:latent_to_observable}, we can note that it also implies this independence via d-separation. On the other hand, \eqref{eq:structural_model_2} does not imply such a conditional independence via Theorem \ref{th:latent_to_observable} or via d-separation. This correspondence is not exact - the structure \eqref{eq:structural_model_3} implies $\RV{Y}_i \CI^e_{\prob{P}} \RV{E}_i | (\RV{X}_i, \mathcal{G}, \RV{G}, \mathrm{id}_C)$ via d-separation, but not via Theorem \ref{th:latent_to_observable}. The existence of such stuctures may not be especially surprising, given that Theorem \ref{th:latent_to_observable} establishes sufficient but not necessary conditions.

\begin{align}
	\mathcal{G}_2 := \tikzfig{structural_model_3} \label{eq:structural_model_3}
\end{align}

The alternative d-separation criterion does uphold the general rule we observed, where $\RV{E}_i\rightarrowtriangle \RV{X}_i$ and $\RV{Z}_i\rightarrowtriangle \RV{X}_i$ or $\RV{E}_i\rightarrowtriange \RV{X}_i$ and $\RV{Z}_i\rightarrowtriangle \RV{E}_i$ are sufficient to establish $\RV{Y}_i \CI^e_{\prob{P}} \RV{E}_i | (\RV{X}_i, \mathcal{G}_i, \RV{G}, \mathrm{id}_C)$, while the absence of either of these edges is compatible with structures that do not imply the relevant independence.

If we assume structural models are expressing the assumption that parental conditional probabilities are mutually absolutely continuous, it may be the case that d-separation properties capture all interesting consequences like those of Theorem \ref{th:latent_to_observable} where mutual absolute continuity supports additional conditional independence implications. It is well known that conditional independences relationships not implied by a structure $\mathcal{G}$ can be broken by choosing slightly different parental conditionals (see for example \citet{meek_strong_1995,zhang_strong_2003}), thus under the assumption that all parental conditionals relative to a structure $\mathcal{G}$ are mutually absolutely continuous we would not find additional conditional independence properties. We may wonder if there are interesting mutual absolute continuity assumptions that are not representable as a directed acyclic graph. In fact, the conditions for Theorem \ref{th:latent_to_observable} are somewhat weaker than those expressed by a structural model -- it only assumes mutual absolute continuity for a subset of conditionals, whereas a structural model assigns every variable a parental set and implies the mutual absolute continuity of every parental conditionals. However, as we have argued, this difference does not seem to matter in this case as both approaches yield the same conclusion.

In this discussion, we have suggested that structural causal models may play a role informing judgements of mutual absolute continuity. While this is a standard feature of the interpretation of structural causal models -- and \citet{lemeire_replacing_2013} has proposed that this idea should play a critical role in our understanding of causal structures -- for the purposes of assessing consequences of actions, the interpretation of causal models as ``interventional oracles'' \citep[Section 1.3.1]{pearl_causality:_2009} is usually given prominence. Here, we consider precedent as an alternative to the assumption of structural interventions, and lean on the interpretation of mutual absolute continuity to derive nontrivial conclusions.

\section{Conclusion}

We employ a decision theoretic approach to causal inference to investigate two different approaches to answering the question ``how do my observations relate to the consequences of my choices?''. Firstly, we examined the assumption of conditionally independent and identical responses, and its equivalent form in IO contractibility, which we argued was often an unreasonable assumption and secondly, we examined an approach based on the principle of precedent, or the idea that the decision maker's options have been taken before, and some of their consequences observed. Our approach allows us to consider the question of what observations and consequences have in common independently from any prior knowledge the decision maker might have about how their choices influence outcomes -- neither Theorem \ref{th:ciid_rep_kernel} nor Theorem \ref{th:latent_to_observable} depend on any assumptions about a decision maker's prior knowledge of the effects of their different options (though the plausibility of the assumptions in both theorems may well depend on such prior knowledge).

The grand aim of this work is to facilitate causal inference in situations where a decision maker has relatively little causal knowledge at the outset. We think avoiding structured interventions in this setting is advantageous because we regard the question of whether an action is known in advance to influence a particular variable as substantially more transparent than the question of whether it is well modeled by a structured intervention (of any type) on that variable.

Nevertheless, this work leaves many open questions for causal inference in the low prior knowledge setting. We have argued that the assumptions required for Theorem \ref{th:ciid_rep_kernel} are unlikely to be compelling in many situations. While the diverse precedent assumption may be more broadly plausible, it is at this stage difficult to evaluate. Speculatively, it may be possible to make progress on this question by better understanding when structural assumptions support this conclusion, via for example the causal version of the principle of maximum entropy.

For practical purposes, a generalisation of Theorem \ref{th:latent_to_observable} to approximate independence is in order, and such a generalisation may also bring additional clarity to the diverse precedent assumption.

Despite these challenges, we are encouraged by a number of features of this work. Using decision making as a starting point for constructing models means that, at the outset, we are only making commitments a decision maker is likely to already be making if they want to apply a formal theory of decision making. The informal idea of precedent that underpins Theorem \ref{th:latent_to_observable} seems like a general principle that may be applicable in a broad range of data-driven decision making problems. Finally, the apparent connection between Theorem \ref{th:latent_to_observable} suggests that much of the work already done in the world of causal graphical models may be applicable to our alternative perspective. Causal inference under circumstances of limited prior knowledge presents many hard conceptual as well as practical problems, and our approach is a promising new avenue of investigation.



\bibliographystyle{plainnat}

\bibliography{library}

\appendix

%!TEX root = main.tex


\section{String Diagrams}\label{ssec:mken_diagrams}


We use a string diagram notation to represent probabilistic functions. This is a notation created for reasoning about abstract Markov categories, and is somewhat different to existing graphical languages. The main difference is that in our notation wires represent variables and boxes (which are like nodes in directed acyclic graphs) represent probabilistic functions. Standard directed acyclic graphs annotate nodes with variable names and represent probabilistic functions implicitly. The advantage of explicitly representing probabilistic functions is that we can write equations involving graphics. This is introduced in Section \ref{ssec:mken_diagrams}.

We make use of string diagram notation for probabilistic reasoning. Graphical models are often employed in causal reasoning, and string diagrams are a kind of graphical notation for representing Markov kernels. The notation comes from the study of Markov categories, which are abstract categories that represent models of the flow of information. For our purposes, we don't use abstract Markov categories but instead focus on the concrete category of Markov kernels on standard measurable sets.

A coherence theorem exists for string diagrams and Markov categories. Applying planar deformation or any of the commutative comonoid axioms to a string diagram yields an equivalent string diagram. The coherence theorem establishes that any proof constructed using string diagrams in this manner corresponds to a proof in any Markov category \citep{selinger_survey_2011}. More comprehensive introductions to Markov categories can be found in \citet{fritz_synthetic_2020,cho_disintegration_2019}.

\subsection{Elements of string diagrams}\label{sec:string_diagram_elements}

In the string, Markov kernels are drawn as boxes with input and output wires, and probability measures (which are Markov kernels with the domain $\{*\}$) are represented by triangles:

\begin{align}
\kernel{K}&:=\begin{tikzpicture}[baseline={([yshift=-.5ex]current bounding box.center)}]
    \path (0,0) node (A) {}
    ++ (0.5,0) node[kernel] (K) {$\kernel{K}$}
    ++ (0.5,0) node (B) {};
    \draw (A) -- (K) -- (B);
\end{tikzpicture}\\
\mu&:= \begin{tikzpicture}[baseline={([yshift=-.5ex]current bounding box.center)}]
    \path (0,0) node[dist] (K) {$\kernel{P}$}
    ++ (0.5,0) node (B) {};
    \draw (K) -- (B);
\end{tikzpicture}
\end{align}

Given two Markov kernels $\kernel{L}:X\kto Y$ and $\kernel{M}:Y\kto Z$, the product $\kernel{L}\kernel{M}$ is represented by drawing them side by side and joining their wires:

\begin{align}
    \kernel{L}\kernel{M}:= \begin{tikzpicture}[baseline={([yshift=-.5ex]current bounding box.center)}]
    \path (0,0) node (A) {$X$}
    ++ (0.5,0) node[kernel] (K) {$\kernel{K}$}
    ++ (0.7,0) node[kernel] (M) {$\kernel{M}$}
    ++ (0.5,0) node (B) {$Z$};
    \draw (A) -- (K) -- (M) -- (B);
\end{tikzpicture}
\end{align}

Given kernels $\kernel{K}:W\kto Y$ and $\kernel{L}:X\kto Z$, the tensor product $\kernel{K}\otimes\kernel{L}:W\times X\kto Y\times Z$ is graphically represented by drawing kernels in parallel:

\begin{align}
    \kernel{K}\otimes \kernel{L}&:=\begin{tikzpicture}[baseline={([yshift=-.5ex]current bounding box.center)}]
    \path (0,0) node (A) {$W$}
    ++ (0.5,0) node[kernel] (K) {$\kernel{K}$}
    ++ (0.5,0) node (B) {$Y$};
    \path (0,-0.5) node (C) {$X$}
    ++ (0.5,0) node[kernel] (L) {$\kernel{L}$}
    ++ (0.5,0) node (D) {$Z$};
    \draw (A) -- (K) -- (B);
    \draw (C) -- (L) -- (D);
\end{tikzpicture}
\end{align}

Given $\prob{K}:X\kto Y$ and $\prob{L}:Y\times X\kto Z$, the semidirect product is graphically represented by connecting $\kernel{K}$ and $\kernel{L}$ and keeping an extra copy

\begin{align}
    \prob{K}\cprod\prob{L}:&= \text{Copy}_X(\prob{K}\otimes \text{id}_X)(\text{Copy}_Y\otimes\text{id}_X )(\text{id}_Y \otimes \prob{L})\\
                            &= \tikzfig{copy_product}
\end{align}

A space $X$ is identified with the identity kernel $\mathrm{id}^X:X\to \Delta(\sigalg{X})$. A bare wire represents the identity kernel:

\begin{align}
\mathrm{Id}^X:=\begin{tikzpicture}
\path (0,0) node (X) {$X$}
++(2,0) node (Y) {$X$};
\draw (X) -- (Y);
\end{tikzpicture}
\end{align}

Product spaces $X\times Y$ are identified with tensor product of identity kernels $\mathrm{id}^X\otimes \mathrm{id}^Y$. These can be represented either by two parallel wires or by a single wire representing the identity on the product space $X\times Y$:
\begin{align}
X\times Y \cong \mathrm{Id}^X\otimes \mathrm{Id}^Y &:= \begin{tikzpicture}
\path (0,0) node (E) {$X$}
++(1,0) node (F) {$X$}
(0,-0.5) node (F1) {$Y$}
+(1,0) node (G) {$Y$};
\draw (E) -- (F);
\draw (F1) -- (G);
\end{tikzpicture}\\
&= \begin{tikzpicture}
\path (0,0) node (X) {$X\times Y$}
++(2,0) node (Y) {$X\times Y$};
\draw (X) -- (Y);
\end{tikzpicture}
\end{align}

A kernel $\kernel{L}:X\to \Delta(\mathcal{Y}\otimes\mathcal{Z})$ can be written using either two parallel output wires or a single output wire, appropriately labeled:

\begin{align}
&\begin{tikzpicture}
\path (0,0) node (E) {$X$}
++ (1,0) node[kernel] (L) {$\kernel{L}$}
++ (1,0.15) node (F) {$Y$}
+(0,-0.3) node (G) {$Z$};
\draw (E) -- (L);
\draw ($(L.east) + (0,0.15)$) -- (F);
\draw ($(L.east)+ (0,-0.15)$) -- (G);
\end{tikzpicture}\\
&\equiv\\
&\begin{tikzpicture}
\path (0,0) node (E) {$X$}
++ (1,0) node[kernel] (L) {$\kernel{L}$}
++ (1.5,0) node (F) {$Y\times Z$};
\draw (E) -- (L) -- (F);
\end{tikzpicture}
\end{align}

We read diagrams from left to right (this is somewhat different to \citet{fritz_synthetic_2020,cho_disintegration_2019,fong_causal_2013} but in line with \citet{selinger_survey_2011}), and any diagram describes a set of nested products and tensor products of Markov kernels. There are a collection of special Markov kernels for which we can replace the generic ``box'' of a Markov kernel with a diagrammatic elements that are visually suggestive of what these kernels accomplish.

\subsection{Special maps}

\begin{definition}[Identity map]\label{def:ident_k}
The identity map $\text{Id}_X:X\kto X$ defined by $(\text{id}_X)(A|x)= \delta_x(A)$ for all $x\in X$, $A\in\sigalg{X}$, is represented by a bare line.
\begin{align}
    \mathrm{id}_X&:=\begin{tikzpicture}[baseline={([yshift=-.5ex]current bounding box.center)}]
    \path (0,0) node (A) {$X$} ++ (0.5,0) node (B) {$X$};
    \draw (A) -- (B);
\end{tikzpicture}
\end{align}
\end{definition}

\begin{definition}[Erase map]\label{def:erase}
Given some 1-element set $\{*\}$, the erase map $\text{Del}_X:X\kto \{*\}$ is defined by $(\text{Del}_X)(*|x) = 1$ for all $x\in X$. It ``discards the input''. It looks like a lit fuse:
\begin{align}
    \text{Del}_X&:=\begin{tikzpicture}[baseline={([yshift=-.5ex]current bounding box.center)}]
    \path (0,0) ++ (1,0) node (B) {$X$};
    \draw[-{Rays[n=8]}] (A) -- (B);
\end{tikzpicture}
\end{align}
\end{definition}

\begin{definition}[Swap map]\label{def:swap}
The swap map $\text{Swap}_{X,Y}:X\times Y\kto Y\times X$ is defined by $(\text{Swap}_{X,Y})(A\times B|x,y)=\delta_x(B)\delta_y(A)$ for $(x,y)\in X\times Y$, $A\in \sigalg{X}$ and $B\in \sigalg{Y}$. It swaps two inputs and is represented by crossing wires:
\begin{align}
    \text{Swap}_{X,Y} &:=  \begin{tikzpicture}[baseline={([yshift=-.5ex]current bounding box.center)}]
        \path (0,0) node (A) {} 
        + (0,-0.5) node (B) {}
        ++ (1,0) node (C) {}
        + (0,-0.5) node (D) {};
        \draw (A) to [out=0,in=180] (D) (B) to [out=0, in=180] (C);
    \end{tikzpicture}
\end{align}
\end{definition}

\begin{definition}[Copy map]\label{def:copy}
The copy map $\text{Copy}_X:X\kto X\times X$ is defined by $(\text{Copy}_X)(A\times B|x)=\delta_x(A)\delta_x(B)$ for all $x\in X$, $A,B\in \sigalg{X}$. It makes two identical copies of the input, and is drawn as a fork:
\begin{align}
    \text{Copy}_X&:=\begin{tikzpicture}[baseline={([yshift=-.5ex]current bounding box.center)}]
    \path (0,0) node (A) {$X$} 
    ++ (0.5,0) node[copymap] (copy0) {}
    ++ (0.5,0.15) node (B) {$X$}
    + (0,-0.3) node (C) {$X$};
    \draw (A) -- (copy0) to [out=45,in=180] (B) (copy0) to [out=-45, in=180] (C);
\end{tikzpicture}
\end{align}
\end{definition}

\begin{definition}[$n$-fold copy map]
The $n$-fold copy map $\text{Copy}^n_X:X\kto X^n$ is given by the recursive definition
\begin{align}
    \text{Copy}^1_X &= \text{Copy}_X\\
    \text{Copy}^n_X &= \tikzfig{n_fold_copy} &n>1
\end{align}
\end{definition}

\paragraph{Plates}\label{pgph:plates}

In a string diagram, a plate that is annotated $i\in A$ means the tensor product of the $|A|$ elements that appear inside the plate. A wire crossing from outside a plate boundary to the inside of a plate indicates an $|A|$-fold copy map, which we indicate by placing a dot on the plate boundary. For our purposes, we do not define anything that allows wires to cross from the inside of a plate to the outside; wires must terminate within the plate.

Thus, given $\kernel{K}_i:X\kto Y$ for $i\in A$,

\begin{align}
    \bigotimes_{i\in A} \kernel{K}_i &:= \tikzfig{plate_without_copymap}
    \text{Copy}^{|A|}_X(\bigotimes_{i\in A} \kernel{K}_i) &:= \tikzfig{plate_with_copymap}
\end{align}

\subsection{Commutative comonoid axioms}

Diagrams in Markov categories satisfy the commutative comonoid axioms.

\begin{align}
    \tikzfig{ccom_lhs} = \tikzfig{ccom_rhs}\label{eq:ccom_1}
\end{align}
\begin{align}
    \tikzfig{ccom2_lhs} = \tikzfig{ccom2_mhs} = \tikzfig{ccom2_rhs}\label{eq:ccom2_del}
\end{align}
\begin{align}
    \tikzfig{ccom3_lhs} = \tikzfig{ccom3_rhs} \label{eq:ccom3_swap}
\end{align}
as well as compatibility with the monoidal structure
\begin{align}
    \tikzfig{mstruct1_lhs} &= \tikzfig{mstruct1_rhs}\\
    \tikzfig{mstruct2_lhs} &= \tikzfig{mstruct2_rhs}
\end{align}
and the naturality of \emph{Del}, which means that
\begin{align}
    \tikzfig{naturality_lhs} &= \tikzfig{naturality_rhs}\label{eq:nat}
\end{align}


\subsection{Manipulating String Diagrams}\label{sssec:string_diagram_manipulation}

Planar deformations along with the applications of Equations \eqref{eq:ccom_1} through to Equation \eqref{eq:nat} are almost the only rules we have for transforming one string diagram into an equivalent one. One further rule is given by Theorem \ref{th:fong_det_kerns}.

\begin{theorem}[Copy map commutes for deterministic kernels \citep{fong_causal_2013}]\label{th:fong_det_kerns}
For $\kernel{K}:X\kto Y$
\begin{align}
	\tikzfig{deterministic_copymap_commute}
\end{align}
holds iff $\kernel{K}$ is deterministic.
\end{theorem}

\subsubsection{Examples}

String diagrams can always be converted into definitions involving integrals and tensor products. A number of shortcuts can help to make the translations efficiently.

For arbitrary $\kernel{K}:X\times Y\kto Z$, $\kernel{L}:W\kto Y$

\begin{align}
    \tikzfig{identity_tensor_L} &= (\text{id}_X\otimes \kernel{L})\kernel{K}\\
    [(\text{id}_X\otimes \kernel{L})\kernel{K}](A|x,w) &= \int_{Y}\int_X   \kernel{K}(A|x',y')\kernel{L}(\mathrm{d}y'|w)\delta_x(\mathrm{d}x')\\
                                           &= \int_Y  \kernel{K}(A|x,y') \kernel{L}(dy'|w)
\end{align}

That is, an identity map ``passes its input directly to the next kernel''. 

For arbitrary $\kernel{K}: X\times Y\times Y\kto Z$:

\begin{align}
 \tikzfig{identity_tensor_copy} &= (\text{id}_X\otimes \text{Copy}_Y)\kernel{K}\\
 [(\text{id}_X\otimes \text{Copy}_Y)\kernel{K}](A|x,y) &= \int_Y\int_Y \kernel{K}(A|x,y',y'') \delta_y(\mathrm{d}y')\delta_y(\mathrm{d}y'')\\
                                           &= \kernel{K}(A|x,y,y)
\end{align}

That is, the copy map ``passes along two copies of its input'' to the next kernel in the product. 

For arbitrary $\kernel{K}:X\times Y\kto Z$

\begin{align}
    \tikzfig{swap_example} &= \text{Swap}_{YX} \kernel{K}\\
    (\text{Swap}_{YX}\kernel{K})(A|y,x) &= \int_{X\times Y} \kernel{K}(A|x',y')\delta_y(\mathrm{d}y')\delta_x(\mathrm{d}x')\\
                                        &= \kernel{K}(A|x,y)
\end{align}

The swap map before a kernel switches the input arguments.

For arbitrary $\kernel{K}:X\kto Y\times Z$

\begin{align}
    \tikzfig{swap_example_2} &= \kernel{K}\text{Swap}_{YZ}\\
    (\kernel{K}\text{Swap}_{YZ})(A\times B|x) &= \int_{Y\times Z} \delta_{y}(B)\delta_{z}(A)\kernel{K}(\mathrm{d}y\times\mathrm{d}z|x)\\
    &= \int_{B\times A} \kernel{K}(\mathrm{d}y\times\mathrm{d}z|x)\\
    &= \kernel{K}(B\times A|x)
\end{align}

Given $\kernel{K}:X\kto Y$ and $\kernel{L}:Y\kto Z$:

\begin{align}
	(\kernel{K}\cprod \kernel{L})(\mathrm{id}_{Y}\otimes \mathrm{Del}_Z) &= \tikzfig{semidirect_K_L}\\
	 &= \tikzfig{semidirect_K_L_2} &\text{by Eq. \eqref{eq:nat}}\\
	 &= \tikzfig{semidirect_K_L_3} &\text{by Eq. \eqref{eq:ccom2_del}}
\end{align}

Thus the action of the $\text{Del}$ map is to marginalise over the deleted wire. With integrals, we can write

\begin{align}
	(\kernel{K}\cprod \kernel{L})(\mathrm{id}_{Y}\otimes \mathrm{Del}_Z)(A\times\{*\}|x) &= \int_{Y}\int_{\{*\}}\delta_y(A)\delta_{*}(\{*\})\kernel{L}(\mathrm{d}z|y)\kernel{K}(\mathrm{d}y|x)\\
	&= \int_A \kernel{K}(\mathrm{d}y|x)\\
	&= \kernel{K}(A|x)
\end{align}

\section{Symmetries of conditional probabilities}\label{app:io_contract_examples}

\subsection{Equality of equally sized contractions}\label{sec:equal_condits}

This is the proof of Theorem \ref{th:equal_of_condits}.

All swaps can be written as a product of transpositions, so proving that a property holds for all finite transpositions is enough to show it holds for all finite swaps. It's useful to define a notation for transpositions.

\begin{definition}[Finite transposition]
Given two equally sized sequences $A,B\in \mathbb{N}^n$ with $A=(a_i)_{i\in [n]}$, $B=(b_i)_{i\in [n]}$, ${A\rightarrow B}:\mathbb{N}\to \mathbb{N}$ is the permutation such that 
\begin{align}
    [A\rightarrow B](a_i) = b_i
\end{align}that sends the $i$th element of $A$ to the $i$th element of $B$ and vise versa. Note that $B\rightarrow A$ is the inverse of $A\rightarrow B$.
\end{definition}

Lemma \ref{lem:infinitely_extended_kernels} is used to extend conditional probabilities of finite sequences to infinite ones. 

\begin{lemma}[Infinitely extended kernels]\label{lem:infinitely_extended_kernels}
Given a collection of Markov kernels $\kernel{K}_i:W\times X^{\mathbb{N}}\kto Y^i$ for all $i\in \mathbb{N}$, if we have for every $j>i$
\begin{align}
    \kernel{K}_j(\text{id}_{Y^i}\otimes \text{Del}_{Y^{j-i}}) &= \kernel{K}_i\otimes \text{Del}_{X^{j-i}}\label{eqApp:marginalise_comb}
\end{align} 
then there is a unique Markov kernel $\kernel{K}:X^{\mathbb{N}}\kto Y^{\mathbb{N}}$ such that for all $i,j\in \mathbb{N}$,$j>i$
\begin{align}
    \kernel{K}(\text{id}_{Y^i}\otimes \text{Del}_{Y^{\mathbb{N}}})&= \kernel{K}_i\otimes \text{Del}_{X^{j-i}}
\end{align}
\end{lemma}

\begin{proof}
Take any $x\in X^{\mathbb{N}}$ and let $x_{|m}\in X^n$ be the first $n$ elements of $x$. By Equation \eqref{eqApp:marginalise_comb}, for any $A_i\in \sigalg{Y}$, $i\in [m]$
\begin{align}
    \kernel{K}_n(\bigtimes_{i\in [m]}A_i\times Y^{n-m}|x_{|n}) &= \kernel{K}_m(\bigtimes_{i\in [m]}A_i|x_{|m})
\end{align}

Furthermore, by the definition of the $\mathrm{Swap}$ map for any permutation $\rho:[n]\to[n]$
\begin{align}
    \kernel{K}_n\mathrm{Swap}_{\rho}(\bigtimes_{i\in [m]}A_{\rho(i)}\times Y^{n-m}|x_{|n}) &= \kernel{K}_n(\bigtimes_{i\in [m]}A_{i}\times Y^{n-m}|x_{|n})
\end{align}
thus by the Kolmogorov Extension Theorem \citep{cinlar_probability_2011}, for each $x\in X^{\mathbb{N}}$ there is a unique probability measure $\prob{Q}_x\in \Delta(Y^{\mathbb{N}}$ satisfying
\begin{align}
    \prob{Q}_x(\bigtimes_{i\in [n]}A_i\times Y^{\mathbb{N}}) &= \kernel{K}_n(\bigtimes_{i\in [n]}A_{\rho(i)}|x_{[n]})\label{eqApp:q_is_Markov}
\end{align}

Furthermore, for each $\{A_i\in\sigalg{Y}|i\in \mathbb{N}\}$, $n\in \mathbb{N}$ note that for $p>n$
\begin{align}
\prob{Q}_x(\bigtimes_{i\in[n]} A_i \times Y^{\mathbb{N}})&\geq \prob{Q}_x(\bigtimes_{i\in [p]} A_i\times Y^{\mathbb{N}})\\
&\geq \prob{Q}_x(\bigtimes_{i\in \mathbb{N}} A_i)
\end{align}
so by the Monotone convergence theorem, the sequence $\prob{Q}_x(\bigtimes_{i\in[n]} A_i)$ converges as $n\to \infty$ to $\prob{Q}_x(\bigtimes_{i\in\mathbb{N}} A_i)$. $x\mapsto \prob{Q}_x^{\RV{Z}_n}(\bigtimes_{i\in[n]} A_i)$ is measurable for all $n$, $\{A_i\in\sigalg{Y}|i\in \mathbb{N}\}$ by Equation \eqref{eqApp:q_is_Markov}, and so $x\mapsto Q_x$ is also measurable.

Thus $x\mapsto Q_x$ is the desired Markov kernel $\kernel{K}$.
\end{proof}

\begin{corollary}\label{cor:equal_subconditionals}
Given $(\prob{P}_C,\Omega,\sigalg{F})$, $\RV{W}:\Omega\to V$ and two pairs of sequences $(\RV{V},\RV{X}):=(\RV{V}_i,\RV{X}_i)_{i\in\mathbb{N}}$ and $(\RV{Y},\RV{Z}):=(\RV{Y}_i,\RV{Z}_i)_{i\in \mathbb{N}}$ with corresponding variables taking values in the same sets $V=Y$ and $X=Z$, if $(\prob{P}_C,\RV{V},\RV{X})$ and $(\prob{P}_C,\RV{Y},\RV{Z})$ are both local over $\RV{W}$ and
\begin{align}
    \prob{P}^{\RV{X}_{[n]}|\RV{W}\RV{V}_{[n]}} &= \prob{P}^{\RV{Z}_{[n]}|\RV{W}\RV{Y}_{[n]}}
\end{align}
for all $n\in\mathbb{N}$ then
\begin{align}
    \prob{P}^{\RV{X}|\RV{W}\RV{V}} &= \prob{P}^{\RV{Z}|\RV{W}\RV{Y}}
\end{align}
\end{corollary}

\begin{proof}
By assumption of locality
\begin{align}
    \prob{P}^{\RV{X}_{[n]}|\RV{W}\RV{V}_{[n]}}\otimes\mathrm{Del}_{W^\mathbb{N}} &= \prob{P}^{\RV{X}|\RV{W}\RV{V}}(\mathrm{id}_{X^n}\otimes \mathrm{Del}_{X^{\mathbb{N}}})\\
    \prob{P}^{\RV{Z}_{[n]}|\RV{W}\RV{Y}_{[n]}}\otimes\mathrm{Del}_{W^\mathbb{N}} &= \prob{P}^{\RV{Z}|\RV{W}\RV{Y}}(\mathrm{id}_{X^n}\otimes \mathrm{Del}_{X^{\mathbb{N}}})
\end{align}
hence for all $n,m>n$
\begin{align}
    \prob{P}^{\RV{X}_{[m]}|\RV{W}\RV{V}_{[m]}}(\mathrm{id}_{X^n}\otimes \mathrm{Del}_{X^{m-n}}) &= \prob{P}^{\RV{Z}_{[m]}|\RV{V}\RV{Y}_{[m]}}(\mathrm{id}_{X^n}\otimes \mathrm{Del}_{X^{m-n}})\\
    &= \prob{P}^{\RV{X}_{[n]}|\RV{W}\RV{V}_{[n]}}\otimes\mathrm{Del}_{W^{m-n}}
\end{align}
and, in particular, by lemma \ref{lem:infinitely_extended_kernels}, $\prob{P}^{\RV{X}|\RV{W}\RV{V}}$ and $\prob{P}^{\RV{Z}|\RV{W}\RV{Y}}$ are the limits of the same sequence.
\end{proof}

\begin{reptheorem}{th:equal_of_condits}
Given a sequential input-output model $(\prob{P}_C,\RV{D},\RV{Y})$ and some $\RV{W}$, $\prob{P}_\alpha^{\RV{Y}|\RV{WD}}$ is IO contractible over $\RV{W}$ if and only if for all subsequences $A,B\subset \mathbb{N}^{|A|}$ and for every $\alpha$
\begin{align}
    \prob{P}_\alpha^{\RV{Y}_A|\RV{WD}_{A,\mathbb{N}\setminus A}} &= \prob{P}_\alpha^{\RV{Y}_B|\RV{WD}_{B,\mathbb{N}\setminus B}}\\
    &= \prob{P}_\alpha^{\RV{Y}_A|\RV{WD}_A}\otimes \text{del}_{D^{|\mathbb{N}\setminus A|}}
\end{align}
\end{reptheorem}

\begin{proof}
Only if:
For $Z\in \mathbb{N}^{|A|}$, let $\text{del}_{Z^\complement}$ be the Markov kernel associated with the map that sends $\RV{Y}$ to $\RV{Y}_Z:=(\RV{Y}_i)_{i\in Z}$.

If $A$ is finite, then let $n:=|A|$ and by exchange commutativity
\begin{align}
        \prob{P}_\alpha^{\RV{Y}_A|\RV{WD_{A,\mathbb{N}\setminus A}}}&= \prob{P}_\alpha^{\RV{Y}_A|\RV{WD_{A\rightarrow [n]}}}\\
         &= \prob{P}_\alpha^{\RV{Y}|\RV{WD_{A\rightarrow [n]}}}\text{del}_{A^{\complement}}\\
        &=  \prob{P}_\alpha^{\RV{Y}_{[n]\rightarrow A}|\RV{WD}}\text{del}_{A^{\complement}}
\end{align}
Use the fact that $[n]\rightarrow A \circ B\rightarrow [n]= B\rightarrow A$ and apply exchange commutativity to get
\begin{align}
    \prob{P}_\alpha^{\RV{Y}_{[n]\rightarrow A}|\RV{WD}}\kernel{F}_{\Pi_{A}} &= \prob{P}_\alpha^{\RV{Y}_{B\rightarrow A}|\RV{WD}_{B\rightarrow [n]}}\text{del}_{A^{\complement}}\\
    &= \prob{P}_\alpha^{\RV{Y}|\RV{WD}_{B\rightarrow [n]}}\text{del}_{B^{\complement}}\\
    &= \prob{P}_\alpha^{\RV{Y}_B|\RV{WD_{B,\mathbb{N}\setminus B}}}
\end{align}

if $A$ is infinite, then we can take finite subsequences $A_m$ that are the first $m$ elements of $A$ and similarly for $B_m$. Then by previous reasoning
\begin{align}
            \prob{P}_\alpha^{\RV{Y}_{A_m}|\RV{WD_{A_m\rightarrow [m]}}} &= \prob{P}_\alpha^{\RV{Y}_{[m]}|\RV{WD}}\\
        &= \prob{P}_\alpha^{\RV{Y}_{B_m}|\RV{WD_{B_m\rightarrow [m]}}}
\end{align}
then by Corollary \ref{cor:equal_subconditionals}
\begin{align}
\prob{P}_\alpha^{\RV{Y}_A|\RV{WD_{A\rightarrow [n]}}}=\prob{P}_\alpha^{\RV{Y}_{B_m}|\RV{WD_{B_m\rightarrow [m]}}}
\end{align}

Finally, by locality
\begin{align}
    \prob{P}_\alpha^{\RV{Y}_A|\RV{WD_{A\rightarrow [n]}}} &= \prob{P}_\alpha^{\RV{Y}_A|\RV{WD}_A}\otimes \text{Del}_{D^{|\mathbb{N}\setminus A}}
\end{align}

If:
Taking $A=[n]$ for all $n$ establishes locality, and taking $A=(\rho(i))_{i\in \mathbb{N}}$ for arbitrary finite permutation $\rho$ establishes exchange commutativity.
\end{proof}

\subsection{Examples of symmetries}\label{app:examples_symmetries}

These are the examples referenced in Section \ref{sec:ccontracibility}. Example \ref{ex:no_implication} shows that neither locality nor exchange commutativity is implied by the other.

\begin{example}\label{ex:no_implication}
We prove the claim by way of presenting counterexamples.

First, a model that exhibits exchange commutativity but not locality. Suppose $D=Y=\{0,1\}$ and $\prob{P}_C^{\RV{Y}|\RV{D}}:D^{\mathbb{N}}\kto Y^{\mathbb{N}}$ is given by
\begin{align}
    \prob{P}_C^{\RV{Y}|\RV{D}}(\bigtimes_{i\in\mathbb{N}} A_i |(d_i)_{i\in\mathbb{N}}) &= \prod_{i\in \mathbb{N}} \delta_{\lim_{n\to\infty} \sum_{i\in\mathbb{N}} \frac{d_i}{n}}(A_i)
\end{align}
for some sequence $(d_i)_{i\in\mathbb{N}}$ such that this limit exists. Then for any finite permutation $\rho$
\begin{align}
    \prob{P}_C^{\RV{Y}_\rho|\RV{D}_\rho}(\bigtimes_{i\in\mathbb{N}} A_i |(d_i)_{i\in\mathbb{N}}) &= \prod_{i\in \mathbb{N}} \delta_{\lim_{n\to\infty} \sum_{i\in\mathbb{N}} \frac{d_{\rho^{-1}(i)}}{n}}(A_{\rho^{-1}(i)})\\
    &= \prob{P}_C^{\RV{Y}|\RV{D}}(\bigtimes_{i\in\mathbb{N}} A_i |(d_i)_{i\in\mathbb{N}})
\end{align}
so $(\prob{P}_C,\RV{D},\RV{Y})$ commutes with exchange, but
\begin{align}
    \prob{P}_C^{\RV{Y}_1|\RV{D}}(A_1 |0,1,1,1....) &= \delta_1(A_1)\\
    \prob{P}_C^{\RV{Y}_1|\RV{D}}(A_1 |0,0,0,0....) &= \delta_0(A_1)
\end{align}
so $(\prob{P}_C,\RV{D},\RV{Y})$ is not local.

Next, a model that satisfies locality but does not commute with exchange. Suppose again $D=Y=\{0,1\}$ and $\prob{P}_C^{\RV{Y}|\RV{D}}:D^{\mathbb{N}}\kto Y^{\mathbb{N}}$ is given by
\begin{align}
    \prob{P}_C^{\RV{Y}|\RV{D}}(\bigtimes_{i\in\mathbb{N}} A_i |(d_i)_{i\in\mathbb{N}}) &= \prod_{i\in \mathbb{N}} \delta_i(A_i)
\end{align}
then
\begin{align}
    \prob{P}_C^{\RV{Y}_\rho|\RV{D}_\rho}(\bigtimes_{i\in\mathbb{N}} A_i |(d_i)_{i\in\mathbb{N}}) &= \prod_{i\in \mathbb{N}} \delta_i(A_{\rho^{-1}(i)})\\
    &\neq \prod_{i\in \mathbb{N}} \delta_i(A_{i})\\
    =\prob{P}_C^{\RV{Y}|\RV{D}}(\bigtimes_{i\in\mathbb{N}} A_i |(d_i)_{i\in\mathbb{N}})
\end{align}
so $(\prob{P}_C,\RV{D},\RV{Y})$ does not commute with exchange but for all $n$
\begin{align}
    \prob{P}_C^{\RV{Y}_{[n]}|\RV{D}}(\bigtimes_{i\in[n]} A_i |(d_i)_{i\in\mathbb{N}}) &= \prod_{i\in [n]} \delta_i(A_{\rho^{-1}(i)})\\
    &= \prob{P}_C^{\RV{Y}_{[n]}|\RV{D}}(\bigtimes_{i\in[n]} A_i |(0)_{i\in\mathbb{N}})
\end{align}
so $(\prob{P}_C,\RV{D},\RV{Y})$ is local.
\end{example}

Although locality seems to an assumption that there is no interference between inputs and outputs of different indices, by itself it actually permits models with certain kinds of interference. This is shown in Example \ref{ex:interference_w_locality}.

\begin{example}\label{ex:interference_w_locality}
Consider an experiment where I first flip a coin and record the results of this flip as the outcome $\RV{Y}_1$ of ``step 1''. Subsequently, I can either copy the outcome from step 1 to the result for ``step 2'' (this is the input $\RV{D}_1=0$), or flip a second coin use this as the input for step 2 (this is the input $\RV{D}_1=1$). $\RV{D}_2$ is an arbitrary single-valued variable. Then for all $d_1, d_2$
\begin{align}
    \prob{P}^{\RV{Y}_1|\RV{D}}(y_1|d_1,d_2) &= 0.5\\
    \prob{P}^{\RV{Y}_2|\RV{D}}(y_2|d_1,d_2) &= 0.5
\end{align}
Thus the marginal distribution of both experiments in isolation is $\text{Bernoulli}(0.5)$ no matter what choices I make, but the input $\RV{D}_1$ affects the joint distribution of the results of both steps, which is not ruled out by locality.
\end{example}

\subsection[Representation]{Representation theorem preliminaries}\label{app:representation_proof}

This is the proof of Lemmas \ref{th:table_rep_kernel} and \ref{lem:ciid_yd} and Theorem \ref{th:any_infinite_sequence} from Section \ref{sec:rep_theorem_background}. In addition, Lemmas \ref{lem:ciid_yd} and \ref{lem:hw_interchange} are presented and proved, which will be later used in the proof of Theorem \ref{th:ciid_rep_kernel}.

The following definitions are reproduced for the reader's convenience. Note that these proofs use the string diagram notation explained in Appendix \ref{ssec:mken_diagrams}.

\begin{repdefinition}{def:count_of_inputs}
Given a sequential input-output model $(\prob{P}_{\cdot},\RV{D},\RV{Y})$ on $(\Omega,\sigalg{F})$ with countable $D$, $\#_{j}^k$ is the variable
\begin{align}
    \#_{j}^k := \sum_{i=1}^{k-1} \llbracket \RV{D}_i = j \rrbracket
\end{align}
In particular, $\#_{j}^k$ is equal to the number of times $\RV{D}_i=j$ over all $i<k$.
\end{repdefinition}

\begin{repdefinition}{def:tab_cd}
Given a sequential input-output model $(\prob{P}_{\cdot},\RV{D},\RV{Y})$ on $(\Omega,\sigalg{F})$, define the tabulated conditional distribution $\RV{Y}^D:\Omega\to Y^{\mathbb{N}\times D}$ by
\begin{align}
    \RV{Y}^D_{ij} = \sum_{k=1}^{\infty} \llbracket \#_j^k = i-1\rrbracket \llbracket \RV{D}_k = j \rrbracket \RV{Y}_k
\end{align}
That is, the $(i,j)$-th coordinate of $\RV{Y}^D(\omega)$ is equal to the coordinate $\RV{Y}_k(\omega)$ for which the corresponding $\RV{D}_k(\omega)$ is the $i$th instance of the value $j$ in the sequence $(\RV{D}_1(\omega),\RV{D}_2(\omega),...)$, or 0 if there are fewer than $i$ instances of $j$ in this sequence.
\end{repdefinition}

\begin{replemma}{th:table_rep_kernel}
Suppose a sequential input-output model $(\prob{P}_C,\RV{D},\RV{Y})$ is given with $D$ countable and $\RV{D}$ infinitely supported. Then for some $\RV{W}$, $\alpha$, $\prob{P}_\alpha^{\RV{Y}|\RV{WD}}$ is IO contractible if and only if
\begin{align}
    \prob{P}_\alpha^{\RV{Y}|\RV{WD}} &= \tikzfig{lookup_representation_kernel}\label{eqApp:lup_rep_kernel}\\
    &\iff\\
    \prob{P}_\alpha^{\RV{Y}|\RV{WD}}(\bigtimes_{i\in \mathbb{N}}A_i|w,(d_i)_{i\in \mathbb{N}}) &= \prob{P}_\alpha^{(\RV{Y}^D_{i d_i})_{i\in\mathbb{N}}|\RV{W}}(\bigtimes_{i\in \mathbb{N}}A_i|w)&\forall A_i\in \sigalg{Y}^{D}, w\in W, d_i\in D
\end{align}
Where $\prob{F}_{\text{lu}}$ is the Markov kernel associated with the lookup map
\begin{align}
    \text{lu}:X^\mathbb{N}\times Y^{\mathbb{N}\times D}&\to Y\\
    ((x_i)_\mathbb{N},(y_{ij})_{i,j\in \mathbb{N}\times D})&\mapsto (y_{i d_i})_{i\in \mathbb{N}}
\end{align}
and for any finite permutation within rows $\eta:\mathbb{N}\times D\to \mathbb{N}\times D$
\begin{align}
    \prob{P}_\alpha^{(\RV{Y}^D_{ij})_{\mathbb{N}\times D}|\RV{W}}&= \prob{P}_\alpha^{(\RV{Y}^D_{\eta(i,j)})_{\mathbb{N}\times D}|\RV{W}}\label{eqApp:col_exch}
\end{align}
\end{replemma}

\begin{proof}
Only if:
We define a random invertible function $\RV{R}:\Omega\times \mathbb{N}\to \mathbb{N}\times {D}$ that reorders the indicies so that, for $i\in \mathbb{N},j\in D$, $\RV{D}_{\RV{R}^{-1}(i,j)}=j$ almost surely. We then use IO contractibility to show that $\prob{P}_\alpha^{\RV{Y}|\RV{D}}(\cdot|d)$ is equal to the distribution of the elements of $\RV{Y}^D$ selected according to $d\in D^{\mathbb{N}}$.

Note that at most one of $\llbracket \#_j^k = i-1\rrbracket\llbracket \RV{D}_k=j\rrbracket$ and $\llbracket \#_j^l = i-1\rrbracket\llbracket \RV{D}_l=j\rrbracket$ can be greater than 0 for $k\neq l$ and, by assumption, $\sum_{j\in D}\sum_{k\in \mathbb{N}} \llbracket \#_j^k = i-1\rrbracket\llbracket \RV{D}_k=j\rrbracket=1$ almost surely (that is, for any $i,j$ there is some $k$ such that $\RV{D}_k$ is the $i$th occurrence of $j$). Define $\RV{R}_k:\Omega\to \mathbb{N}\times D$ by $\omega \mapsto \argmax_{i\in\mathbb{N},j\in D} \llbracket \#_j^k = i-1\rrbracket\llbracket \RV{D}_k=j\rrbracket(\omega)$ (i.e. $\RV{R}_k$ returns the $(i,j)$ pair where $j$ is the value of $\RV{D}_k$ and $i$ is the count of $j$ occurrences up to $\RV{D}_k$). Let $\RV{R}:\mathbb{N}\to \mathbb{N}\times D$ by $k\mapsto \RV{R}_k$. $\RV{R}$ is almost surely bijective and 
\begin{align}
    \RV{Y}^D&:= (\RV{Y}^D_{ij})_{i\in \mathbb{N},j\in D}\\
    &= (\RV{Y}_{\RV{R}^{-1}(i,j)})_{i\in \mathbb{N},j\in D}\\
    &=: \RV{Y}_{\RV{R}^{-1}}
\end{align}

By construction, $\RV{D}_{\RV{R}^{-1}(i,j)}=j$ almost surely; that is, $\RV{D}_{\RV{R}^{-1}}$ is a single-valued variable. In particular, it is almost surely equal to $e:=(e_{ij})_{i\in\mathbb{N},j\in D}$ such that $e_{ij}=j$ for all $i$. Hence
\begin{align}
    \prob{P}_\alpha^{\RV{Y}^D|\RV{W}\RV{D}_{\RV{R}^{-1}}}(A|w,d)&= \prob{P}_\alpha^{\RV{Y}_{\RV{R}^{-1}}|\RV{W}\RV{D}_{\RV{R}^{-1}}}(A|w,d)\\
    &\overset{\prob{P}_{\cdot}}{\cong} \prob{P}_\alpha^{\RV{Y}_{\RV{R}^{-1}}|\RV{W}\RV{D}_{\RV{R}^{-1}}}(A|w,e)\label{eqApp:yd_is_indep}\\
    &= \prob{P}_\alpha^{\RV{Y}^D}(A|w)\label{eqApp:yd_dist}
\end{align}
for any $d\in D^{\mathbb{N}}$.

Now,
\begin{align}
    \prob{P}^{\RV{Y}_{\RV{R}^{-1}}|\RV{W}\RV{D}_{\RV{R}^{-1}}}_\alpha(A|w,d) &= \int_R \prob{P}_\alpha^{\RV{Y}_\rho|\RV{W}\RV{D}_{\rho}}(A|d)\prob{P}_\alpha^{\RV{R}^{-1}|\RV{W}\RV{D}_{\RV{R}^{-1}}}(\mathrm{d}\rho|w,d)\label{eqApp:need_ccont}\\
\end{align}
For each $\rho$, define $\rho^n:\mathbb{N}\to \mathbb{N}$ as the finite permutation that agrees with $\rho$ on the first $n$ indices and is the identity otherwise. By IO contractibility, for $n\in \mathbb{N}$
\begin{align}
    \prob{P}^{\RV{Y}_{\rho^n([n])}|\RV{W}\RV{D}_{\rho^n([n])}} &= \prob{P}^{\RV{Y}_{\rho([n])}|\RV{W}\RV{D}_{\rho([n])}}\\
    &= \prob{P}^{\RV{Y}_{[n]}|\RV{W}\RV{D}_{[n]}}
\end{align}
By Corollary \ref{cor:equal_subconditionals}, it must therefore be the case that
\begin{align}
    \prob{P}^{\RV{Y}|\RV{W}\RV{D}} = \prob{P}^{\RV{Y}_{\rho}|\RV{W}\RV{D}_{\rho}}
\end{align}
Then from Equation \eqref{eqApp:need_ccont}
\begin{align}
    \prob{P}^{\RV{Y}_{\RV{R}^{-1}}|\RV{W}\RV{D}_{\RV{R}^{-1}}}_\alpha(A|w,d) &\overset{\prob{P}_{\cdot}}{\cong} \int_R \prob{P}_\alpha^{\RV{Y}_\rho|\RV{W}\RV{D}_{\rho}}(A|d)\prob{P}_\alpha^{\RV{R}^{-1}|\RV{W}\RV{D}_{\RV{R}^{-1}}}(\mathrm{d}\rho|w,d)\\
    &\overset{\prob{P}_{\cdot}}{\cong} \int_R \prob{P}_{\cdot}^{\RV{Y}|\RV{WD}}(A|w,d)\prob{P}_\alpha^{\RV{R}^{-1}|\RV{W}\RV{D}_{\RV{R}^{-1}}}(\mathrm{d}\rho|w,d)\\
    &\overset{\prob{P}_{\cdot}}{\cong} \prob{P}_{\cdot}^{\RV{Y}|\RV{WD}}(A|w,d)\label{eqApp:rotated_conditional}
\end{align}
 for all $i,j\in \mathbb{N}$. Then by Equation \eqref{eqApp:yd_dist} and Equation \eqref{eqApp:rotated_conditional}
\begin{align}
    \prob{P}_\alpha^{\RV{Y}^D|\RV{W}}(A|w) &= \prob{P}_\alpha^{\RV{Y}|\RV{WD}}(A|w,e)\label{eqApp:rel_bet_y_yd}
\end{align}

Take some $d\in D^{\mathbb{N}}$. From Equation \eqref{eqApp:rel_bet_y_yd} and IO contractibility of $\prob{P}_{\cdot}^{\RV{Y}|\RV{WD}}(A|e)$,
\begin{align}
    (\prob{P}_\alpha^{\RV{Y}^D|\RV{W}}\otimes \mathrm{id}_D)\kernel{F}_{lu}(A|w,d) &= \prob{P}_\alpha^{(\RV{Y}^D_{i d_i})_{i\in \mathbb{N}}|\RV{W}}(A|d)\\
    &=\prob{P}_\alpha^{(\RV{Y}_{i d_i})_{i\in \mathbb{N}}|\RV{WD}}(A|w,e)\\
    &= \prob{P}_\alpha^{(\RV{Y}_{i d_i})_{i\in \mathbb{N}}|\RV{W}(\RV{D}_{i d_i})_{\mathbb{N}})}(A|w,(e_{i d_i})_{i\in \mathbb{N}})\\
    &= \prob{P}_\alpha^{\RV{Y}|\RV{WD}}(A|w,(e_{i d_i})_{i\in \mathbb{N}})\\
    &= \prob{P}_\alpha^{\RV{Y}|\RV{WD}}(A|w,(d_i)_{i\in\mathbb{N}})
\end{align}

It remains to be shown that $\RV{Y}^D$ is invariant to finite permutations within rows. Consider some finite permutation within columns $\eta:\mathbb{N}\times D\to \mathbb{N}\times D$, note that $e_{\eta(i,j)}=j$ and hence $(e_{\eta(i,j)})_{i\in\mathbb{N},j\in D}=e$. Thus
\begin{align}
    \prob{P}_\alpha^{(\RV{Y}^D_{\eta_(i,j)})_{\mathbb{N}\times D}|\RV{W}}(A|w) &= \prob{P}_\alpha^{(\RV{Y}^D)_{\mathbb{N}\times D}|\RV{W}}\text{Swap}_{\eta}(A|w)\\
    &= \prob{P}_\alpha^{\RV{Y}|\RV{WD}}\text{Swap}_{\eta}(A|w,e)&\text{from Eq. }\eqref{eqApp:rel_bet_y_yd}\\
    &= \prob{P}_\alpha^{\RV{Y}_\eta|\RV{WD}}(A|w,e)\\
    &= \prob{P}_\alpha^{\RV{Y}|\RV{WD}_{\eta^{-1}}}(A|w,e)&\text{by exchange commutativity}\\
    &= \prob{P}_\alpha^{\RV{Y}|\RV{WD}}(A|w,(e_{\eta^{-1}(i,j)})_{i\in \mathbb{N},j\in D})\\
    &= \prob{P}_\alpha^{\RV{Y}|\RV{WD}}(A|w,e)\\
    &= \prob{P}_\alpha^{(\RV{Y}^D_{ij})_{\mathbb{N}\times D}|\RV{W}}(A|w)&\text{from Eq. }\eqref{eqApp:rel_bet_y_yd}
\end{align}

If:
We construct a conditional probability according to Definition \ref{def:tab_cd} and verify that it satisfies IO contractibility.

Suppose 
\begin{align}
    \prob{P}_\alpha^{\RV{Y}|\RV{WD}} &= \tikzfig{lookup_representation_kernel}
\end{align}
where $\prob{P}_\alpha^{\RV{Y}^D|\RV{W}}$ satisfies Equation \eqref{eqApp:col_exch}.

Consider any two $d,d'\in D^{\mathbb{N}}$ such that for some $S,T\subset\mathbb{N}$ with $|S|=|T|=n$, $d_S=d'_T$. Let $S\leftrightarrow T$ be the transposition that swaps the $i$th element of $S$ with the $i$th element of $T$ for all $i$.
\begin{align}
    \prob{P}_\alpha^{\RV{Y}_S|\RV{WD}}(\bigtimes_{i\in [n]} A_i|w,d) &= \prob{P}_\alpha^{(\RV{Y}^D_{i d_i})_{i\in S}|\RV{W}} (\bigtimes_{i\in [n]} A_i|w)\\
    &= \prob{P}_\alpha^{(\RV{Y}^D_{S\leftrightarrow T(i) d_i})_{i\in S}|\RV{W}} (\bigtimes_{i\in [n]} A_i|w)\\
    &= \prob{P}_\alpha^{(\RV{Y}^D_{i d_{S\leftrightarrow T(i)}})_{i\in T}|\RV{W}} (\bigtimes_{i\in [n]} A_i|w)\\
    &= \prob{P}_\alpha^{(\RV{Y}^D_{i d'_{i}})_{i\in T}|\RV{W}} (\bigtimes_{i\in [n]} A_i|w)\\
    &=  \prob{P}_\alpha^{\RV{Y}_T|\RV{WD}}(\bigtimes_{i\in [n]} A_i|w,d')
\end{align}
and, in particular, taking $T=[n]$
\begin{align}
    &= \prob{P}_\alpha^{\RV{Y}_{[n]}|\RV{WD}} (\bigtimes_{i\in [n]} A_i|w,d')
\end{align}
but $d'$ is an arbitrary sequence such that the $T$ elements match the $S$ elements of $d$, so this holds for any other $d''$ whose $T$ elements also match the $S$ elements of $d$. That is
\begin{align}
    \prob{P}_\alpha^{\RV{Y}_S|\RV{WD}}(\bigtimes_{i\in [n]} A_i|w,d)&= (\prob{P}_\alpha^{\RV{Y}_{[n]}|\RV{WD}_{[n]}}\otimes \mathrm{Del}_{D^{\mathbb{N}}}) (\bigtimes_{i\in [n]} A_i|w,d')
\end{align}
so $\kernel{K}$ is IO contractible by Theorem \ref{th:equal_of_condits}.
\end{proof}

As a consequence of Lemma \ref{th:table_rep_kernel} along with De Finetti's representation theorem, we can say that given $(\prob{P}_{\cdot},\RV{D},\RV{Y})$ IO contractible, conditioning on $\RV{H}$ renders the columns of $\RV{Y}^D$ independent and identically distributed.

\begin{lemma}\label{lem:ciid_yd}
Suppose a sequential input-output model $(\prob{P}_{\cdot},\RV{D},\RV{Y})$ is given with $D$ countable, $\RV{D}$ infinitely supported over some $\RV{W}$ and $(\prob{P}_\cdot,\RV{D},\RV{Y})$ IO contractible over the same $\RV{W}$. Then, letting $\RV{H}$ be the directing random conditional of $(\prob{P}_{\cdot},\RV{D},\RV{Y})$ (Definition \ref{def:dir_rand_cond}) and $\RV{Y}^D_{iD}:=(\RV{Y}^D_{ij})_{j\in D}$, we have for all $i\in\mathbb{N}$, $\RV{Y}^D_{iD}\CI^e_{\prob{P}_{\cdot}} (\RV{Y}^D_{\mathbb{N}\setminus\{i\}D},\RV{W},\text{id}_C) | \RV{H}$ and
\begin{align}
    \prob{P}_C^{\RV{Y}^D_{iD}|\RV{H}}(A|\nu) \overset{\prob{P}_\alpha}{\cong} \nu(A)
\end{align}
\end{lemma}

\begin{proof}
Fix $w\in W$ and consider $\prob{P}_{\alpha,w}^{\RV{Y}^D}:= \prob{P}_{\alpha}^{\RV{Y}^D|\RV{W}}(\cdot|w)$. From Lemma \ref{th:table_rep_kernel}, we have the exchangeability of the sequence $(\RV{Y}^D_{1D},\RV{Y}^D_{2D},...)$ with respect to $(\prob{P}_{\alpha,w},\Omega,\sigalg{F})$ as a special case of the invariance of $\prob{P}_\alpha^{(\RV{Y}^D_{ij})_{\mathbb{N}\times D}|\RV{W}}$ to permutations of rows. By the column exchangeability of $\prob{P}_{\alpha,w}^{\RV{Y}^D}$, from \citet[Prop. 1.4]{kallenberg_basic_2005} (where $\RV{H}$ is precisely what Kallenberg calls the directing random measure)
\begin{align}
    \prob{P}_{\alpha,w}^{\RV{Y}^D|\RV{H}} &= \tikzfig{de_finetti_conditional}
\end{align}
Because the right hand side does not depend on $w$, we can say
\begin{align}
    \prob{P}_{\alpha}^{\RV{Y}^D|\RV{HW}} &= \tikzfig{de_finetti_conditional_erase}
\end{align}
and because it also does not depend on $\alpha$ we have $\RV{Y}^D\CI^e_{\prob{P}_{\cdot}} (\RV{W},\text{id}_C) | \RV{H}$. Further application of \citet[Prop. 1.4]{kallenberg_basic_2005} yields $\RV{Y}^D_{iD}\CI^e_{\prob{P}_{\cdot}} (\RV{Y}^D_{\mathbb{N}\setminus\{i\}D},\RV{W}) | (\RV{H},\text{id}_C)$ and
\begin{align}
    \prob{P}_\alpha^{\RV{Y}^D_{iD}|\RV{H}}(A|\nu) \overset{\prob{P}_\alpha}{\cong} \nu(A)
\end{align}
Again, the right hand side does not depend on $\alpha$, which yields $\RV{Y}^D_{iD}\CI^e_{\prob{P}_{\cdot}} (\RV{Y}^D_{\mathbb{N}\setminus\{i\}D},\RV{W},\text{id}_C) | \RV{H}$.
\end{proof}


\begin{reptheorem}{th:any_infinite_sequence}
Suppose a sequential input-output model $(\prob{P}_\cdot,\RV{D},\RV{Y})$ is given with $D$ countable,  $\RV{D}$ infinitely supported and for some $\RV{W}$, $\prob{P}_\alpha^{\RV{Y}|\RV{WD}}$ is IO contractible for all $\alpha$. Consider an infinite set $A\subset \mathbb{N}$, and let $\RV{D}_A:=(\RV{D}_i)_{i\in A}$ and $\RV{Y}_A:=(\RV{Y}_i)_{i\in A}$. Then $\RV{H}_A$, the directing random conditional of $(\prob{P}_{\cdot},\RV{D}_A,\RV{Y}_A)$ is almost surely equal to $\RV{H}$, the directing random conditional of $(\prob{P}_{\cdot},\RV{D},\RV{Y})$.
\end{reptheorem}

\begin{proof}
The strategy we will pursue is to show that an arbitrary subsequence of $(\RV{D}_i,\RV{Y}_i)$ pairs induces a random contraction of the rows of $\RV{Y}^D$. Then we show that the contracted version of $\RV{Y}^D$ has the same distribution as the original, and consequently the normalised partial sums converge to the same limit.

Define $\RV{Y}^{D,A}$ as the tabulated conditional of $(\RV{D}_A,\RV{Y}_A)$, i.e. let $\#^{A,k}_j$ be the count restricted to $A$:
\begin{align}
    \#^{A,k}_j := \sum_{i\in A}^{k-1} \llbracket \RV{D}_i = j \rrbracket
\end{align}
then
\begin{align}
    \RV{Y}^{D,A}_{ij} &:= \sum_{k\in A} \llbracket\#^{A,k}_j=i-1\rrbracket\llbracket \RV{D}_k=j\rrbracket \RV{Y}_k\\
        &= \sum_{k\in A} \llbracket\#^{A,k}_j=i-1\rrbracket\llbracket \RV{D}_k=j\rrbracket \RV{Y}^D_{\RV{R}_k j}
\end{align}
That is, defining $\RV{Q}:\mathbb{N}\to \mathbb{N}$ by $i\mapsto \sum_{k\in A} \llbracket\#^{A,k}_j=i-1\rrbracket\llbracket \RV{D}_k=j\rrbracket \RV{R}_k$ then
\begin{align}
    \RV{Y}^{D,A}_{ij} &= \RV{Y}^D_{\RV{Q}(i) j}\label{eqApp:random_contraction}
\end{align}
where $\RV{Q}(i)\in \mathbb{N}$ by the assumption that each value of $D$ occurs infinitely often in $A$ (otherwise $\RV{Q}(i)$ might be 0).

Equation \eqref{eqApp:random_contraction} is what is meant by ``the subsequence $(\RV{D}_A,\RV{Y}_A)$ induces a random contraction over the rows of $\RV{Y}^D$''. We will now show that $\RV{Y}^{D,A}$ has the same distribution as $\RV{Y}^D$.

Let $\text{con}_{q}:Y^{\mathbb{N}\times D}\kto Y^{\mathbb{N}\times D}$ be the Markov kernel associated with the function that sends $(\RV{Y}^D_{ij})_{i\in \mathbb{N},j\in D}$ to $(\RV{Y}^D_{q(i)j})_{i\in \mathbb{N},j\in D}$. Then for any $B\in \sigalg{Y}^{\mathbb{N}\times D}$, $w,q$:
\begin{align}
    \prob{P}_\alpha^{\RV{Y}^{D,A}|\RV{WQ}}(B|w,q) &= \prob{P}_\alpha^{\RV{Y}^D|\RV{W}}\text{con}_q(B|w)\\
    &= \prob{P}_\alpha^{\RV{Y}|\RV{WD}}\text{con}_q(B|w,e)&\text{by Eq.} \eqref{eqApp:rel_bet_y_yd}\\
    &= \prob{P}_\alpha^{\RV{Y}|\RV{WD}}(B|w,e)&\text{by Theorem }\ref{th:equal_of_condits}\\
    &= \prob{P}_\alpha^{\RV{Y}^D|\RV{W}}(B|w)&\text{by Eq.} \eqref{eqApp:rel_bet_y_yd}\label{eqApp:equal_of_tabs}
\end{align}

Finally, take $\RV{H}_A$ the directing random measure of $\RV{Y}^{D,A}$. We conclude from the equality Eq. \eqref{eqApp:equal_of_tabs} and from the fact that there is a one-to-one map from directing random measures to exchangeable distributions that $\RV{H}_A\overset{\prob{P}_\alpha}{\cong} \RV{H}$.
\end{proof}

The following is a technical lemma that will be used in Theorem \ref{th:ciid_rep_kernel}.

\begin{lemma}\label{lem:hw_interchange}
Suppose a sequential input-output model $(\prob{P}_{\cdot},\RV{D},\RV{Y})$ is given with $D$ countable, $\RV{D}$ infinitely supported over $\RV{W}$, for some $\RV{W}$, $\prob{P}_\alpha^{\RV{Y}|\RV{WD}}$ is IO contractible for all $\alpha$ and for all $\alpha$
\begin{align}
    \prob{P}_\alpha^{\RV{Y}|\RV{WD}} &= \tikzfig{lookup_representation_kernel}\label{eq:lup_rep_kernel_2}
\end{align}
then $\RV{Y}\CI^e_{\prob{P}_{\cdot}} \RV{W} | (\RV{H},\RV{D},\text{id}_C)$ and for all $\alpha$
\begin{align}
    \prob{P}_{\alpha}^{\RV{Y}|\RV{HD}} &= \tikzfig{lookup_representation_kernel_h}
\end{align}
\end{lemma}

\begin{proof}
We show that the function that maps the variables $\RV{Y}$ and $\RV{D}$ to $\RV{H}$ also maps $\RV{Y}^D$ and the constant $e\in D^{\mathbb{N}}$ to $\RV{H}'$ with $\RV{H}'\overset{\prob{P}_{\cdot}}{\cong} \RV{H}$, and the result follows from disintegration along with a conditional independence given by Lemma \ref{th:table_rep_kernel}.

$\RV{Y}^D$ is a function of $\RV{Y}$ and $\RV{D}$ (see Definition \ref{def:tab_cd}) and $\RV{H}$ is a function of $\RV{Y}^D$. Say $f:Y\times D\to H$ is such that $\RV{H}=f(\RV{Y},\RV{D})$ (see Definition \ref{def:dir_rand_meas}). Because $\RV{H}=f(\RV{Y},\RV{D})$, we have $\RV{H}\CI^e_{\prob{P}_C} (\RV{W},\text{id}_C)|(\RV{Y},\RV{D})$. Thus
\begin{align}
    \prob{P}_\alpha^{\RV{YH}|\RV{WD}} &= \tikzfig{lookup_representation_kernel_joint}\label{eqApp:luprep_joint}
\end{align}
For a sequence $d\in D^{\mathbb{N}}$ where each $j\in D$ occurs infinitely often, take $[d=j]_i$ to be the $i$th coordinate of $d$ equal to $j\in D$ and $\#_{[d=j]_i}$ to be the position in $d$ of $[d=j]_i$. Concretely, $f$ is given by
\begin{align}
    f(y,d) &= \bigtimes_{j\in D} A_j \mapsto \lim_{n\to \infty} \frac{1}{n}\sum_{i=1}^n \prod_{j\in D} \mathds{1}_{A_j}(y_{\#_{[d=j]_i}})\\
    &=: f_d(y)
\end{align}
where the limit exists. Note that for $y^D\in Y^{D\times\mathbb{N}}$ we have
\begin{align}
    f_d\circ \mathrm{lu}(y^D,d) &= \bigtimes_{j\in D} A_j \mapsto \lim_{n\to \infty} \frac{1}{n}\sum_{i=1}^n \prod_{j\in D} \mathds{1}_{A_j}(y^D_{\#_{[d=j]_i} j})
\end{align}
Let $g:=(y^D,d)\mapsto f_d\circ \mathrm{lu}(y^D,d)$ for some $d\in D^{\mathbb{N}}$ where each $j\in D$ occurs infinitely often.

We aim to show that $g(\RV{Y}^D,d)\overset{\prob{P}_\alpha}{\cong} g(\RV{Y}^D,d')$ for all $d,d'\in D^{\mathbb{N}}$ such that each $j\in D$ occurs infinitely often.

Consider, for arbitrary $A\in \sigalg{Y}^D$
\begin{align}
    \prob{P}_\alpha (g(\RV{Y}^D,d)(A)\yields g(\RV{Y}^D,d')(A)) &= \int_H \prob{P}_\alpha^{\mathrm{Id}_{\Omega}|\RV{H}}(g(\RV{Y}^D,d)(A)\yields g(\RV{Y}^D,d')(A)|\nu)\prob{P}_\alpha^{\RV{H}}(\mathrm{d}\nu)
\end{align}

Note that
\begin{align}
     \prob{P}_\alpha^{\mathrm{Id}_{\Omega}|\RV{H}}(g(\RV{Y}^D,d)(A)\yields \nu(A)|\nu) &= \prob{P}_\alpha^{\RV{Y}^D|\RV{H}}(\lim_{n\to\infty}\frac{1}{n}\sum_{i=1}^n \prod_{j\in D} \mathds{1}_{A_j}(y^D_{\#_{[d=j]_i},j})\yields \nu(A)| \nu)\prob{P}_\alpha^{\RV{H}}(\mathrm{d}\nu)
\end{align}
by independent permutability of the rows of $\RV{Y}^D$ (Lemma \ref{th:table_rep_kernel}), for each row we can send $\#_{[d=j]_i}$ to $i$ and obtain
\begin{align}
    \prob{P}_\alpha^{\RV{Y}^D|\RV{H}}(\lim_{n\to\infty}\frac{1}{n}\sum_{i=1}^n \prod_{j\in D} \mathds{1}_{A_j}(y^D_{\#_{[d=j]_i},j})\yields \nu(A)| \nu)\prob{P}_\alpha^{\RV{H}}(\mathrm{d}\nu) &= \prob{P}_\alpha^{\RV{Y}^D|\RV{H}} (\lim_{n\to\infty}\frac{1}{n}\sum_{i=1}^n \prod_{j\in D} \mathds{1}_{A_j}(y^D_{i,j})\yields \nu(A)| \nu)\\
    &= \prob{P}_\alpha^{\RV{Y}^D_{iD}|\RV{H}} (\lim_{n\to\infty}\frac{1}{n}\sum_{i=1}^n \mathds{1}_{A}(y^D_{i,D})\yields \nu(A)| \nu)
\end{align}
but by Lemma \ref{lem:ciid_yd}, the sequence $(\RV{Y}^D_{iD})_{i\in \mathbb{N}}$ are mutually independent conditional on $\RV{H}$ and for all $\alpha$, $\prob{P}_\alpha^{\RV{Y}_{iD}|\RV{H}}(A|\nu)\overset{\prob{P}_C}{\cong}\nu(A)$. Thus, by the law of large numbers
\begin{align}
    \prob{P}_\alpha^{\RV{Y}^D|\RV{H}} (\lim_{n\to\infty}\frac{1}{n}\sum_{i=1}^n \mathds{1}_{\prod_{j\in D} A_j}(y^D_{i,D})\yields \nu(A)| \nu)&= 1
\end{align}
which implies
\begin{align}
     &\phantom{=}\int_H \prob{P}_\alpha^{\mathrm{Id}_{\Omega}|\RV{H}}(g(\RV{Y}^D,d)(A)\yields g(\RV{Y}^D,d')(A)|\nu)\prob{P}_\alpha^{\RV{H}}(\mathrm{d}\nu) \\
     &= \int_H \prob{P}_\alpha^{\mathrm{Id}_{\Omega}|\RV{H}}(g(\RV{Y}^D,d)(A)\yields \nu(A) \cap  g(\RV{Y}^D,d')(A)\yields \nu(A)|\nu)\prob{P}_\alpha^{\RV{H}}(\mathrm{d}\nu)\\
    &= 1
\end{align}

Because this holds for all $A$,
\begin{align}
    g(\RV{Y}^D,d)&\overset{\prob{P}_\alpha}{\cong} g(\RV{Y}^D,d') & \text{as this holds for all }A
\end{align}
And, as a consequence, defining
\begin{align}
    i:(y^d,d,d')\mapsto (\mathrm{lu}(\RV{Y}^D,d),g(\RV{Y}^D,d'))
\end{align}
we have
\begin{align}
    i(y^d,d,d) &\overset{\prob{P}_\alpha}{\cong} i(y^d,d,d')
\end{align}
which in turn implies the almost sure equality of the associated Markov kernels:
\begin{align}
     \tikzfig{hw_interchange_2} &= \tikzfig{hw_interchange_3}
\end{align}
but we also have, by the definitions of $f$ and $g$
\begin{align}
    \tikzfig{hw_interchange_1} &=  \tikzfig{hw_interchange_2}
\end{align}
we can therefore write $\prob{P}_\alpha^{\RV{YH}|\RV{WD}}$ as
\begin{align}
    &\phantom{=} \tikzfig{lookup_representation_kernel_joint}\\
    &= \tikzfig{lookup_representation_kernel_joint_half}\\
    &=: \tikzfig{lookup_representation_kernel_joint_2}
\end{align}
because $\RV{H}$ is a deterministic function of $\RV{Y}^D$, this implies
\begin{align}
    \prob{P}_\alpha^{\RV{YH}|\RV{WD}} &= \tikzfig{lookup_representation_kernel_joint_2_subbed}\label{eq:sub_in_lookup}
\end{align}

Noting that $\kernel{F}_h\otimes\mathrm{Del}_W = \prob{P}_\alpha^{\RV{H}|\RV{Y}^D\RV{W}}$
\begin{align}
    \prob{P}_\alpha^{\RV{Y}^D\RV{H}|\RV{W}} &= \tikzfig{p_yd_on_w} \\
    &= \tikzfig{yd_h_on_w_invert} \label{eq:yd_on_h}
\end{align}
and so by substituting Equation \eqref{eq:yd_on_h} into \eqref{eq:sub_in_lookup}
\begin{align}
    \prob{P}_\alpha^{\RV{YH}|\RV{WD}} &= \tikzfig{lookup_representation_kernel_joint_3}
\end{align}
From Lemma \ref{th:table_rep_kernel} we also have $\RV{Y}^D\CI^e_{\prob{P}_C} (\RV{W},\text{id}_C)|\RV{H}$ , so
\begin{align}
    \prob{P}_\alpha^{\RV{YH}|\RV{WD}} &= \tikzfig{lookup_representation_kernel_joint_4}\label{eqApp:disintegrate_twice}
\end{align}
and so by higher order conditionals $\RV{Y}\CI^e_{\prob{P}_C} \RV{W} | (\RV{H},\RV{D},\text{id}_C)$ and
\begin{align}
    \prob{P}_{\alpha}^{\RV{Y}|\RV{HD}} &= \tikzfig{lookup_representation_kernel_h}
\end{align}
Because the right hand side does not depend on $\alpha$, we finally have $\RV{Y}\CI^e_{\prob{P}_C} (\RV{W},\text{id}_C) | (\RV{H},\RV{D})$ and the result
\begin{align}
    \prob{P}_{C}^{\RV{Y}|\RV{HD}} &= \tikzfig{lookup_representation_kernel_h}
\end{align}
Furthermore, by marginalising the right hand side of Equation \ref{eqApp:disintegrate_twice} we have
\begin{align}
    \prob{P}_\alpha^{\RV{H}|\RV{WD}} &= \tikzfig{lookup_representation_kernel_joint_4_marged}
\end{align}
Hence $\RV{H} \CI^e_{\prob{P}_C} \RV{D} | (\RV{W},\text{id}_C)$.
\end{proof}

\subsection{Representation theorem}\label{sec:io_contract_models}

This is the proof of the main result from Section \ref{sec:evaluating_decisions}, Theorem \ref{th:ciid_rep_kernel}. 

\begin{reptheorem}{th:ciid_rep_kernel}
Suppose a sequential input-output model $(\prob{P}_C,\RV{D},\RV{Y})$ with sample space $(\Omega,\sigalg{F})$ is given with $D$ countable and $\RV{D}$ infinitely supported. Then the following are equivalent:
\begin{enumerate}
    \item There is some $\RV{W}$ such that $\prob{P}_\alpha^{\RV{Y}|\RV{WD}}$ is IO contractible for all $\alpha$
    \item For all $i$, $\RV{Y}_i\CI^e_{\prob{P}_C} (\RV{Y}_{\neq i},\RV{D}_{\neq i},\text{id}_C)|(\RV{H},\RV{D}_i)$ and for all $i,j$ $$\prob{P}_C^{\RV{Y}_i|\RV{H}\RV{D}_i}\overset{\prob{P}_\alpha^{\RV{D}_i|\RV{H}}}{\cong}\prob{P}_C^{\RV{Y}_j|\RV{H}\RV{D}_j}$$
    \item There is some $\kernel{L}:H\times X\kto Y$ such that $$\prob{P}_C^{\RV{Y}|\RV{HD}}= \tikzfig{do_model_representation_conditional}$$
\end{enumerate}
\end{reptheorem}

\begin{proof}
As a preliminary, we will show
\begin{align}
    \kernel{F}_{\mathrm{lu}} &= \tikzfig{lookup_rep_intermediate_kernel}\label{eqApp:ev_alternate_rep}
\end{align}
where  $\mathrm{lus}:D\times Y^D\to Y$ is the single-shot lookup function
\begin{align}
    ((y_i)_{i\in D},d)\mapsto y_d
\end{align}

Recall that $\mathrm{lu}$ is the function
\begin{align}
    ((d_i)_\mathbb{N},(y_{ij})_{i,j\in \mathbb{N}\times D})&\mapsto (y_{i d_i})_{i\in \mathbb{N}}
\end{align}
By definition, for any $\{A_i\in\sigalg{Y}|i\in \mathbb{N}\}$
\begin{align}
    \kernel{F}_{\mathrm{lu}}(\bigtimes_{i\in \mathbb{N}}A_i|(d_i)_\mathbb{N},(y_{ij})_{i\in \mathbb{N}\times D}) &= \delta_{(y_{i d_i})_{i\in \mathbb{N}}}(\bigtimes_{i\in \mathbb{N}}A_i)\\
        &= \prod_{i\in \mathbb{N}} \delta_{y_{i d_i}} (A_i)\\
        &= \prod_{i\in \mathbb{N}} \kernel{F}_{\text{evs}} (A_i|d_i,(y_{ij})_{j\in D})\\
        &= \left(\bigotimes_{i\in\mathbb{N}} \kernel{F}_{\mathrm{evs}} \right)(\bigtimes_{i\in \mathbb{N}}A_i|(d_i)_\mathbb{N},(y_{ij})_{i,j\in \mathbb{N}\times D})
\end{align}
which is what we wanted to show.

(1)$\implies$(3):
From Lemma \ref{th:table_rep_kernel}, we have some $\RV{Y}^D$ such that
\begin{align}
    \prob{P}_{\alpha}^{\RV{Y}|\RV{WD}} &= \tikzfig{lookup_representation_kernel}
\end{align}
and by Lemma \ref{lem:ciid_yd}
\begin{align}
    \prob{P}_{C}^{\RV{Y}^D|\RV{H}} &= \tikzfig{de_finetti_conditional_w_upd}\label{eqApp:df_rep_mu}\\
\end{align}

By Lemma \ref{th:table_rep_kernel}, for each $w\in W$
\begin{align}
    \prob{P}_{\alpha}^{\RV{Y}|\RV{WD}} &= \tikzfig{lookup_representation_kernel}
\end{align}
and so by Lemma \ref{lem:hw_interchange}
\begin{align}
    \prob{P}_{C}^{\RV{Y}|\RV{HD}} &= \tikzfig{lookup_representation_kernel_h}\label{eqApp:lu_rep_h}
\end{align}

We can substitute Equations \eqref{eqApp:df_rep_mu} and \eqref{eqApp:ev_alternate_rep} into \eqref{eqApp:lu_rep_h} for
\begin{align}
    \prob{P}_{C}^{\RV{Y}|\RV{HD}} &= \tikzfig{do_model_representation_conditional}
\end{align}

(3)$\implies$ (2):
If
\begin{align}
    \prob{P}_{C}^{\RV{Y}|\RV{HD}} &= \tikzfig{do_model_representation_conditional}
\end{align}
then by the definition of higher order conditionals, for any $i\in \mathbb{N}$ and any $\alpha\in C$
\begin{align}
    \prob{P}_C^{\RV{Y}_i|\RV{HD}_i\RV{Y}_{\neq i}\RV{D}_{\neq i}} &\overset{\prob{P}_C}{\cong} \kernel{L}\otimes \text{Del}_{Y^{\mathbb{N}}\times X^{\mathbb{N}}}
\end{align}
hence $\RV{Y}_i\CI^e_{\prob{P}_C} (\RV{Y}_{\neq i},\RV{D}_{\neq i},\text{id}_C)|(\RV{H},\RV{D}_i)$

(2)$\implies$ (1):
Take $\RV{W}:=\RV{H}$. Because we assume $\RV{Y}_i\CI^e_{\prob{P}_C} (\RV{Y}_{[1,i)},\RV{D}_{[1,i),\text{id}_C})|(\RV{H},\RV{D}_i)$ we can take $\kernel{L}:= \RV{H}^Y_X = \prob{P}_\alpha^{\RV{Y}_i|\RV{H}\RV{X}_i}$ for all $i, \alpha$ (existence given by Theorem \ref{th:repr_cond}) and
\begin{align}
    \prob{P}_C^{\RV{Y}_i|\RV{HD}_i\RV{Y}_{[1,i)}\RV{D}_{[1,i)}} &\overset{\prob{P}_C}{\cong} \kernel{L}\otimes \text{Del}_{Y^{i-1}\times X^{i-1}}
\end{align}
by taking the semidirect product of the conditionals
\begin{align}
    \prob{P}_{C}^{\RV{Y}|\RV{HD}} &= \tikzfig{do_model_representation_conditional}\\
    &= \tikzfig{do_model_representation_conditional_permuted}
\end{align}
hence $(\prob{P}_C,\RV{D},\RV{Y})$ is exchange commutative over $\RV{H}$. Furthermore, take $A\subset \mathbb{N}$. Then
\begin{align}
    &\tikzfig{do_model_representation_conditional_deleted}\\
    =& \tikzfig{do_model_representation_conditional_deleted1}
\end{align}
so $(\prob{P}_C,\RV{D},\RV{Y})$ is also local over $\RV{H}$.
\end{proof}

\subsection{Consequences of Theorem \ref{th:ciid_rep_kernel}}\label{sec:data_independent_proofs}

Theorem \ref{th:ciid_rep_kernel} says that a data independent sequential input-output model $(\prob{P}_{\cdot},\RV{D},\RV{Y})$ features conditionally independent and identical response functions $\prob{P}_\alpha^{\RV{Y}_i|\RV{HD}_i}$ for all $\alpha$ if and only if there is some $\RV{W}$ such that $\prob{P}_\alpha^{\RV{Y}|\RV{WD}}$ is IO contractible over $\RV{W}$ for all $\alpha$.

A simple special case to consider is when $\RV{W}$ is single valued -- that is, when $\prob{P}_\alpha^{\RV{Y}|\RV{D}}$ is IO contractible. As Theorem \ref{th:data_ind_CC} shows, this corresponds to the CIIR sequence models where the inputs $\RV{D}$ are unconditionally data-independent and independent of the hypothesis $\RV{H}$. We can also consider the case where $(\prob{P}_{\cdot}, \RV{D},\RV{Y})$ is only exchange commutative over $*$. This corresponds to models where the inputs $\RV{D}$ are data-independent and the hypothesis $\RV{H}$ depends on a symmetric function of the inputs $\RV{D}$ (under some side conditions).


\begin{theorem}[Data-independent IO contractibility]\label{th:data_ind_CC}
Suppose a sequential input-output model $(\prob{P}_{\cdot},\RV{D},\RV{Y})$ with sample space $(\Omega,\sigalg{F})$ is given with $D$ countable and, letting $E\subset D^{\mathbb{N}}$ be the set of all sequences for which each $j\in D$ occurs infinitely often, $\prob{P}_\alpha^{\RV{D}}(E)=1$ for all $\alpha$. Then the following are equivalent:
\begin{enumerate}
    \item $\prob{P}_\alpha^{\RV{Y}|\RV{D}}$ is IO contractible for all $\alpha$
    \item For all $i$, $\RV{Y}_i\CI^e_{\prob{P}_{\cdot}} (\RV{Y}_{\neq i},\RV{D}_{\neq i},\text{id}_C)|(\RV{H},\RV{D}_i)$, for all $i,j,\alpha$ $$\prob{P}_\alpha^{\RV{Y}_i|\RV{H}\RV{D}_i}=\prob{P}_\alpha^{\RV{Y}_j|\RV{H}\RV{D}_j}$$, $\RV{H}\CI^e_{\prob{P}_{\cdot}} \RV{D}|\text{id}_C$ and for all $i$ $\RV{D}_i\CI^e_{\prob{P}_{\cdot}} \RV{D}_{(i,\infty]}) | (\RV{D}_{[1,i)},\text{id}_C)$
    \item There is some $\kernel{L}:H\times X\kto Y$ such that for all $\alpha$, $$\prob{P}_\alpha^{\RV{YH}|\RV{D}}= \tikzfig{do_model_representation_with_h}$$
\end{enumerate}
\end{theorem}

\begin{proof}
See Appendix \ref{sec:data_independent_proofs}.
\end{proof}

While $\prob{P}_{\cdot}^{\RV{Y}|\RV{D}}$ exchange commutative is not necessarily IO contractible, exchange commutativity of this conditional implies IO contractibility over the directing random conditional $\RV{H}$, and thus is sufficient for conditionally independent and identical responses.

\begin{theorem}\label{lem:exch_prod_ciid}
If $\prob{P}_{\alpha}^{\RV{Y}|\RV{D}}$ is exchange commutative, and for each $\alpha$ $\prob{P}_\alpha^{\RV{D}}$ is absolutely continuous with respect to some exchangeable distribution $\prob{Q}_\alpha^{\RV{D}}$ in $\Delta(D^{\mathbb{N}})$ with directing random measure $\RV{F}$ and $\RV{D}$ infinitely supported over $\RV{F}$ with respect to $\prob{Q}_\alpha$ , then $\prob{P}_\alpha^{\RV{Y}|\RV{HD}}$ is IO contractible, where $\RV{H}$ is the directing random conditional for $\prob{P}_\alpha^{\RV{Y}|\RV{D}}$.
\end{theorem}

\begin{proof}
We show that there is an exchangeable distribution for which the relevant conditional automatically satisfies IO contractibility and is almost surely equal to $\prob{P}_\alpha^{\RV{Y}|\RV{GD}}$ for some $\RV{G}$.
\end{proof}

\begin{lemma}[Exchangeably dominated conditionals]\label{lem:dom_cond}
Given $(\prob{P}_C,\Omega,\sigalg{F})$ and variables $\RV{D},\RV{Y}$, if for any $\alpha$ there is some $\prob{Q}_\alpha$ such that $\prob{Q}_\alpha^{\RV{DY}}$ is exchangeable with directing random measure $\RV{G}$, $\RV{D}$ is infinitely supported over $\RV{G}$ with respect to $\prob{Q}_\alpha$ and for any $i$, $\prob{Q}_\alpha^{\RV{Y}_i|\RV{D}\RV{Y}_{\{i\}^{\complement}}}\overset{\prob{P}}{\cong} \prob{P}_\alpha^{\RV{Y}_i|\RV{D}\RV{Y}_{\{i\}^{\complement}}}$ then $\prob{P}_\alpha^{\RV{Y}|\RV{HD}}$ is IO contractible (where $\RV{H}$ is the directing random conditional for $\prob{P}_\alpha^{\RV{Y}|\RV{D}}$).
\end{lemma}

\begin{proof}
By \citet[Prop. 1.4]{kallenberg_basic_2005}, there is a $\RV{G}$ such that $(\RV{D}_i,\RV{Y}_i)\CI^e_{\prob{Q}_C} (\RV{D}_{\{i\}^{\complement}}\RV{Y}_{\{i\}^{\complement}})|(\RV{G},\text{id}_C)$ and for all $i,j$
\begin{align}
    \prob{Q}_\alpha^{\RV{Y}_i\RV{D}_i|\RV{G}} &= \prob{Q}_\alpha^{\RV{Y}_j\RV{D}_j|\RV{G}}\label{eqApp:joint_given_g}
\end{align}

There is some function $f:D^{\mathbb{N}}\times Y^{\mathbb{N}}$ such that $\RV{G}=f(\RV{D},\RV{Y})$, i.e.
\begin{align}
    \prob{Q}_\alpha^{\RV{Y_iG}|\RV{D}\RV{Y}_{\{i\}^{\complement}}} &= \tikzfig{qyg_eq_pyg}\\
                                     &\overset{P}{\cong} \prob{P}_\alpha^{\RV{Y_iG}|\RV{D}\RV{Y}_{\{i\}^{\complement}}}\\
    \implies \prob{Q}_\alpha^{\RV{Y_i}|\RV{G}\RV{D}\RV{Y}_{\{i\}^{\complement}}}&\overset{P}{\cong} \prob{P}_\alpha^{\RV{Y}_i|\RV{G}\RV{D}\RV{Y}_{\{i\}^{\complement}}}\label{eqApp:cond_on_g}
\end{align}

It follows from weak union that
\begin{align}
    \RV{Y}_i\CI^e_{\prob{Q}_C} (\RV{D}_{\{i\}^{\complement}}\RV{Y}_{\{i\}^{\complement}}) | (\RV{D}_i,\RV{G},\text{id}_C)\\
    \iff \prob{P}_\alpha^{\RV{Y}_i|\RV{D}_i\RV{G}\RV{Y}_{\{i\}^{\complement}}\RV{D}_{\{i\}^{\complement}}}(A|d_i,g,d,y) &\overset{P}{\cong} \prob{P}_\alpha^{\RV{Y}_i|\RV{D}_i\RV{G}}(A|d_i,g) & \forall A,d_i,g,d,y,\alpha\label{eqApp:swap_q_for_p}\\
    \implies \RV{Y}_i\CI^e_{\prob{P}_C} (\RV{D}_{\{i\}^{\complement}}\RV{Y}_{\{i\}^{\complement}}) | (\RV{D}_i,\RV{G},\text{id}_C)
\end{align}
where Eq. \eqref{eqApp:swap_q_for_p} follows from Eq. \eqref{eqApp:cond_on_g}.

Finally, from Eq. \eqref{eqApp:joint_given_g} and Eq. \eqref{eqApp:swap_q_for_p}
\begin{align}
    \prob{P}_\alpha^{\RV{Y}_i|\RV{D}_i\RV{G}} &\overset{\prob{P}}{\cong} \prob{P}_\alpha^{\RV{Y}_j\RV{D}_j|\RV{G}}
\end{align}
Thus $(\prob{P}_C,\RV{D},\RV{Y})$ features independent and identical responses conditioned on $\RV{G}$, and by Lemma \ref{lem:ci_drc} it also has independent and identical responses conditioned on $\RV{H}$. Finally, the infinite support of $\RV{D}$ over $\RV{G}$ with respect to $\RV{Q}_\alpha$ implies $\RV{D}$ is also infinitely supported over $\RV{G}$ with respect to $\prob{P}_\alpha$, so by Theorem \ref{th:ciid_rep_kernel} $\prob{P}_\alpha^{\RV{Y}|\RV{HD}}$ is IO contractible.
\end{proof}

\begin{reptheorem}{th:infinite_condition_swaps}
A data-independent sequential input-output model $(\prob{P}_C,\RV{D},\RV{Y})$ features conditionally independent and identical response functions $\prob{P}_\alpha^{\RV{Y}_i|\RV{D}_i\RV{G}}$ with $\RV{D}$ infinitely supported over $\RV{G}$ only if for any sets $A,B\subset \mathbb{N}$ such that $\RV{D}_A$ and $\RV{D}_B$ are also infinitely supported over $\RV{G}$ and any $i,j\in \mathbb{N}$ such that $i\not\in A$, $j\not\in B$, $$\prob{P}_\alpha^{\RV{Y}_i|\RV{D}_i\RV{Y}_A,\RV{D}_A}=\prob{P}_\alpha^{\RV{Y}_j|\RV{D}_j|RV{Y}_B\RV{D}_B}$$.  If in addition each $\prob{P}_\alpha^{\RV{YD}}$ is dominated by some $\prob{Q}_\alpha$ such that $\prob{Q}_\alpha^{\RV{Y}\RV{D}}$ is exchangeable, then the reverse implication also holds.
\end{reptheorem}

\begin{proof}
Only if: By Theorem \ref{th:ciid_rep_kernel} and Lemma \ref{lem:ci_drc}, $\prob{P}_\alpha^{\RV{Y}|\RV{HD}}$ is IO contractible. By Theorem \ref{th:any_infinite_sequence}, $\RV{H}$ is almost surely a function of both $(\RV{D}_A,\RV{Y}_A)$ and $(\RV{D}_B,\RV{Y}_B)$ and, furthermore, $\RV{Y}_i\CI_{\prob{P}_C}^e (\RV{D}_A,\RV{Y}_A)|(\RV{D}_i,\RV{H},\text{id}_C)$, $\RV{Y}_j\CI_{\prob{P}_C}^e (\RV{D}_B,\RV{Y}_B)|(\RV{D}_j,\RV{H},\text{id}_C)$. Hence there is some $f:D^{\mathbb{N}}\times Y^{\mathbb{N}}\to H$ such that for all $E\in \sigalg{Y}, d_i\in D, d\in D^{\mathbb{N}}, y\in Y^{\mathbb{N}}$
\begin{align}
    \prob{P}_\alpha^{\RV{Y}_i|\RV{D}_i\RV{Y}_A,\RV{D}_A}(E|d_i,y,d) &= \prob{P}_\alpha^{\RV{Y}_i|\RV{D}_i\RV{H}}(E|d_i,f(y,d))\\
     &= \prob{P}_\alpha^{\RV{Y}_j|\RV{D}_j\RV{H}}(E|d_i,f(y,d))\label{eqApp:eq_cond}\\
     &= \prob{P}_\alpha^{\RV{Y}_j|\RV{D}_j\RV{Y}_B,\RV{D}_B}(E|d_i,y,d)
\end{align}
Where Eq. \eqref{eqApp:eq_cond} follows from Theorem \ref{th:equal_of_condits}.

If:
By construction
\begin{align}
    \prob{Q}_\alpha^{\RV{Y}_i\RV{D}_i\RV{Y}_{\{i^{\complement}\}}\RV{D}_{\{i^{\complement}\}}}:=\prob{Q}_\alpha^{\RV{D}_i\RV{Y}_{\{i^{\complement}\}}\RV{D}_{\{i^{\complement}\}}}\odot \prob{P}_\alpha^{\RV{Y}_i|\RV{D}_i\RV{Y}_{\{i^{\complement}\}},\RV{D}_{\{i^{\complement}\}}}
\end{align}
is exchangeable, and by domination $\prob{Q}_\alpha^{\RV{Y}_i|\RV{D}_i\RV{Y}_{\{i^{\complement}\}},\RV{D}_{\{i^{\complement}\}}}\overset{\prob{P}}{\cong}\prob{Q}_\alpha^{\RV{Y}_i|\RV{D}_i\RV{Y}_{\{i^{\complement}\}},\RV{D}_{\{i^{\complement}\}}}$. The result follows from Lemma \ref{lem:dom_cond}.
\end{proof}

\begin{theorem}\label{lem:dom_exch_to_IO}
Given $(\prob{P}_cdot,\RV{Y},\RV{D})$, if $\prob{P}_\alpha^{\RV{Y}|\RV{D}}$ is exchange commutative for each $\alpha$, and for each $\alpha$ $\prob{P}_\alpha^{\RV{D}}$ is absolutely continuous with respect to some exchangeable distribution $\prob{Q}_\alpha^{\RV{D}}$ in $\Delta(D^{\mathbb{N}})$ with directing random measure $\RV{F}$, and if $\RV{D}$ is infinitely supported over $\RV{F}$ with respect to $\prob{Q}_\alpha$, then $(\prob{P}_cdot,\RV{Y},\RV{D})$ is IO contractible.
\end{theorem}

\begin{proof}
For each $\alpha$, extend $\prob{Q}_\alpha^{\RV{D}}$ to a distribution on $(\RV{D},\RV{Y})$ by asserting that $\prob{P}_\alpha^{\RV{Y}|\RV{D}}\overset{\prob{Q}_\alpha}{\cong} \prob{Q}_\alpha^{\RV{Y}|\RV{D}}$. Because $\prob{Q}_\alpha^{\RV{D}}$ dominates $\prob{P}_\alpha^{\RV{D}}$, we have in fact $\prob{Q}_\alpha^{\RV{Y}|\RV{D}}\overset{\prob{P}}{\cong}\prob{P}_\alpha^{\RV{Y}|\RV{D}}$

We will show $\prob{Q}_\alpha^{\RV{DY}}$ is unchanged by finite permutations of $(\RV{D}_i,\RV{Y}_i)$ pairs. For some finite permutation $\rho:\mathbb{N}\to\mathbb{N}$:
\begin{align}
    \prob{Q}_\alpha^{\RV{D}_\rho\RV{Y}_\rho} &= \prob{Q}_\alpha^{\RV{D}_\rho\RV{Y}_\rho} (\text{Swap}_{\rho,D^{\mathbb{N}}}\otimes \text{Swap}_{\rho,Y^{\mathbb{N}}})\\
    &= \prob{Q}_\alpha^{\RV{D}}\odot \prob{Q}_\alpha^{\RV{Y}|\RV{D}}(\text{Swap}_{\rho,D^{\mathbb{N}}}\otimes \text{Swap}_{\rho,Y^{\mathbb{N}}})\\
    &= \tikzfig{exch_dom_swap1}\\
    &= \tikzfig{exch_dom_swap2}\\
    &= \tikzfig{exch_dom_swap3}\label{eqApp:exch_comep}\\
    &= \tikzfig{exch_dom_swap4}\label{eqApp:det_comep}\\
    &= \tikzfig{exch_dom_swap5}\label{eqApp:exchep}\\
    &= \prob{Q}_\alpha^{\RV{D}\RV{Y}}
\end{align}
Where line \eqref{eqApp:exch_comep} follows from exchange commutativity, \eqref{eqApp:det_comep} follows from Theorem \ref{th:fong_det_kerns} and the fact that the swap map is deterministic and line \eqref{eqApp:exchep} comes from the exchangeability of $\prob{Q}_\alpha^{\RV{D}}$.

Because $\prob{P}_\alpha^{\RV{D}}$ is dominated by $\prob{Q}_\alpha^{\RV{D}}$ by assumption, we have $\prob{P}_\alpha^{\RV{Y}|\RV{D}} \overset{\prob{P}}{\cong} \prob{Q}_\alpha^{\RV{Y}|\RV{D}}$, which implies $\prob{Q}_\alpha^{\RV{Y}_i|\RV{D}\RV{Y}_{\{i\}^{\complement}}}\overset{\prob{P}}{\cong} \prob{Q}_\alpha^{\RV{Y}_i|\RV{D}\RV{Y}_{\{i\}^{\complement}}}$ and from Lemma \ref{lem:dom_cond} we therefore have $\prob{P}_\alpha^{\RV{Y}|\RV{HD}}$ IO contractible over $\RV{H}$, and from Theorem \ref{th:ciid_rep_kernel} we have $\RV{Y}\CI^e_{\prob{P}_C} \text{id}_C | (\RV{D},\RV{H})$ and so $\prob{P}_\alpha^{\RV{Y}|\RV{HD}}$ IO contractible over $\RV{H}$ also.
\end{proof}

\section{Precedented options}\label{sec:proof_precedent}

\subsection{IO contractibility from diverse precedent}

This is the proof of Theorem \ref{th:latent_to_observable} in Section \ref{sec:precedent}.

\begin{repdefinition}{def:diverse_precedent}
Given a latent CIIR see-do model $(\prob{P}_\cdot,(\RV{E}_i,\RV{X}_i,\RV{Y}_i,\RV{Z}_i)_{i\in\mathbb{N}\cup\{c\}})$ with $E,X,Y$ and $Z$ all discrete, recall $\RV{G}$ is the directing random conditional of $(\prob{P}_\cdot,\RV{Z}_{\mathbb{N}},(\RV{E}_i,\RV{X}_i,\RV{Y}_i)_{i\in \mathbb{N}})$. 

We say that the options $C$ have \emph{diverse precedent} with respect to $(\prob{P}_\cdot,(\RV{E}_i,\RV{X}_i,\RV{Y}_i,\RV{Z}_i)_{i\in\mathbb{N}\cup\{c\}})$ if $\prob{P}_\cdot$ satisfies the diversity condition:
\begin{align}
    \prob{P}_{\alpha}^{\RV{G}^{EX}_{Z}|\RV{G}^{Y}_{EXZ}}(\cdot|g^{Y}_{EXZ}) &\ll U_{\Delta(E)}& \forall \alpha, z, \prob{P}_\alpha-\text{almost all }g^{Y}_{EXZ}\label{eqApp:lebesgue_dom}
\end{align}
as well as the precedent condition:
\begin{align}
    \prob{P}_\alpha^{\RV{E}_c|\RV{G}} &\ll \sum_{z\in Z}\prob{P}_\alpha^{\RV{E}_i|\RV{G}}(\cdot|g)&\prob{P}_\alpha-\text{almost all }g
\end{align}
Where $U_{\Delta(E)}$ is the uniform measure on the $|E-1|$ simplex of discrete probability distributions with $|E|$ outcomes.
\end{repdefinition}

\begin{reptheorem}{th:latent_to_observable}
Given a see-do model $(\prob{P}_\cdot,(\RV{E}_i,\RV{X}_i,\RV{Y}_i,\RV{Z}_i)_{i\in\mathbb{N}\cup\{c\}})$ with $E,X,Y$ and $Z$ all discrete sets, suppose among the observations $i\in \mathbb{N}$ the pairs $(\RV{Z}_i,(\RV{E}_i,\RV{X}_i,\RV{Y}_i))$ share conditionally independent and identical responses and, for all observations and consequences $i\in \mathbb{N}\cup\{c\}$, pairs $(\RV{E}_i,(\RV{X}_i,\RV{Y}_i))$ also share conditionally independent and identical responses. Take $\RV{G}$ to be the directing random conditional of $(\prob{P}_\cdot,\RV{Z}_{\mathbb{N}},(\RV{E}_i,\RV{X}_i,\RV{Y}_i)_{i\in \mathbb{N}})$.

Let $I\subset \Delta(Y)^{XZ}$ be the event $\RV{G}^Y_{Xz}=\RV{G}^Y_{Xz'}$ for all $z,z'\in Z$; i.e. the event that $\RV{Y}_i$ is independent of $\RV{Z}_i$ conditional on $\RV{X}_i$ and $\RV{G}$. Define $\prob{Q}_\alpha\in \Delta(\Omega)$ to be the probability measure such that, for all $A\in \sigalg{F}$
\begin{align}
\prob{Q}_\alpha(A) := \prob{P}_\alpha^{\mathrm{id}_\Omega|\mathds{1}_I\circ \RV{G}}(A|1)
\end{align}
i.e. $\prob{Q}_\alpha$ is $\prob{P}_\alpha$ conditioned on $\RV{G}^Y_{XZ}\in I$, so $\RV{Y}_i\CI^e_{\prob{Q}_\cdot} \RV{Z}_i|(\RV{X}_i,\mathrm{id}_C)$.

If the options $C$ have diverse precedent with respect to $(\prob{Q}_\cdot,(\RV{E}_i,\RV{X}_i,\RV{Y}_i,\RV{Z}_i)_{i\in\mathbb{N}\cup\{c\}})$, then $(\prob{Q}_\cdot,\RV{X},\RV{Y})$ is also IO contractible.
\end{reptheorem}

\begin{proof}
We apply the diversity condition to show that $\RV{Y}_i\CI^e_{\prob{Q}} \RV{E}_i|(\RV{Z}_i,\RV{X}_i,\RV{G},\text{id}_C)$ for $i\in \mathbb{N}$. We then apply the precedent condition to extend this independence to $\RV{Y}_c\CI^e_{\prob{Q}} \RV{E}_c|(\RV{Z}_c,\RV{X}_c,\RV{G},\text{id}_C)$ to complete the proof.

Note that by construction of $\prob{Q}_\alpha$ we have $\RV{Y}_i\CI^e_{\prob{Q}} \RV{Z}_i|(\RV{X}_i,\RV{G},\text{id}_C)$. This in turn implies, for all $\alpha$ the following holds $\prob{Q}_\alpha$-almost surely:
\begin{align}
    \sum_{e\in E} \RV{G}^y_{exz}\frac{\RV{G}^x_{ez}\RV{G}^e_z}{\sum_{e'\in E}\RV{G}^x_{e'z}\RV{G}^{e'}_z}&\overset{\prob{Q}_\alpha}{\cong} \sum_{e\in E} \RV{G}^y_{exz'}\frac{\RV{G}^x_{ez'}\RV{G}^e_{z'}}{\sum_{e'\in E}\RV{G}^x_{e'z'}\RV{G}^{e'}_{z'}}\label{eqApp:polynomial_base}
\end{align}

Conditioning on $\RV{G}^Y_{EXZ}=g^Y_{EXZ}$

\begin{align}
    \sum_{e\in E} g^y_{exz}\frac{\RV{G}^x_{ez}\RV{G}^e_z}{\sum_{e'\in E}\RV{G}^x_{e'z}\RV{G}^{e'}_z}&\overset{\prob{P}_C}{\cong} \sum_{e\in E} \RV{g}^y_{exz'}\frac{\RV{G}^x_{ez'}\RV{G}^e_{z'}}{\sum_{e'\in E}\RV{G}^x_{e'z'}\RV{G}^{e'}_{z'}}\label{eqApp:polynomial}
\end{align}

Eq. \eqref{eqApp:polynomial} defines a polynomial constraint on $\RV{G}^{\RV{Ex}}_{\{z,z'\}}$ for each $x,z,z'$. If $g^y_{exz}=g^y_{e'xz}$ for all $e,e'$ and likewise $g^y_{exz'}=g^y_{e'xz'}$, then this constraint is trivial; it is satisfied for every possible value of $\RV{G}^x_{E\{z,z'\}}$.

We will show that, unless $g^y_{exz}= g^y_{e'xz}$ for all $e,e'$ and $z$, that this constraint is nontrivial for some $z$. Consequently, the set of solutions for $\RV{G}^x_{EZ}$ subject to the restriction $g^y_{exz}\neq g^y_{e'xz}$ has Lebesgue measure 0. We will do this by showing that, assuming $g^y_{exz} > g^y_{e^<xz}$ for some $e,e^<$, we can find alternative realisations of $\RV{G}^e_{z}$ that lead to unequal values of the left hand side of Eq \eqref{eqApp:polynomial} without affecting the right hand side.

Let $g^x_{ez}$ and $g^e_z$ be a possible realisation of $\RV{G}^x_{ez}$ and $\RV{G}^e_z$. Assuming $g^y_{exz} > g^y_{e^<xz}$, either $g^x_{ez}=g^x_{e^<z}$, $g^x_{ez}< g^x_{e^<z}$ or $g^x_{ez}>g^x_{e^<z}$. Consider the first case, and take $g'$ such that $g^{\prime e}_{z}=0.5g^{e}_{z}$ and $g^{\prime e^<}_{z}=g^{e^<}_{z}+0.5g^{e}_{z}$ and equal to $g^{e''}_z$ for all other $e''\in E$. Note that $g^{\prime E}_z$ is also a possible realisation of $\RV{G}^e_z$, as it is everywhere positive and sums to 1, and $g'^{\prime e}_{z}<g^{e}_{z}$ almost surely as $g^{e}_{z}>0$ almost surely. Then
\begin{align}
    \frac{g^x_{ez}g^e_z}{\sum_{e'\in E}g^x_{e'z}g^{e'}_z} &> \frac{g^x_{ez}g^{\prime e}_z}{\sum_{e'\in E}g^x_{e'z}g^{\prime e'}_z}\\
    \frac{g^x_{e^<z}g^{e^<}_z}{\sum_{e'\in E}g^x_{e'z}g^{e'}_z} &< \frac{g^x_{e^<z}g^{\prime e^<}_z}{\sum_{e'\in E}g^x_{e'z}g^{\prime e'}_z}
\end{align}
because by assumption the denominator remains the same. But then
\begin{align}
    \sum_{e\in E} g^y_{exz}\frac{g^x_{ez}^e_z}{\sum_{e'\in E}g^x_{e'z}^{e'}_z}&> \sum_{e\in E} g^y_{exz'}\frac{g^x_{ez}^{\prime e}_{z'}}{\sum_{e'\in E}g^x_{e'z'}^{\prime e'}_{z'}}\label{eqApp:inequality}
\end{align}
because on the right side a smaller term in the sum receives more weight, a larger term receives less weight and all other terms are weighted equally.

Consider $g^x_{ez'}>g^x_{e^<z'}$. Then we still have
\begin{align}
    \frac{g^x_{ez}g^e_z}{\sum_{e'\in E}g^x_{e'z}g^{e'}_z} &> \frac{g^x_{ez}g^{\prime e}_z}{\sum_{e'\in E}g^x_{e'z}g^{\prime e'}_z}\\
    \frac{g^x_{e^<z}g^{e^<}_z}{\sum_{e'\in E}g^x_{e'z}g^{e'}_z} &< \frac{g^x_{e^<z}g^{\prime e^<}_z}{\sum_{e'\in E}g^x_{e'z}g^{\prime e'}_z}
\end{align}
For the second inequality, the right hand numerator grows and the denominator shrinks. For the first, note that
\begin{align}
    \frac{g^x_{ez}g^{\prime e}_z}{\sum_{e'\in E}g^x_{e'z}g^{\prime e'}_z} &= \frac{0.5 g^x_{ez}g^{e}_z}{\sum_{e'\in E}g^x_{e'z}g^{e'}_z - 0.5 g^{e}_z (g^x_{ez} - g^x_{e^< z})}
\end{align}
$g^e_z g^x_{ez}<1$ (an almost sure event) implies that the right hand denominator is greater than $0.5 \sum_{e'\in E}g^x_{e'z}g^{e'}_z$, and hence the right hand side is less than $\frac{g^x_{ez}g^e_z}{\sum_{e'\in E}g^x_{e'z}g^{e'}_z}$.

Thus the conclusion in Eq. \eqref{eqApp:inequality} follows for the same reasons as before. Considering $g^x_{ez'}< g^x_{e^<z'}$, analogous reasoning implies Eq. \eqref{eqApp:inequality} once again.

Thus, unless $g^y_{exz}=g^y_{e'xz}$ for all $e,e'$ and $z$, Eq. \eqref{eqApp:polynomial} implies a nontrivial constraint on $\RV{G}^x_{Ez}$ for some $z$. Thus for some $e,e'$,$z$, $x$ and $y$ the set of solutions $S:=\{g^X_{EZ}|\RV{G}^X_{EZ}=g^X_{EZ}\text{ satisfies Eq. \eqref{eqApp:polynomial} for all }x, z\land g^y_{exz}\neq g^y_{e'xz}\}$ has Lebesgue measure 0 \citep{okamoto_distinctness_1973}, and so by domination
\begin{align}
    \prob{Q}_{\alpha}^{\RV{G}^X_{EZ}|\RV{G}^{XY}_{EZ}}(S|g^{XY}_{EZ}) = 0
\end{align}
On the other hand, by assumption, the set $T:=\{g^E_{z}|\RV{G}^E_z=g^E_z\text{ satisfies Eq. \eqref{eqApp:polynomial}}\}$ has measure 1. Thus we conclude that with the exception of a $\prob{Q}_\alpha$ measure 0 set, $g^y_{exz}=g^y_{e'xz}$. That is, $\RV{Y}_i\CI^e_{\prob{Q}} \RV{E}_i|(\RV{Z}_i,\RV{X}_i,\RV{G},\text{id}_C)$. 

From the diversity condition, we have $\RV{Y}_i\CI^e_{\prob{Q}} \RV{E}_i|(\RV{Z}_i,\RV{X}_i,\RV{G},\text{id}_C)$. By contraction with $\RV{Y}_i\CI^e_{\prob{Q}} \RV{Z}_i|(\RV{X}_i,\RV{G},\text{id}_C)$, we have $\RV{Y}_i\CI^e_{\prob{Q}} (\RV{Z}_i,\RV{E}_i)|(\RV{X}_i,\RV{G},\text{id}_C)$. 

By CIIR of the $(\RV{E}_i|(\RV{X}_i,\RV{Y}_i))$ pairs, we have for all $i$, $\prob{Q}_\alpha^{\RV{Y}_i\RV{X}_i|\RV{E}_i\RV{G}}\overset{\prob{Q}_\alpha^{\RV{E}_i|\RV{G}}}{\cong}\prob{Q}_\alpha^{\RV{Y}_c\RV{X}_c|\RV{E}_c\RV{G}}$. Because we have a representative version $\RV{G}^{XY}_E$ of \prob{Q}_\alpha^{\RV{Y}_i\RV{X}_i|\RV{E}_i\RV{G}} for all $i\in \mathbb{N}$ (Theorem \ref{th:repr_cond}) and precedent implies that any set of measure 0 with respect to  $\prob{Q}_\alpha^{\RV{E}_i|\RV{G}}$ for all $i\in\mathbb{N}$ also has measure 0 with respect to $\prob{Q}_\alpha^{\RV{E}_c|\RV{G}}$, we have
\begin{align}
    \RV{G}^{XY}_{E} \overset{\prob{Q}_\alpha^{\RV{E}_c|\RV{G}}}{\cong} \prob{Q}_\alpha^{\RV{Y}_c\RV{X}_c|\RV{E}_c\RV{G}}
\end{align}
and thus
\begin{align}
    \RV{G}^Y_X \overset{\prob{Q}_\alpha^{\RV{X}_c|\RV{G}}}{\cong} \prob{Q}_\alpha^{\RV{Y}_c|\RV{X}_c\RV{G}}
\end{align}
completing the proof.
\end{proof}

\subsection{Diverse precedent from independent causal mechanisms}\label{app:dp_from_indep}

Here we prove Theorem \ref{th:dp_from_int}.
%  First, a lemma: the image of the uniform measure over joint distribuitons under the map than sends a joint distribution to a conditional distribution is the uniform measure on the space of conditional distributions. This will be used to show that if the distribution of joint distributions is dominated by the uniform measure, then so is the distribution of its conditionals.

% \begin{lemma}[Image of the uniform measure under the conditioning map]\label{lem:image_unif_measure}
% Given $(\mu, \sigalg{F}, \Omega)$ and variables $\RV{X}\in [n]$, $\RV{Y}\in [m]$, let $f_x$ be the \emph{conditioning map} that maps  $\mu^{\RV{XY}}$ to the distribution conditioned on $\RV{X}=x$, $\mu^{\RV{Y}}_x$. That is
% \begin{align}
%     f_x(\mu^{\RV{XY}})(\{y\}) = \frac{\mu^{\RV{XY}}(\{x\},\{y\})}{\mu^{\RV{X}}(\{x\})}
% \end{align}
% with the convention $\frac{0}{0}=1$.

% Take $U_{\Delta(X\times Y)}$ to be the uniform measure on the set of distributions over $X\times Y$. We can identify $\Delta(X\times Y)$ with the $(mn-1)$-simplex with the Borel $\sigma$-algebra, which we will denote $\sigma(\Delta(X\times Y))$. Then for any $A\in \sigma(\Delta(Y))$, $x\in X$, $U_{\Delta(X\times Y)}(f_x^{-1}(A))=U_{\Delta(Y)}(A)$.
% \end{lemma}

% \begin{proof}
% Note that we can write
% \begin{align}
%     f_x^{-1}(A) &= \cup_{c\in (0,1]} \{\mu:\mu^x=c \land \frac{\mu^{x\RV{Y}}}{c}\in A\}\\
%     &:= \cup_{c\in (0,1]}B_c
% \end{align}
% Now
% \begin{align}
%     U_{\Delta(X\times Y)}(f_x^{-1}(A)) = (mn-1)!\int_{f_x^{-1}(A)} \mathrm{d}p_{11}...\mathrm{d}p_{n(m-1)}
% \end{align}

% Define
% \begin{align}
%     \phi_x(p_11,...,\mathrm{d}p_{|X||Y|}):=\sum_{y\in Y} p^{xy}, (p^{xy})_{y\in [|Y|-1]}, (p^{x'y})_{x'\neq x, y\in [m-1]}
% \end{align}
% and observe
% \begin{align}
%     |\det (D\phi)| = 1
% \end{align}
% letting $p^{xY}:=(p^{xy})_{y\in [|Y|-1]}$, $p^x:=\sum_{y\in Y} p^{xy}$, $p^{x^\complement}:=(p^{x'y})_{x'\neq x, y\in Y}$ we have
% \begin{align}
%     U_{\Delta(X\times Y)}(f_x^{-1}(A)) &= (mn-1)!\int_{\phi(\cup_{c\in (0,1]} B_c)} \mathrm{d}p^{xY}\mathrm{d}p^x \mathrm{d}p^{x^\complement}
% \end{align}
% Define $cA:=\{c\mu^{x\RV{Y}}|\mu^{x\RV{Y}}\in A\}$
% \begin{align}
%     U_{\Delta(X\times Y)}(f_x^{-1}(A)) &= (mn-1)!\int_{\cup_{c\in (0,1]} \{c\}\times(1-c)\Delta([n-1][m]) \times cA} \mathrm{d}p^{xY} \mathrm{d}p^{x^\complement} \mathrm{d}p^x  \\
%     &= (mn-1)!\int_0^1 \int_{(1-p^x)\Delta([n-1][m])} \int_{p^xA} \mathrm{d}p^{xY} \mathrm{d}p^{x^\complement} \mathrm{d}p^x\\
%     &= (mn-1)!\int_0^1 \frac{(1-p^x)^{(n-1)m-1}}{((n-1)m-1)!} \cdot \frac{(p^x)^{m-1}U_{\Delta(Y)}(A)}{(m-1)!} dp^x\\
%     &= U_{\Delta(Y)}(A)
% \end{align}
% \end{proof}

\begin{reptheorem}{th:dp_from_int}
Consider a latent CIIR see-do model $(\prob{P}_\cdot,(\RV{E}_i,\RV{X}_i,\RV{Y}_i,\RV{Z}_i)_{i\in\mathbb{N}\cup\{c\}})$ and define $\prob{Q}_\cdot$ as $\prob{P}_\cdot$ conditioned on $\mathds{1}_I=1$ where $\mathds{1}_I:=\llbracket\RV{G}^Y_{Xz}=\RV{G}^Y_{Xz'}\rrbracket$ for all $z,z'\in Z$.

If either of the following hold:
\begin{align}
    \prob{P}^{\RV{G}^E_z|\RV{G}^E_{z'}}(\cdot|g^E_{z'})&\ll U_{\Delta(E)} \text{ almost all }g^E_{z'} &\text{and }\RV{G}^{E}_{Z} \CI^e_{\prob{P}} \RV{G}^{XY}_{EZ}|\mathrm{Id}_C\label{eqApp:dp_from_int1}\\
    \prob{P}^{\RV{G}^X_{Ez}|\RV{G}^X_{Ez'}}(\cdot|g^X_{Ez'})&\ll U_{\Delta(X)}\text{ almost all }g^X_{Ez'}&\text{and }\RV{G}^{X}_{{EZ}}\CI^e_{\prob{P}} \RV{G}^{\RV{Y}}_{\RV{EXZ}}|\mathrm{Id}_C\label{eqApp:dp_from_int2}
\end{align}
then the options $C$ have diverse precedent with respect to $(\prob{Q}_\cdot,(\RV{E}_i,\RV{X}_i,\RV{Y}_i,\RV{Z}_i)_{i\in\mathbb{N}\cup\{c\}})$.
\end{reptheorem}

\begin{proof}
If the conditions on line \eqref{eqApp:dp_from_int1} hold, then we require $(\RV{G}^E_{z'}, \RV{G}^E_{z}) \CI^e_{\prob{P}}  (\RV{G}^{XY}_{EZ},\mathrm{1}_I) | \mathrm{id}_C$ for diverse precedent. $\mathds{1}_I=\llbracket \RV{G}^Y_{Xz}=\RV{G}^Y_{Xz'} \rrbracket$ is determined by $\RV{G}^{\RV{XY}}_{\RV{EZ}}$, so by decomposition it is sufficient to show $(\RV{G}^E_{z'}, \RV{G}^E_{z}) \CI^e_{\prob{P}}  \RV{G}^{XY}_{EZ} | \mathrm{id}_C$, but this follows directly from the assumption $\RV{G}^{\RV{E}}_{\RV{Z}} \CI^e_{\prob{P}} \RV{G}^{\RV{XY}}_{\RV{EZ}}|\mathrm{Id}_C$.

If the conditions on line \eqref{eqApp:dp_from_int2} hold, then we require $(\RV{G}^X_{Ez'}, \RV{G}^X_{Ez}) \CI^e_{\prob{P}}  (\RV{G}^{XY}_{EZ},\mathrm{1}_I) | \mathrm{id}_C$ for diverse precedent. As $\mathds{1}_I$ is also determined by $\RV{G}^Y_{EXZ}$, this follows by an argument analogous to the above.
\end{proof}

\end{document}